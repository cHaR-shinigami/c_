\def\Section#1{\section{#1}\input{Classes/#1}}

A class is used to associate well-defined functionalities with data objects:
the behavior is described by protocols and implemented by procedures.
A class defines a concrete data type, which means that the names
and types of data members are declared along with the class type,
and this information can be used to directly modify an attribute.
Every class is associated with a set of methods, which provide a controlled
mechanism for accessing and updating the state of an instance; however,
concrete data types can be modified directly as their member details are known.

Each value of a class type is called an ``instance''
of that class: it is (pointer to) a structure that
contains the member attributes of that class type.
Every instance has an implicit attribute that specifies the dynamic
type of that object: it is nothing but a pointer to a non-modifiable
structure of function pointers, which is same for all instances
of the same class, but different across multiple classes.
Additionally, the type pointer also creates a static
relationship between a class and its base type, establishing
a type lineage all the way to the root ancestor \tt{Object}.

Influenced by existing object-oriented languages, instances are typically
manipulated via references, guaranteeing a small constant overhead (independent
of structure size) when passed as function arguments or copied as return value.
Another existing convention we shall follow is to start the name of an
object-oriented type with an uppercase letter, and have at least one lowercase
letter in the name (to differentiate them from object-like macros usually named
entirely in uppercase); this practice is consistent with the notion of classes
and interfaces being user defined data types, hence we follow the same naming
scheme as for C\_ type synonyms (a trailing underscore means modifiable).

\note We shall use the term ``object-oriented types''
for collectively referring to both classes and interfaces.

\section{\type{Type} structure}
\def\Subsection#1{\subsection{\idxy{struct Type}{#1}}
\input{Classes/Type structure/#1}}

\idx{struct Type} is a collection of pointers that contain
information about identity, lineage, name, and implementation
of a class (concrete type) or interface (abstract type); here
``implementation'' refers to procedures for a fixed set of protocols.
The name \tt{Type_} is a synonym for \tt{Ptr_(const struct Type)},
which means a modifiable pointer to a non-modifiable structure;
its counterpart \tt{Type} specifies that the pointer is also
non-modifiable, which is same as \tt{Ptr (const struct Type)}.
Each class (concrete type) and interface (abstract type)
is characterized by a non-modifiable \tt{Type} property associated
with it, which is declared as an identifier with external linkage.

\Subsection{self}

\Subsection{base}

\Subsection{name}


\section{\type{Object} class structure}
\def\Subsection#1{\subsection{\idxy{struct Object}{#1}}
\input{Classes/Object class structure/#1}}

The \tt{Object} class declares \idx{struct Object}
with only one member named \tt{type}.
All classes are related to the \tt{Object} class through inheritance: each class
has exactly one base class, and \tt{Object} is the root ancestor of all classes.

\note The name \tt{Object_} is a synonym for \tt{struct Object},
and \tt{Object} is its non-modifiable counterpart.

\Subsection{type}


\Section{Obtaining the type}

\Section{Type inheritance}

\Section{Type relationships}

\section{\tt{Type} methods}
\def\Subsection#1{\subsection{\idxy{struct Type}{#1}}
\index{Type@\tt{Type}!#1@\tt{#1}}\input{Classes/Type methods/#1}}

During instantiation, the \tt{type} attribute inherited from
\tt{Object} class is initialized to point to the \tt{Type}
structure associated with the class or interface being instantiated.
The following subsections describe those \tt{Type} methods
for which a corresponding function pointer is declared in
the \tt{Type} structure, whose value is used for callback.

Except for \tt{validate}, each \tt{Type} method defines a pair of functions,
protocol and procedure: protocol describes the expected behavior with
pre-conditions and post-conditions, whereas its associated procedure invokes
a callback function given by the corresponding member of a \tt{Type}
parameter, or from the \tt{type} attribute of an \tt{Object} pointer.

Except for \tt{comparable}, each subsection lists three declarations.
Prototype declarations uses generic \tt{Void/Void_} pointers for parameters and
return type, which is done to avoid boilerplate type casts during invocations.
Procedure associated with a prototype invokes a
callback function obtained from a type structure.
The callback pointer need not have the same function type as the procedure,
and in all cases, their parameter and return types are different:
prototypes use \tt{Void}/\tt{Void_} pointers, whereas function pointers
in the \tt{Type} structure use \tt{Object}/\tt{Object_} pointers.

When a new object-oriented type is created, one procedure
is declared for each function pointer in the type structure.
If the new type (class or interface) is named as $T$,
then its associated procedures use $T/T$\_ pointers instead of
\tt{Object/Object_} pointers for parameters and return type.
This is done to allow precise type checking of arguments and return value when
these procedures are directly invoked; if they are used as callback via function
pointers in type structure, the arguments and return value are expected to be
\tt{Object/Object_} pointers, which works correctly because both $T$ and \tt{Object}
identify structure types, and their pointers have identical representation.

\subsection{\idxy{struct Type}{validate}}
\head{Declaration}

\tt{Bool_ solver_(Object, validate)(Object *this);}

\head{Description}

The procedure returns \tt{true} if \tt{this} is not null and
\tt{is_type(this->type)} is \tt{true}; otherwise it returns \tt{false}.


\Subsection{init}

\Subsection{free}

\Subsection{compare}

\subsection{\idxy{Type}{comparable}}
\def\Subsubsection#1{\subsubsection{#1}\input{Classes/Type methods/comparable/#1}}

\Subsubsection{Declarations}

\Subsubsection{Pre-conditions}

\enlargethispage*{\baselineskip}
\enlargethispage*{\baselineskip}
\Subsubsection{Procedure}
\pagebreak

\Subsubsection{Invocation}


\Subsection{copy}

\Subsection{read}

\Subsection{write}

\Subsection{parse}

\Subsection{text}

\Subsection{decode}

\Subsection{encode}

\Subsection{add}

\Subsection{sub}

\Subsection{mul}

\Subsection{div}


\section{\tt{Object} class procedures}
\def\Subsection#1{\subsection{\idxy{Object}{#1}}
\input{Classes/Object class procedures/#1}}

The \tt{Object} class provides procedures for each \tt{Type} method that has a
corresponding member in the \tt{Type} structure; these functions are basic stubs
with minimal code, just enough to satisfy the post-conditions imposed by protocols.

The expression \tt{type_(Object)} is a pointer to the \tt{Type} structure
of \tt{Object} class: members of this structure provide access to procedures
implemented by \tt{Object} class, as described in the subsections.
Any class using \tt{Object} as base class inherits these procedures in
its own \tt{Type} structure, unless it overrides them with other functions.

The object-like macros \idx{Object_EXTENDS} and
\idx{SELF_C_EXTENDS} are reserved for implementations of C\_.

\Subsection{validate}

\Subsection{init}

\Subsection{free}

\Subsection{compare}

\Subsection{copy}

\Subsection{read}

\Subsection{write}

\Subsection{parse}

\Subsection{text}

\Subsection{decode}

\Subsection{encode}

\Subsection{add}

\Subsection{sub}

\Subsection{mul}

\Subsection{div}


\Section{Creating a class}

\Section{Implementing procedures}

\Section{More examples}
