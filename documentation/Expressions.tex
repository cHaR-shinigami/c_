\def\Section#1{\section{#1}\input{Expressions/#1}}

This chapter describes several features of C\_ that are syntactically classified
as expressions: each expression has a well-defined type, which can be a complete
type, a function type, or an incomplete type; if the type is not \tt{Void_}
or an incomplete aggregate type, an expression also has a value.
Unlike statements discussed in the previous chapter,
these features can be used as sub-expressions within a larger expression;
an expression becomes a statement when it is ended by a semicolon.
Before proceeding to the main content, it should be mentioned that this chapter
does not contain a complete collection of all expression-like features in C\_;
the rest have been documented in later chapters.

\note ISO C grammar does not allow statements within expressions;
however, the GNU C dialect permits this as a non-standard language extension
that is supported by many compilers (including \tt{gcc} and \tt{clang}).
C\_ features marked with an asterisk (*) are provided
by the reference implementation using this extension;
depending on how a C\_ feature is implemented, this is required to
avoid multiple evaluation of arguments with variably modified types.
Use of such features can generate compiler warnings that can be
disabled with \tt{-Wno-pedantic} for \tt{gcc} and \tt{clang}.

\Section{Constant expressions}

\Section{Compound operators}

\Section{Scalar to text}

\Section{Bit shifting}

\Section{Evaluation}

\Section{Conditionals}

\Section{Generic selections}

\Section{Detecting qualifiers}

\Section{Call stack growth}

\Section{Allocation}

\Section{Input}

\Section{Output}

\Section{Logging}
