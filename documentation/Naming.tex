The reference implementation uses name mangling to generate various
identifiers associated with methods and object-oriented types.
It is also important to discuss how the name
mangling scheme works, for two primary reasons:
firstly, the mangled names are used as identifiers with external linkage,
so other implementations need to follow the same scheme for portability;
secondly, some of the mangled names are used as function identifiers, and
consequently, the predefined identifier \tt{__func__} stores the mangled name.
\tt{__func__} is used as an initializer by \tt{SITE}, which is in turn used by
several diagnostic features, including the \tt{pre_} and \tt{post_} families.
When a pre-condition or post-condition is violated, it is the
mangled function name that gets printed in the error message.
Hence it is important to know the name mangling scheme
for identifying which function a mangled name refers to.

Name mangling is nothing but pasting together two
identifiers with a ``gluing text'' inserted between them.
This technique is well-known and employed in many other programming languages.
For example, Python uses name mangling for class members that are named with
multiple leading underscores and not ending with multiple trailing underscores,
such as \tt{__dob}; if the member \tt{__dob} is an attribute of class
\tt{Person}, then the actual identifier used is \tt{_Person__dob}.
C\_ is influenced by this scheme, though it uses name mangling
for a different purpose: emulating the functionality of namespaces.
However, name mangling by itself cannot capture all the traits of a
proper namespace, and therefore we refer to it as pseudo-namespace.
For instance, multiple classes and interfaces can have
methods with the same name, since the actual identifier
being used is formed by prefixing the class or interface name.

For this scheme to work as expected, it is necessary that two distinct pairs
of \tt{(}$prefix$\tt{,} $name$\tt{)} should not generate the same identifier;
in other words, the mapping must be one-to-one.
The so-called ``gluing text'' inserted between $prefix$ and $name$
is intended to prevent name collision, and one of the naming
restrictions discussed in the introductory chapter requires that
programmers should not declare identifiers with multiple underscores.
Also, $prefix$ and $name$ should not have leading or trailing underscores
(recall that leading underscores have restricted usage in C as well).
For example, if the consider the naming scheme tabulated below,
then \tt{property_(Prefix_, name)} and \tt{property_(Prefix, _name)} both
generate the same identifier \tt{Prefix___name}, leading to name collision.

Assuming that naming restrictions on the use of underscores are
respected by the programmer, the following name mangling scheme should work
as expected in creating pseudo-namespaces (mostly for object-oriented types).

\table{rl@{\qquad}|@{\qquad}l}

\multicolumn{2}{c}{\bf Source text} & {\bf Mangled name}\\

\tt  {method_} & \tt{(} $prefix$ \tt{,} $name$ \tt{)} & $prefix$\tt{__2}$name$\\

\tt  {Method_} & \tt{(} $prefix$ \tt{,} $name$ \tt{)} & $prefix$\tt{__3}$name$\\

\tt   {proxy_} & \tt{(} $prefix$ \tt{,} $name$ \tt{)} & $prefix$\tt{__4}$name$\\

\tt  {solver_} & \tt{(} $prefix$ \tt{,} $name$ \tt{)} & $prefix$\tt{__5}$name$\\

\tt{verifier_} & \tt{(} $prefix$ \tt{,} $name$ \tt{)} & $prefix$\tt{__6}$name$\\

\tt{property_} & \tt{(} $prefix$ \tt{,} $name$ \tt{)} & $prefix$\tt{__}$name$\\

\tt  {Typex}\s & \tt{(} $interface$ \tt{)} & $interface$\tt{__0}\\

\tt  {Typex_}  & \tt{(} $interface$ \tt{)} & $interface$\tt{__0_}\\

\elbat

\head{Recommended practice}

For maximum portability, the number of combined characters
in $prefix$ and $name$ should be less than 29 for method names,
and should not exceed 29 for property names.
This is because the mangled names are used as identifiers with external
linkage for which implementations can consider only the first 31 characters;
in other words, if two distinct identifiers are identical in
their initial 31 characters, then they may be considered as same.
Hence it is recommended that the full length of mangled
name (with the gluing underscores) should not exceed 31.
