The following options have been used with \tt{gcc} to test the
reference implementation and all sample examples in this document;
same set of options is also used by the build script
to generate object files within \tt{object/} directory.

\code{Introduction/Diagnostics/gcc.txt}

\tt{-O3} is an optimization flag that we have chosen not just to improve
runtime efficiency, but also because some diagnostic options come into effect
only when certain optimizations are enabled in conjunction with warning flags.
This ensures a more rigorous diagnosis of our examples,
which is all the more crucial for something as new as C\_.
\tt{-s} is used to reduce size of the executable by stripping non-essential
data pertaining to symbol table and relocation of object code;
this option is not suitable for use with debugging tools.
It should also not be used in conjunction with the \tt{-c} option,
as it removes crucial information required by the linker,
making the object file(s) unlinkable.
Header files placed within the \tt{class/} and \tt{interface/}
directories will be required later in chapters 7 and 8.

The reference implementation relies heavily on macros;
in fact, it is essentially a preprocessing-based transpiler,
and a simple typographic error in a small code can trigger an
avalanche of error messages, which can be overwhelming for beginners,
and mildly annoying for seasoned programmers.
The option \tt{-ftrack-macro-expansion=0} limits each error message to the
macro that is used in the program, skipping any nested expansion of helper
macros that actually produce the erroneous code.
Not using this option is mostly useful for debugging expansions
of low-level iterated composition and metaprogramming macros from
the ellipsis framework, all of which is introduced in chapter 4.

\tt{-Werror} turns all warnings into errors,
excluding any exceptions specified with the \tt{-Wno-error=} option.
This sounds reasonable in a textbook setting, but may not always be
suitable in a production environment where certain warnings are tolerable,
but should not be outright disabled as they are informative to be programmer;
for example, use of non-portable features may be acceptable with a
particular compiler, but it should be reported as a warning.
To quickly get back to the main agenda, we shall skip a thorough explanation
of specific warning options; these are summarized in appendix E.
The last few options starting with \tt{-Wno-} disable the specified
warning; these warn about suspected bugs, but are otherwise perfectly
legal in ISO C, and false positives for the reference implementation.
