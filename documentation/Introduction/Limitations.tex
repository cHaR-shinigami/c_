Emulating an entire dialect with the preprocessor has some inherent drawbacks.
Readers who were able to successfully compile the first
example may have observed some slowness in translation;
a large fraction of this time is spent transpiling C\_ code to C code,
which involves highly resource intensive computations.
Another unimpressive aspect of the reference implementation is
that diagnostic messages (errors and warnings) can be cryptic,
and deciphering the precise cause of an error may require familiarity
with inner workings of how some feature has been implemented.
Most features rely on helper macros or functions whose names end with \tt{_C} or
\tt{_c} (possibly followed by underscores); these are not part of C\_ and should
not be directly used by programmers, as their availability is not guaranteed.

It is hoped that future implementors of the C\_ dialect will provide
alternatives that have faster compilation time and more informative diagnostics,
possibly in conjunction with better code generation;
this optimism is the primary motivation for documenting C\_ in
terms of abstract semantics rather than concrete implementation.

The reference implementation provides almost everything as macros,
and the underlying preprocessor may impose a translation limit on the maximum
number of parameters for function-like macros, in which case the same limit
should also be applicable to the number of arguments for macro invocations.
The C\_ standard requires that any such limit shall be at least 127,
which is the default value of \tt{PP_MAX} used by the ellipsis framework.
Portable code should not exceed this limit,
which is respected by the reference implementation and sample examples.

It should be noted that the preprocessor is a separate
component that does not understand most rules of the C language;
one serious limitation in calling function-like macros is that brace-enclosed
commas that are not parenthesized get interpreted as separate arguments, which
can cause issues with unparenthesized compound literals having multiple elements
in its initializer (or even a single element with a redundant trailing comma).
We have already seen an example with a wrapper around the \tt{abs} function,
during the discussion on trivial function-like macros;
unfortunately, that workaround is often limited to only the last argument.
As an example, the library function \tt{putc} can be
additionally provided as a macro in \tt{<stdio.h>};
for discussion, let us consider it as a plain wrapper over \tt{fputc}.\\

\tt{#define putc(c, stream) fputc(c, stream)}\\

As we saw earlier, macros like this suffer from subtle problems with compound
literals; a contrived example is \tt{putc('\\n', *(Stream [])\{stdout,\})}
(\tt{Stream} is a synonym for non-modifiable \tt{File_} pointer,
and \tt{File_} itself means a modifiable \tt{FILE} object).
To deal with such unusual arguments,
one should use ellipsis (\tt{...}) instead.\\

\tt{#define putc(c, ...) fputc(c, __VA_ARGS__)}\\

This is a small example where the first parameter \tt{c} may be omitted,
as it can be part of the ellipsis itself; however, macros may need to
isolate arguments for additional logic (such as \tt{_Generic} selection).
A general recommendation is that compound literal arguments having commas
should be parenthesized, which neatly avoids all these complications;
a couple of alternatives are available for non-polymorphic features:
a comma that does not separate arguments can be written as \tt{COMMA},
or an entire argument with unparenthesized commas can be wrapped
within \tt{echo_}, a basic primitive of ellipsis framework.
A small relaxation is that the last argument of a non-polymorphic
feature allows unparenthesized commas, without requiring any workaround.
As another artificial example, \tt{guard__1_(prices, 0)} can be written as \tt
{guard__1_(prices, *(Int [])\{0,\})}, but \tt{guard_(prices, *(Int [])\{0,\})}
does not work with the reference implementation, as it counts three arguments:
\tt{prices}, \tt{*(Int [])\{0}, and \tt{\}}; consequently, the invocation gets
resolved to \tt{guard_3_}, which is not available in C\_ and causes an error.

Before moving ahead, it should be stated that this section is only meant
to inform readers of some inherent drawbacks of the reference implementation,
that are mostly due to dependency on the preprocessor.
This is not an exhaustive list of limitations which mostly apply to pathological
corner-cases in contrived code snippets; from a practical perspective,
such artificial counter-examples should be of no concern to most programmers.
It is also worth stressing that these weaknesses are specific to the
reference implementation that serves a proof of concept for C\_, and the dialect
itself is not hindered by such drawbacks; however, for some features, the
behavior of certain cases has been documented as implementation-defined,
and implementations of C\_ are also permitted to disallow such cases entirely.
In particular, the reference implementation
currently does not support most of these scenarios.

\note The choice of writing a preprocessing-based
implementation is best explained by quoting Donald Knuth:

``\it{One rather curious thing I've noticed about aesthetic
satisfaction is that our pleasure is significantly enhanced
when we accomplish something with limited tools.}''

At the very least, the reference implementation is a
testament to metaprogramming capabilities of preprocessor.
