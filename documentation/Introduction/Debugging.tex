C\_ offers out of the box debugging support
without the use of any additional software.
Programs can be compiled in debugging mode simply by defining the
macro \idx{DEBUG}; this setting can be applicable to a whole project,
or even toggled selectively for specific source files.
Several C\_ headers are configurable,
which means that the features implemented by them behave differently depending
on whether the macro \tt{DEBUG} was defined at the time the header was included;
more precisely, these headers are re-configurable, so if we change the
\tt{defined} state of the \tt{DEBUG} macro and then include the header again,
it will be automatically re-configured.
For standard headers of C\_,
the current configuration is internally recorded in an object-like macro,
whose name is same as the header name entirely in uppercase,
and the dot replaced by an underscore;
as an example, \tt{<assert._>} is a re-configurable header
whose current configuration is given by the macro \tt{ASSERT__}.
Such a macro shall expand to either 0 or 1,
where 0 indicates that the \tt{DEBUG} macro was not defined
when the header was last included, so debugging is disabled;
whereas 1 means the latest inclusion of that header was preceded by
an active definition of \tt{DEBUG}, so debugging is currently enabled.

\idx{<c._>} is a top-level header that needs to be included in
C\_ source files (either directly or via another header).
Recall that this file is placed in \tt{include/} directory,
whose contents are meant to be updated by programmers as per their needs.
As this file will be our companion through the rest of this journey,
its worth taking a look inside.

\code{../include/c._}

Besides aggregating contents of several other headers,
\tt{<c._>} also sets a global configuration for the entire project,
which can changed locally within a source file,
either for specific headers or the entirety of C\_.
The uncommented configuration is for ``debugging mode'';
any replacement text of the \tt{DEBUG} macro is ignored.
We also define another macro \idx{TODO}, whose purpose is quite trivial:
it is used to mark incomplete sections of code, such as temporary stubs
(\tt{TODO} is often highlighted by code editors);
the identifier \tt{TODO} should not be used to declare any lexical element.

The section after \tt{/*/} configures ``production mode'', where debugging is
disabled and (non-comment) occurrences of \tt{TODO} are
not erased by the preprocessor, causing compilation errors.
Debugging mode increases code size and the program incurs some runtime overhead,
which can cause a noticeable difference in performance for large projects.
Once a codebase has been thoroughly tested,
it may be desirable to disable ``defensive'' checks.
\tt{NDEBUG} affects the \tt{assert} macro
on a subsequent inclusion of \tt{<assert.h>};
this should be used with caution,
as the expression passed to \tt{assert} is not evaluated,
which can cause unexpected ``Heisenbugs'' due to unevaluated side effects.

The headers included before \tt{#endif} are non-configurable;
those included afterwards are re-configurable.
When a newly created module is used with a well tested project,
it may be desirable to enable debugging only for that module,
while compiling rest of the project in production mode.
As global configurations are within header guard,
they are effective only the first time \tt{<c._>} is included,
so debugging can be enforced locally by including \tt{<c._>} twice.\\

\tt{#include <c._>}

\tt{#define DEBUG}

\tt{#include <c._>}\\

It may also be desirable to enforce debugging for certain features of C\_,
irrespective of the global configuration in \tt{<c._>}.
This can be done by re-including only the specific
header(s) that provide these features; for instance:\\

\tt{#include <c._>}

\tt{#define DEBUG}

\tt{#include <array._>}\\

The above example shows how one can locally enforce
debugging for features provided by the \tt{<array._>} header.
Note that this approach re-configures only the selected headers,
not their dependencies.
For example, \tt{<array._>} depends on some
non-configurable components provided by \tt{<pointer._>};
the latter is also a re-configurable header,
but its existing configuration is preserved,
and to change that one needs to explicitly include it after defining \tt{DEBUG}.
Similarly, one can also selectively disable debugging:
simply undefine \tt{DEBUG} and include the appropriate headers.

The \tt{LOGGER} macro enables logging facilities provided by \tt{<logger._>};
we have chosen to always enable logging,
so it is kept outside the mode configuration.
Note that \tt{<c._>} configures several headers,
but does not provide any debuggable feature by itself.
As such, the macro \tt{C__} does not record the \tt{defined}
state of \tt{DEBUG} to introspect the current configuration;
it would not be of much use because the actual
providers can be individually re-configured.

The extent of debugging support depends
largely on the feature under consideration;
the basic checks are for common sources of bugs,
such as null pointer dereference or accessing an array out of bounds.
A major feature of debugging mode is that
for methods (discussed in chapter 6),
the \tt{call__} expression invokes protocols,
which are used to specify pre-conditions and post-conditions.
A protocol can be bypassed if and only if both the ``provider''
and the ``consumer'' (caller) agree that debugging is not required,
i.e. they are both compiled in production mode.
In other words, if either of them enables debugging,
then the transformation logic (process) is invoked via protocol.

Finally, one important observation is that the global configuration can
be altered simply by changing the second character of the first line;
the mechanism relies on a preprocessing rule: comments do not nest.
In other words, a comment started by \tt{/*} is ended by the first occurrence
of \tt{*/}, disregarding any instance of \tt{/*} within the comment.
If we change the second character inside \tt{<c._>} from \tt{/} to \tt{*},
then the first line becomes \tt{/*},
starting a comment that ends at the line \tt{/*/},
and what follows is the uncommented configuration for production mode.
With \tt{//} as the first line, the line \tt{/*/} acts differently:
instead of ending an ongoing comment, it actually starts a new comment,
which consumes the production mode configuration and ends at the line \tt{/**/}.
In summary, the global configuration can be toggled with
minimum effort just by changing the second character in the file:
\tt{/} means debugging mode; \tt{*} means production mode.
Additional macros can be defined in both
configuration segments as per project requirements.
