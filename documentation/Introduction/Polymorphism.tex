C\_ is strongly influenced by functional programming and
object-oriented programming languages; a common feature of
the latter is polymorphism, which means ``many forms''.
Several function-like features of C\_ are polymorphic in nature,
which means their behavior depends on the number of arguments passed,
and sometimes on the type of arguments as well; most often, additional arguments
extend the basic functionality by allowing for more fine-grained specification.
We are already familiar with one such feature:
the \tt{guard_} statement discussed in the first example.

Polymorphic features follow a particular naming convention:
if the argument sequence to a polymorphic feature named \tt{multiform} expands
to \tt{n} arguments (equivalently, \tt{n - 1} unparenthesized commas),
then a feature named \tt{multiform_n} gets invoked;
some features may also provide an additional form named
\tt{multiform_0} that is used if \tt{multiform_n} is not defined.
In our first example, \tt{guard_(prices)} is resolved to \tt{guard_1_(prices)},
whereas \tt{guard_(prices, 0)} would be equivalent to \tt{guard_2_(prices, 0)}.
Most polymorphic features are provided as function-like macros by the
reference implementation, so as per our conventions, their
names end with underscore.
For documenting the syntax, optional arguments are surrounded by square brackets
(such as [\tt{,} \it{opt}]), and the notation [\tt{,} \it{arg}\tt{=value}]
indicates that if the optional argument \it{arg} is omitted,
then \tt{value} acts as a default argument.

If one of the forms is permitted to evaluate some argument more than once,
then the polymorphic feature is named with two trailing underscores.
The multiple forms themselves have trailing underscores depending
on whether they are permitted multiple evaluation of arguments;
for example, \tt{output__} is a polymorphic feature that has four forms:
\tt{output__0__}, \tt{output__1_}, \tt{output__2__}, and \tt{output__3__}.
Only the form \tt{output__1_} is not permitted to evaluate its argument
more than once; rest of the forms may possibly evaluate only the first
argument multiple times, and only if it has a variably modified type.
In summary, the naming convention for multiple argument expansion dominates:
if at least one of the forms is permitted to evaluate some argument more than
once, then the polymorphic macro itself is named with two trailing underscores,
which also appear before the argument count in the names of individual forms.
\tt{output__} and the complementary \tt{input__} are examples
of dual polymorphism, whose behavior depends on both number and
type of arguments; their precise semantics are discussed in chapter 3.

This section only provides a brief introduction to static polymorphism,
which means that the exact form is resolved at translation time itself;
as the reference implementation provides these features as function-like macros,
their name resolution is done by the preprocessor.
We shall encounter several polymorphic features in the next three chapters,
before chapter 4 provides a complete picture that also demonstrates how such
features can be implemented as function-like macros with the ellipsis framework;
polymorphism based on argument type is commonly
implemented with \tt{_Generic} selections.
C\_ also supports a form dynamic polymorphism with runtime binding
of method calls, which is used with extension or ``inheritance''
of classes and interfaces (discussed in chapter 7 and 8).

There are certain advantages of directly using a specific form instead of its
polymorphic generalization: theoretically, the most obvious benefit is that
the name resolution is skipped, which improves translation time.
Readers who are interested in the reference implementation will find that it
does not invoke any of the polymorphic macros; the specific forms are directly
used instead, which leads to an appreciable decrease in preprocessing time.
