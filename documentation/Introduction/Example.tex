Let us start our journey with a non-trivial example
that illustrates some of the basic features of C\_.
The following program reads a list of prices and discounts (in \%),
and then computes the final price after applying discounts.

\code{../compile/discount.c_}

The first line \tt{#include <c._>} includes the contents of a header file
named \tt{c._}, which itself aggregates and configures several macros,
enumeration constants, type definitions, and inline functions,
from other header files organized in a multilevel hierarchy.
Type names in C\_ start with an uppercase letter,
and are modifiable if and only if the name ends with an underscore;
for example, \tt{Int argc} means the parameter \tt{argc} is non-modifiable.

Recall that in C, a parameter of array type is adjusted to pointer type,
so \tt{Ptr(Char_) argv[const]} means
``non-modifiable pointer to a pointer to \tt{Char_}''.
Note that \tt{Char_} means that the dereferenced lvalue is modifiable,
and \tt{Ptr(Char_)} means that the pointer itself is non-modifiable.
Readers may intuit that \tt{Ptr_(Char_)} makes the pointer modifiable as well,
whereas \tt{Ptr_(Char)} means the pointer is modifiable,
but the dereferenced lvalue is not.

Blocks or compound statements are started with \tt{begin}
and completed with \tt{end}, which offer the benefit of early exit:
if some condition is (un)met, one may skip rest of the
code within the innermost block that supports early exit,
and continue execution just after the end of that block.
This feature is useful to flatten out ``arrow-shaped code'',
which simplifies the overall design for the programmer,
and reduces cognitive burden for the reader:
it can be cumbersome to remember and match braces in a
deeply nested code to figure out where each nested block ends.

This is exemplified in the next line \tt{guard_(argc == 2)},
which is a guard clause that says exactly two command-line arguments should
be passed at runtime; if the check fails, it skips rest of the code.
Notice the absence of a semicolon at the end of the guard clause:
this is because \tt{guard_} is syntactically not an expression, but a
statement, so a semicolon is redundant and only serves as null statement.
The definition \tt{Auto_ count_ = 0U;} creates a
modifiable counter variable initialized to zero,
whose type is inferred as \tt{ULong_} due to the initializer \tt{0U}
(\tt{ULong_} is a synonym for modifiable \tt{unsigned long};
\tt{ULong} is similar but non-modifiable).

Recall that the first command-line argument is
conventionally the name by which the executable is invoked,
which is available to the program as the first array element \tt{argv[0]}.
For this example, we need to pass the number of items as the second argument,
which is available as the next element \tt{argv[1]}.
The call \tt{input__(argv[1], count_)} reads this string for a valid
unsigned number, and on success, assigns that value to \tt{count_}.
It returns the number of input items scanned and assigned,
which should be 1 in our example,
so this check has been expressed as a guard clause,
which fails if the second command-line argument
does not start with a valid unsigned number of items.

The call \tt{new__(Float_ [count_])} allocates a modifiable array of \tt{float}
in the heap memory segment, having a capacity of \tt{count_} elements.
On success, it returns pointer to a \tt{Float_} array of length \tt{count_},
which is inferred as the type of the non-modifiable variable \tt{prices}.
Note that we could have used \tt{Auto} instead of \tt{Var};
the minor distinction between these two is explained in the next chapter.
We know that memory allocation can fail at runtime,
in which case \tt{new__} returns a null pointer;
we have enforced this check with the guard statement \tt{guard_(prices)}.

The next two lines do a similar allocation and check for \tt{discounts},
whose type is also \tt{Ptr(Float_ [count_])}, which is read as
``non-modifiable pointer to a modifiable array of \tt{Float_},
having a capacity of \tt{count_} elements''.
The primary advantage of using the type ``pointer to array''
is that a complete array type encodes length information,
which is utilized later in the program.
Recall that the lifetime of memory objects
allocated on the heap is managed by the programmer;
however, for the sake of brevity, we have omitted an explicit
deallocation of \tt{prices} and \tt{discounts} by calling \tt{free}
(for programs executing in a hosted environment, the operating
system typically takes care of deallocating the heap memory once
the process terminates or aborts, among other cleanup activities).

\tt{loop_(0, count_ - 1)} iterates from zero through \tt{count_ - 1}, and the
current iteration number is stored in the modifiable ``candle'' variable
\tt{_i_}, whose scope and lifetime are limited to the loop block till \tt{end}.
In each iteration, a prompt message is displayed before reading the values for
price and discount entered by the user.
Note that these variables are of type ``pointer to array'',
so we first obtain the array by dereferencing the pointers,
and then access the element at index \tt{_i_}.
The parentheses are necessary around the dereference due
to its lower precedence than the array subscript operator;
as an alternative, it also can be written as
\tt{prices[0][_i_]} or \tt{discounts[0][_i_]}.

The \tt{print_} syntax is modeled after the \tt{print} function of Python 3,
which prints a space between arguments and also appends a trailing newline.
\tt{scan__} reads from \tt{stdin} (the standard input stream),
and assigns the values to its arguments in the same order.
Similar to \tt{input__}, \tt{scan__} also returns
the number of input items scanned and assigned,
which can be less than the number of arguments if the user
enters an invalid value outside the domain of an argument's type,
or if the input stream is exhausted and a
subsequent read returns \tt{EOF} (end of file).

Notice that the \tt{guard_} statement syntax is different from before:
this time there is an additional argument \tt{1} just after the guard clause.
This is because \tt{guard_} is actually a ``polymorphic'' statement,
which can accept either one or two arguments.
In the first form, if the guard clause fails,
then rest of the statements are skipped.
However, if we use this form within a \tt{loop_} block,
then it will only skip the remaining iterations if the guard clause fails,
and continue execution after the loop.
However, this is not desirable:
if a value cannot be read, the program should not proceed any further.
In the second form of the \tt{guard_} statement, if the clause is unsatisfied,
then the function returns the value specified by the additional argument.
We have used \tt{1} in this example,
because the value returned by \tt{main} is also exit status of the process,
which is conventionally zero on success and non-zero to indicate failure.

After reading the prices and discounts,
we define a variable \tt{price_} to aggregate the total price;
the initializer \tt{0.f} makes its type \tt{Float_}.
The statement \tt{op_(price_, +, prices)} adds each price of
the \tt{prices} array one by one to the variable \tt{price_};
it was initialized to zero earlier,
so its result is the sum of all prices, which is printed.

\tt{op_} is a versatile statement that unifies a rich variety
of operations with both arrays and non-arrays under a common syntax.
The statement \tt{op_(discounts, *, prices)} multiplies each element of the
\tt{discounts} array with the corresponding element in the \tt{prices} array.
The results are stored in the \tt{discounts} array,
but as the discount values are entered as percentage,
we also need to divide each discount by 100:
this is done by the next statement \tt{op_(discounts, /, 100)}.
\tt{op_} is modeled after \tt{op__};
their precise semantics are described in chapters 4 and 5.

Next we run a loop to print the discount amount applied to each item.
This is followed by storing the total discount amount in the variable \tt
{discount_}, which is done by the statement \tt{op_(discount_, +, discounts)}.
We print this amount and also the final price
after subtracting the total discount.
Since the C99 revision of ISO C,
reaching the end of \tt{main} returns zero to the environment,
so an explicit \tt{return 0;} statement has been omitted.
