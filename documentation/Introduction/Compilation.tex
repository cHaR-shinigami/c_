\def\Subsection#1{\subsection{#1}\input{Introduction/Compilation/#1}}

To compile our first example, we need a concrete implementation of C\_.
The reference implementation can be freely downloaded from the repository
\url{https://github.com/cHaR-shinigami/c_}.
It contains a directory \tt{examples/}, which can be considered as a project
directory that contains source codes of examples discussed in this document.
It contains three sub-directories:
\tt{.include/}, \tt{include/}, and \tt{compile/}.
\tt{.include/} contains header files which form the core implementation of C\_;
these files need not be modified by most programmers and their
contents are likely to change over time, mostly for bug fixes
and occasionally to accommodate new features in existing headers.
The directory \tt{include/} contains the header \tt{<c._>},
which we saw in the previous example, along with several
other header files that will be discussed in later chapters.
Files within the directory \tt{compile/} are
given to the compiler as translation units.
Both \tt{include/} and \tt{compile/} files
are meant to be modified by programmers.
By convention, C\_ header files have \tt{._} as filename extension;
C\_ translation units are analogous to \tt{.c} files,
and have \tt{.c_} as extension.

\Subsection{Requirements}

\Subsection{Execution}
