The source codes presented in this document follow a certain style of spacing
between operands and operators: if an operand is placed between two operators,
then we put the operand ``closer'' to the operator that uses this operand;
for example, the evenly spaced expression \tt{a + b + c} is written as
\tt{a+b + c} to emphasize left associativity of addition.
Similarly, we would prefer writing \tt{a * b + c} as
\tt{a*b + c} to indicate that multiplication happens first.

Conversely, if an operator is placed between two operands,
we put the operator closer to the operand that is evaluated first;
an equal spacing on either side indicates that the evaluation of
operands is unsequenced, so there is no specific order of evaluation.
For example, \tt{l & r} indicates that either operand may be evaluated first,
but \tt{p&& q} emphasizes the ``short-circuit'' behavior of logical operators:
in this case, the left expression \tt{p} is evaluated first; \tt{q} is evaluated
if and only if \tt{p} turns out to be zero.
Similarly, we prefer uneven spacing for a conditional expression \tt{c? y : n}
or a comma expression \tt{(x, y)}; the latter aligns with the conventional
writing style, but note that other uses of comma, such as for separating
function call arguments, do not specify any particular order of evaluation.

It is worth stressing that judicious spacing only clarifies that the programmer
is well aware of the precedence and associativity rules of operators;
it has no impact on the meaning of the expression,
and explicit parentheses can often be a cleaner alternative.
As a third alternative, sometimes the use of a different
operator may convey the intent more clearly; for example,
\tt{*ptr_++} may be confusing to beginners: which one gets incremented,
the modifiable pointer \tt{ptr_}, or the dereferenced lvalue?
Experienced developers will be aware of right associativity for unary operators,
but a simple space should make things somewhat less confusing: \tt{* ptr_++}.
Still, the subscript operator can make things more readable; the equivalent
\tt{ptr_++[0]} clearly indicates that the pointer gets incremented.
Unlike the previous naming conventions, the unusual use of spaces is more of
an idiosyncrasy that is not imposed upon programmers, and ritually observing
this practice can lead to awkward gaps if lots of operators are involved;
while a recommended practice is to break such large expressions into smaller
sub-expressions, it is hoped that readers will be forbearing if they perceive
that this advice may not have been diligently followed in some of the examples.
