The underscore character plays a pivotal role in the naming
scheme designed for this dialect, which justifies the name C\_;
it is believed that liberal use of trailing underscores in C\_ also reduces
the chances of name collision with identifiers in existing libraries.
We are already familiar with a couple of rules from the first example:
a type name ending with an underscore means that it is modifiable.
Readers may have already observed that the same
convention is also followed for identifiers of variables:
they do not start with an uppercase letter,
and a trailing underscore means the object is modifiable.
Furthermore, type names begin with an uppercase
letter and contain at least one lowercase letter;
the latter requirement is to differentiate it from object-like macros,
which are traditionally named entirely in uppercase.
Except for a few keyword-like features, C\_ continues the same
tradition for naming object-like macros, with an additional requirement:
except for header guards, they should not end with an underscore.

Names of function-like macros end with underscore to differentiate
them from function names; the latter should not end with an underscore.
One design principle of C\_ is that a function-like macro, whose documented
semantics are required to be similar to those of a proper function call,
should mimic the behavior of an actual function as closely as possible.
However, there is a fundamental limitation that cannot be overcome:
a macro does not have an address,
so it is impossible to obtain a function pointer for a macro.
A trailing underscore acts as a form of self-documentation for such limitations:
it makes the programmer aware that instead of an actual function, some other
form of implementation may be used, most likely a function-like macro.
Such names will be encountered frequently in later chapters, as most features
of C\_ are provided as function-like macros by the reference implementation.

The abstract semantics of C\_ does not mandate that
its features need to be implemented with macros;
thus implementors are not bound to rely on the preprocessor,
and they can write a separate front-end pass that transpiles C\_ code to C code.
For the latter approach, it is strongly recommended that the names for non-macro
implementations should additionally be defined as trivial object-like macros;
it is hoped that any code that works correctly with the reference implementation
should also behave identically when compiled with other implementations of C\_.

One undesirable aspect of function-like macros is that the
replacement text can expand some argument more than once;
this can lead to unfortunate outcomes if the argument contains
side effects and more than one expansion gets evaluated.
C\_ follows the convention of naming such macros with two trailing underscores,
which signifies that the implementation is permitted,
but not obliged, to evaluate some argument multiple times.
Thus any C\_ feature whose name ends with two underscores is permitted to be
implemented as a function-like macro that may possibly evaluate some
argument more than once; the documentation clearly specifies which exact
argument(s) may undergo multiple evaluation, and under what circumstances.
For some features, multiple evaluation of arguments can itself be part of the
semantics; for example, the macro \tt{repeat__(frequency, ...)} from the
ellipsis framework replicates the argument sequence given by \tt{__VA_ARGS__},
for the number of times specified by the \tt{frequency} argument.

On the contrary, a feature whose name ends with a single underscore has
more stringent requirements: it can be implemented as a function-like macro,
but it is not permitted to evaluate any argument more than once.
However, a macro implementation is allowed to expand an argument more
than once, as long as at most one such expansion is evaluated at runtime;
one example is the use of a macro argument in
multiple associations of a \tt{_Generic} selection.
The phrase ``at most'' implies that such macros may ignore some argument or perhaps
expand it in a context where it does not get evaluated under certain conditions.
As before, there may be cases where this behavior is required by the semantics;
for example, the conditional expression \tt{test_(}$check$\tt{,} $yes$\tt{,}
$no$\tt{)} expands all three arguments, but only one of the
sub-expressions $yes$ or $no$ gets evaluated, depending on whether
the scalar expression $check$ is true (non-zero) or false (zero).
Recall that in C, a ``scalar expression'' is of arithmetic type or pointer type;
equivalently, it is an expression that can be assigned to or
cast as boolean type (\tt{_Bool} in C is \tt{Bool_} in C\_).

A function-like macro whose name does not have any
lowercase letter and ends with a single underscore
indicates that a valid invocation expands to a list of constants.
The expanded list can be a singleton; for example, the macro \tt{COUNT_(...)}
from the ellipsis framework counts the number of arguments passed to it
(more precisely, it returns one more than the number of
unparenthesized commas in the expanded argument sequence).

Header guards are defined as object-like macros having
the same name as the filename entirely in uppercase,
with the dot being replaced by an underscore;
this assumes that header filenames should not start with
a digit and not contain any non-identifier characters.
We have already encountered one such example before:
\tt{C__} acts as the guard for the header \tt{<c._>}.
Another example is the macro \tt{ASSERT__}, which records whether
the header \tt{<assert._>} was last configured in debugging mode;
in addition to that, it also doubles as the header guard.
The same practice has been adopted for other configurable headers;
for a header file named \tt{header._},
an object-like macro \tt{HEADER__} serves a dual purpose:
besides acting as a header guard to avoid redefinitions, it also records the
\tt{defined} state of \tt{DEBUG} macro every time \tt{header._} is included.
One common use of this convention is to check
whether a header has already been included or not,
which helps to avoid any unintended re-configuration of dependency headers.

Throughout this document, we have used \tt{<}angle bracket\tt{>} syntax
for including the configuration header \tt{<c._>} and C\_ standard headers;
reference implementation of the standard headers
are placed within \tt{.include/} directory.
The example headers used for illustrative purposes are not part of C\_; these
have been included with the \tt{"}double quotes\tt{"} syntax, to differentiate
them from standard headers that are to be provided by all implementations of C\_.
