The \tt{compile/} directory has a file named \tt{discount.c_},
which contains the previous example.
It has another file named \tt{lib.c_},
which provides external definitions for several helper functions.
These definitions are required by the linker, so instead of recompiling this
file \tt{lib.c_} every time, we can generate an object file to be linked later.
The project directory \tt{examples/} has a shell script named \tt{build.sh},
which automates this task and also generates object files
for all the translation units in the \tt{compile/} directory.
This script is intended for POSIX-compatible shells (such as \tt{bash}),
and needs to be executed only once.
The commands of this script are discussed in appendix E,
based on which similar scripts can be written for other environments
(such as classic Windows \tt{cmd} or \tt{PowerShell}).

The reference implementation requires a compiler that supports C23.
Full C23 support is not necessary; only a few improved language rules
and new features are required to emulate the behavior of C\_ using C.
The precise set of requirements can be found in the companion document
that explains the full architecture of the reference implementation;
many of these dependencies have been available for a long time,
so these are not entirely new, but ``prior art'' which may
be available in compilers that do not fully support C23 yet.

The source codes presented in this document have been tested with two
compilers installed on a 32-bit variant of Ubuntu operating system:
\tt{gcc} (version 14) and \tt{clang} (version 19),
whose precompiled binaries for i386 architecture (IA-32)
were downloaded from the official software repository of Ubuntu.
The most important task is to make sure that our compiler is able to locate the
header files; both \tt{gcc} and \tt{clang} have similar options for this.
For demonstration, if we assume that \tt{examples/} is present
in the home directory, then the following can be used:\\

\tt{gcc/clang -xc -std=c23 -iprefix "\$HOME"/examples/.include}

\tt{-iwithprefix/ellipsis -iwithprefix/dialect -iwithprefix/library}

\tt{-iprefix "\$HOME"/examples/include -iwithprefix/.}

\tt{gcc/clang} needs to be at least \tt{gcc-14} or \tt{clang-19}.
The option \tt{-xc} is required due to the filename extension \tt{.c_}:
it tells the complier to consider them as C files;
\tt{-std=c23} is required just in case the default
language version is set to an older C standard.
\tt{-iprefix "\$HOME"/examples/.include} sets the path relative to which
the subsequent \tt{-iwithprefix} locations are considered, up to the next
\tt{-iprefix} which changes the relative path to \tt{include/} directory.
The sub-directories within \tt{.include/} collectively contain nearly a hundred
header files that are logically structured into multilevel hierarchies.
As a small note, \tt{bash} supports ``tilde expansion'' for the home
directory, so \tt{"\$HOME"} can be replaced with the more terse $\sim$ symbol.

With that, we're all set to compile and execute our first C\_ program.
Later in this chapter, we shall add several more compilation
options to enable rigorous diagnostics; however,
the overall command becomes rather clunky when typed out in its entirety.
A much cleaner workaround is to set a ``command alias'',
which is done in \tt{bash} as:\\

\tt{alias cc_="gcc -xc -std=c23 -iprefix '\$HOME'/examples/.include \\}

\tt{-iwithprefix/ellipsis -iwithprefix/dialect -iwithprefix/library \\}

\tt{-iprefix '\$HOME'/examples/include -iwithprefix/."}\\

Note that the trailing backslash in first
two lines is meant for line splicing.
Most modern shells will have some mechanism for setting a command alias.
By default, \tt{bash} tries to read the contents of \tt{"\$HOME"/.bash_aliases}
(if the file exists), so a better approach is to put
our alias in that file (create the file if needed);
it comes into effect at the next invocation of \tt{bash}, or by ``sourcing''
it as \tt{. "\$HOME"/.bash_aliases} in the current invocation itself.
Finally, if the \tt{examples/} directory is put elsewhere,
then users will have to replace \tt{"\$HOME"} with the correct path.
