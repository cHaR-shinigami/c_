\def\Subsection#1{\subsection{#1}\input{Introduction/Conventions/#1}}

The reference implementation operates with ``pseudo-namespaces''
that use name mangling to emulate the behavior of proper namespaces:
members in different namespaces can be designated by the same identifier.
For a proper functioning of these so-called pseudo-namespaces,
the reference implementation ``trusts the programmer''
to abide by a few simple naming restrictions,
in addition to the existing identifier reservations imposed by the C standard.

Firstly, macros defined by the reference
implementation shall not remain undefined,
and the programmer shall not provide a
non-equivalent redefinition of these macros.
Secondly, non-macro names defined in C\_ shall
not be defined as non-trivial object-like macros;
however, it is permitted to define them as trivial
object-like macros, or even function-like macros.
A trivial object-like macro is one whose
expansion does not alter the source code.
In other words, an object-like macro is trivial if its
replacement text expands to the macro name itself; for example:\\

\tt{#define identifier identifier}\\

One advantage of function-like macros is that their expansion can be suppressed,
if the first non-whitespace character after an occurrence
of the macro name is not the opening parenthesis;
for example, standard library functions may be additionally provided as macros,
and a common trick to suppress macro expansion and call the actual
function is to parenthesize the name, such as the call \tt{(putc)('\\n')}.
On the contrary, the expansion of an object-like macro cannot be suppressed,
which can be problematic if an existing non-macro
identifier is defined as a non-trivial object-like macro.
These restrictions are not specific to C\_,
but applicable to C in general; for instance,
the standard header \tt{<stdlib.h>} defines the structure
type \tt{div_t} with two members \tt{quot} and \tt{rem},
so none of these names can be defined as non-trivial object-like macros;
however, the following macro definitions are unlikely to cause any problem:\\

\tt{#define quot quot}

\tt{#define rem(...)}\\

We can also define the notion of a trivial function-like macro in a similar
fashion: a function-like macro is considered to be trivial is its expansion
is successful and does not modify any text of the source code; for example:\\

\tt{#define abs(...) abs(__VA_ARGS__)}\\

Despite the simplicity of its definition,
there are certain caveats in the design of trivial function-like macros.
\tt{abs} is a standard library function declared in the header \tt{<stdlib.h>}:
it takes a single argument of type \tt{int}
(\tt{Int_} in C\_) and returns its absolute value.
A naive implementation of a corresponding function-like macro might look like:\\

\tt{#define abs(n) abs(n)}\\

However, the above macro is actually non-trivial; in
the absence of this macro, the call \tt{abs(*(int [])\{0,\})} works correctly,
but the preprocessor does not understand compound literals,
so a macro invocation actually reads it as two arguments:
\tt{*(int [])\{0} and \tt{\}}.
This may be a contrived example, but is nonetheless perfectly
legal C code that is rendered invalid by our second definition
of \tt{abs} with a single parameter, making the macro non-trivial.

Note that our definition of ``triviality'' only requires that the
resulting expansion of the macro should not alter the source text;
however, it does not imply that a trivial macro can be freely undefined,
which can impact a subsequent \tt{#if} preprocessing
directive that checks whether a macro name is defined or not.
Another naming restriction of C\_ is that user-defined
identifiers should not have two consecutive underscores anywhere;
this is to avoid name collisions with several internal identifiers
with double underscores that are formed as a result of name mangling.

\Subsection{Naming}

\Subsection{Reservations}

\Subsection{Spacing}
