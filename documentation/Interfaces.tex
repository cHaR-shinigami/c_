\def\Section#1{\section{#1}\input{Interfaces/#1}}

``\it{The psychological profiling [of a programmer] is mostly the
ability to shift levels of abstraction,\\from low level to high level.
To see something in the small and to see something in the large.}''
— Donald Knuth\\

An interface in C\_ defines a structure of function pointers
for operations permitted on an abstract data type.
Unlike concrete data types implemented with classes whose data members
are exposed by the declaration, abstract data types do not specify
how the data is internally organized or represented.
Interfaces provide an insulating layer over data that prevents direct
access and modification: all operations need to be done via functions
pointers that are part of the interface structure.
In other words, the behavior is specified, but the representation is opaque.

As an analogy, the powers and functions of a national government
are described by the constitution of that nation, which
formalizes what a government can (or cannot) do.
An interface is something similar, and in C\_, the expected behavior
is described with pre-conditions and post-conditions in protocols.
A government is materialized by political parties, who must abide by the
laws formulated in the constitution.
Similarly, an interface is an abstract specification of behavior,
and to actually perform these operations on an object instance,
we need functions that know how the data is internally represented.
These functions are provided by classes; more precisely,
these functions are implemented as procedures of class methods.
A class that provides the functions required by an interface structure
is said to ``concretize'' or implement that interface, and instances of that
class are said to materialize that interface.
An interface can be concretized by multiple classes, and a single class
can implement multiple interfaces.

Besides the aesthetic appeal of representation-agnostic code,
interfaces promote modular design, low coupling, and provide
opportunities of code reuse through dependency inversion.
The usual requirement is that every interface depends on a concrete class
that implements the necessary functions as procedures.
However, once the basic operations have been defined for an abstract type,
higher order functionalities can be directly implemented by the interface itself
using the basic operations.
If such an interface is concretized by multiple classes, then each
class can utilize these higher order functionalities without
having to re-implement the same logic.
In other words, each implementing class that concretizes the basic operations
depends on the interface for higher order functionalities that are provided
by the interface.
While we refer to this as ``dependency inversion'', it is part of a
mutual interdependence between abstract and concrete types, and if done correctly,
it can also reduce the chances of programming bugs.

Recall that each class has an associated \tt{Type} structure that provides
pointers to procedures implemented by that class (or inherited from its base class).
The same benefit is also provided to each interface for the basic \tt{Type} methods.
Additionally, the structure of function pointers declared by an interface
is actually an extension of the basic \tt{Type} structure discussed in the
previous chapter, which is done by inheriting from the \tt{Abstract} type.

\note Unlike some object-oriented languages,
C\_ does not support multiple inheritance among interfaces.

\section{\type{Abstract} type}
\def\Subsection#1{\subsection{\idx{#1}}\input{Interfaces/Abstract type/#1}}

\tt{Abstract} type extends the \tt{Object} class, and by virtue of inheritance,
\idx{struct Abstract} contains a member \idxy{struct Abstract}{base}
defined as a singleton array whose element type is \tt{Object_}.
\tt{Abstract_} is a synonym for \tt{struct Abstract},
while \tt{Abstract} is its non-modifiable counterpart.
The member \idxy{struct Abstract}{type} is an alias for \tt{base->type},
which is initialized by the \tt{Abstract} constructor to point to the \tt{Type}
structure of \tt{Abstract}, containing pointers to overridden procedures.
\tt{Abstract} structure also contains a pointer member that
can be accessed with three names having different types:

\begin{itemize}

\item The name \idxy{struct Abstract}{concrete}  is declared
as pointer to \tt{Void},  referring to a non-modifable object.

\item The name \idxy{struct Abstract}{concrete_} is declared
as pointer to \tt{Void_}, referring to a modifiable object.

\item The name \idxy{struct Abstract}{_concrete} is a pointer to an
untagged union with two members, both aliased to the same pointer:
\idxyz{struct Abstract}{_concrete}{type}  is a pointer to \tt{struct Type}, and
\idxyz{struct Abstract}{_concrete}{typex} is a pointer to an extended \tt{Type},
with \tt{Typex_(Abstract)} being a synonym for the pointer type
(\tt{Type} extension using interfaces is documented in a later section).

\end{itemize}

When an interface is instantiated with \tt{Type} of a concrete class,
the above three names alias to the same member, pointing
to an instance of the class that concretizes the interface.
The member \tt{type} (or \tt{base->type}) provides the abstract lineage,
while \tt{_concrete->type} points to \tt{Type} of the
implementing class, providing the concrete lineage.

The macros \idx{Abstract_EXTENDS} and \idx{SELF_IMPLEMENTS_Abstract}
are reserved for implementations of C\_.

\note The generic pointer type aliases are used to avoid
boilerplate casts during assignments and invocations.

\Subsection{concrete_}


\section{\tt{Abstract} procedures}
\def\Subsection#1{\subsection{\idxy{Abstract}{#1}}
\input{Interfaces/Abstract procedures/#1}}

The \tt{Abstract} type overrides all procedures of the \tt{Object} class,
under the assumption that these functions will be called with valid
instances of \tt{Abstract}, or instances of interfaces that inherit
from \tt{Abstract} (Liskov substitution).

\Subsection{validate}

\Subsection{init}

\Subsection{free}

\Subsection{compare}

\Subsection{copy}

\Subsection{read}

\Subsection{write}

\Subsection{parse}

\Subsection{text}

\Subsection{decode}

\Subsection{encode}

\Subsection{add}

\enlargethispage*{\baselineskip}
\enlargethispage*{\baselineskip}
\enlargethispage*{\baselineskip}

\Subsection{sub}

\Subsection{mul}

\Subsection{div}


\Section{Creating an interface}

\Section{Procedures and methods}

\Section{Concretization}

\Section{Dependency inversion}

\Section{Inheritance}

\Section{More examples}
