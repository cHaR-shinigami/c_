\def\Section#1{\section{#1}\input{Types/#1}}

C\_ provides synonyms for the standard data types in C;
the reference implementation defines them in the header \tt{<types._>}.
C\_ uses a type system we shall call ``twin typing'',
where each type comes as a pair:
the non-modifiable variant does not end with an underscore,
and its modifiable ``twin'' has the same name suffixed with an underscore.
By convention, C\_ type names begin with an uppercase letter,
and contain at least one lowercase letter.

The concept of ``immutability'' originates from functional programming,
and is increasingly being adopted by new programming languages.
C\_ follows the same trend,
as it can prevent accidental modifications due to typographic errors;
one common source of bugs is writing the equality operator \tt{==} as \tt{=},
which can cause silent assignment if the left
operand happens to be a modifiable lvalue.
Somewhat ironically to the name C\_, we advocate the use of types without
underscore; modifiable types should be used judiciously, only when necessary.
Besides reducing chances of unintended mutations, it can also aid in
optimizations, and lesser use of underscore improves visual appeal of code.

For consistency, we also follow the same convention for variables; a trailing
underscore is easy to spot, making one realize that the variable is modifiable.
Variable names start with a lowercase letter to differentiate
them from type names; they should not start with an underscore,
which is reserved for special purposes.
We start by listing the basic types with their minimum bit-width,
without repeating their modifiable twins whose existence is implied.

\Section{Integer}

\Section{Floating}

\Section{Optional}

\section{\type{Void}}
\input   Types/Void

\Section{Synonyms}

\section{\idx{typeof_}}
\input  Types/typeof_

\Section{Unaliasable}

\Section{Inference}

\Section{Properties}

\Section{Pointers}
