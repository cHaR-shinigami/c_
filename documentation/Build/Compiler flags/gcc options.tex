The following additional options are used
for invoking \tt{gcc} in our build script.

\begin{itemize}

\item \tt{-ftrack-macro-expansion} is an option for the preprocessor
\tt{cpp} that controls location tracking of preprocessing tokens
that undergo nested macro invocations, and this information is shown
in diagnostic messages when a macro invocation causes an error.
As the foundation of the C\_ reference implementation is
built upon the preprocessor, even a minor typographic mistake
in code can trigger an avalanche of preprocessing errors.
A detailed stack trace report of macro expansions can be
beneficial for finding bugs in the reference implementation itself;
however, they are of limited interest to most programmers, and excessive
verbosity can overwhelm beginners about the precise cause of an error,
which can be as minor as an extra comma in a macro invocation.
Setting this option to zero disables it, thereby
limiting the depth of preprocessing error messages.

\item \tt{-Wduplicated-branches} warns when two blocks of
a conditional expression or statement have identical code.

\item \tt{-Wduplicated-cond} warns when two mutually exclusive branches,
such as \tt{if} and \tt{elif} (short for \tt{else if}), have conditions
whose values can be statically determined to be identical.
This makes the code guarded by the \tt{elif} unreachable:
when the condition is satisfied for \tt{if} branch, then \tt{elif} branch
will not execute due to mutual exclusion, and when the condition fails
for \tt{if} branch, then it will also fail for \tt{elif} branch as well.

This programming fallacy be demonstrated with a concrete example,
as shown in this contrived code snippet.

\code{Build/Compiler flags/unreachable.c_}

The above function cannot execute \tt{return 1} due to the guarding
\tt{elif} having the same condition as its preceding \tt{if}.
However, declaring the parameter as \tt{volatile Int n} suppresses the warning,
as it tells the compiler that the value of \tt{n} can possibly change from zero
to non-zero between the branches, due to external sources of mutation (such as
another concurrent thread of execution updating this value via a pointer to it).

\item \tt{-Winit-self} warns about initializing an uninitialized
variable with its indeterminate value, such as \tt{Int n = n;}

\item \tt{-Wnull-dereference} warns about execution paths
that can lead to dereferencing a possibly null pointer.
This is an example of a warning that is enabled by optimizations: in this case,
\tt{-Wnull-dereference} comes into effect when \tt{-fdelete-null-pointer-checks}
is active, which is enabled by \tt{-O2} and higher optimizations.

\item \tt{-Wshift-overflow=2} warns about integer overflow for bitwise left shift
operations; in particular, setting it to \tt{2} enables warning about shifting
a \tt{1} to sign bit position, when the promoted left operand has a signed type.

\item \tt{-Wstrict-overflow=2} warns about signed overflows
in cases where \tt{gcc} assumes that overflow will not occur.

\end{itemize}

The following options starting with \tt{-Wno-} are used to
disable certain warnings turned on by \tt{-Wall} or \tt{-Wextra}.

\begin{itemize}

\item \tt{-Wno-parentheses} is used to forgo redundant parentheses
in some expressions, notably with \tt{iff} and \tt{implies}.

\item \tt{-Wno-tautological-compare} is used to disable few
false positive warnings for the reference implementation.

\item \tt{-Wno-type-limits} disables warnings on range
checking conditions that are always true on most environments.

\end{itemize}
