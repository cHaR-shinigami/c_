The following additional options are used
for invoking \tt{clang} in our build script.

\begin{itemize}

\item \tt{-fmacro-backtrace-limit} determines the maximum backtrace depth for
the preprocessing call stack that is implicitly updated for macro invocations.
Setting it to \tt{1} reduces the verbosity of preprocessing error messages,
which is analogous (though not entirely equivalent) to
the option \tt{-ftrack-macro-expansion=0} for \tt{gcc}.

\item \tt{-Wassign-enum} when an enumeration type lvalue
is assigned a value other than its enumeration constants.

\item \tt{-Wshift-sign-overflow} warns when a left shift operation with a signed
left operand (after type promotion) might shift a \tt{1} to the sign bit position;
this option is analogous to the setting \tt{-Wshift-overflow=2} in \tt{gcc}.

\item \tt{-Wunreachable-code-aggressive} activates several diagnostic
options for detecting code points that cannot be reached during execution.
Removing such code reduces object file size without changing functional behavior.

\item \tt{-Wno-pointer-arith} is used to disable some false
positive warnings for address arithmetic on pointer to VLA.

\end{itemize}
