\enlargethispage*{\baselineskip}
\enlargethispage*{\baselineskip}

The following script code is available in the source file \src{build.sh}

When the script is executed directly (such as \tt{./build.sh}),
the first line starting with the ``shebang'' \tt{#!} tells that
the program \tt{/bin/bash} is to be used for running this script
(it is also a comment line due to the preceding \tt{#}).

The \tt{set -e} line is to stop the script immediately as soon as a command
terminates with a non-zero exit status, skipping rest of the subsequent commands
(recall that a non-zero exit code indicates failure, whereas zero indicates success).
Exit status is nothing but the return value of \tt{main}
or argument to \tt{exit}/\tt{_Exit}/\tt{quick_exit} calls.

\tt{\$0} gives the first command-line argument, which is the name by
which the script was invoked: passing it to \tt{realpath} gives the
full pathname where the file is located (ending with the filename),
and \tt{dirname} obtains the directory name only (without the filename).
Enclosing the command within parentheses makes it execute within
a subshell, and its outcome is stored in the variable \tt{C_},
which will be used as a path prefix in subsequent commands.

\pagebreak

The heredoc starting with \tt{<<} is used to emulate a multi-line comment:
the \tt{'#'} after that marks the delimiter whose
subsequent occurrence terminates the heredoc text.
In the first case, the heredoc itself is commented out by a preceding \tt{'#'},
and the \tt{#} in a subsequent line that would have otherwise
delimited the heredoc text becomes an ordinary comment line.
Both blocks of heredoc texts are for initializing a variable
\tt{CC_} with an invocation of \tt{gcc} or \tt{clang}, with its
associated command-line flags that includes several diagnostic options.
When the \tt{<<} line is commented out, the following text
comes into effect, which is used to set compilation options.
For example, to use \tt{clang} instead of \tt{gcc}, uncomment the heredoc
before \tt{gcc} options can comment out the one before \tt{clang} options.

The file \tt{build.sh} is located directly within \tt{examples/}
directory, whose absolute path is stored in the variable \tt{C_}.
After initializing \tt{CC_} with the compiler invocation string,
we change the current directory to \tt{C\_/compile/}.
Once inside the \tt{compile/} directory, we execute the compiler command
on the source file \tt{lib.c_}, which generates an object file named
\tt{lib.o} containing various external variables and function definitions
(for the dialect as well as standard library extensions).
The \tt{strip} command is used to reduce size of the object file
by removing non-essential symbols that are not required for code
relocation and linking purposes (such as debugging symbols).
We then create a subdirectory named \tt{object/} within \tt{examples/}
directory, and move the file \tt{lib.o} to that directory (the \tt{-p}
option to \tt{mkdir} avoids an error if the directory is already present).
After that we change our current directory to \tt{object/}.

Once inside the \tt{object/} directory, we create a
subdirectory named \tt{class/} and change to that directory.
The source files directly within \tt{compile/class/} contain class definitions,
which are compiled into object files and placed within \tt{object/class/};
the \tt{strip} command is invoked on each object file to reduce file size.
The loop iterates over each directory name in \tt{compile/class/}:
for this we open a subshell, change the current directory to
\tt{compile/class/}, and the output of \tt{ls -d */} gives all
directory names without path prefix and with a trailing \tt{'/'}.
For each directory in \tt{compile/class/}, we create a corresponding directory
with the same name in \tt{object/class/}, which is used for storing stripped
object files of each source file containing the methods associated with that class.

Once the loop terminates, we go back to the \tt{object/}
directory, and create a subdirectory named  \tt{interface/}.
As done for each class, similar steps are repeated to create
object files for each interface and its associated methods.

\note To run the script as \tt{./build.sh}, make it executable
using \tt{chmod +x build.sh} (if not done already).
