The following compilation options are common for both \tt{gcc} and \tt{clang}.

\begin{itemize}

\item \tt{-c} means compile only; in other words, generate
relocatable object files without linking them to an executable.

\item \tt{-xc} means consider the filename extension as \tt{.c}
for subsequent input files and invoke the C language compiler.

\item \tt{-std=c2x} sets the language dialect to C23 (also called C2x);
this option also affects the diagnosis of \tt{-Wpedantic}.

\item \tt{-O3} enables several optimizations at
level 3, mostly focused at improving runtime efficiency
(though often at the expense of increased code due to space-time tradeoff).
Another benefit of using \tt{-O3} is that some warnings
options are activated only when certain optimizations are
enabled that perform a more rigorous static analysis.

\item \tt{-Werror} turns diagnostic or warning messages
into hard errors that cause a compilation failure.

\item \tt{-iprefix} sets a path prefix for subsequent use of
\tt{-iwithprefix} option, until the next occurrence of \tt{-iprefix}.

\item Each usage of \tt{-iwithprefix} adds the subsequent name to the
list of search directories for \tt{#include} directives: directory name
is prefixed with the path specified by the preceding \tt{-iprefix}.
More precisely, the following directories (relative to \tt{examples/}) are
added to the path for locating header files, searched in the given order:

\enlargethispage*{\baselineskip}

\table{lll}

1. \tt{.include/ellipsis/} & 2. \tt{.include/dialect/} & 3. \tt{.include/library/}\\

4. \s\tt{include/} & 5. \s\tt{include/class/} & 6. \s\tt{include/interface/}\\

\elbat

\item \tt{-Wall} and \tt{-Wextra} enable warnings that can help
diagnose potential sources of bugs and undefined behavior;
some of the warnings vary between \tt{gcc} and \tt{clang},
and the precise lists can be found in their documentations.

\item \tt{-Wpedantic} ensures strict conformance with the language dialect
specified by \tt{-std} option, by flagging the use of non-portable extensions
and certain kinds of code whose behavior is not well-defined by the C standard.
For example, the features that are marked with an asterisk (*) in this
documentation are mostly provided by the reference implementation
using statement expressions, a GNU C feature that is supported by
several C compilers alongside \tt{gcc}, but as of this writing,
the syntax is not permitted by the rules of ISO C grammar.

\item \tt{-Wcast-align} warns on casting a pointer type to another pointer
type whose dereferenced type has stricter alignment (higher power of two).
For instance, a \tt{ULLong} is wider than \tt{UByte} on practically
all existing environments, so \tt{-Wcast-align} would generate a
warning if a pointer to \tt{UByte} is cast as pointer to \tt{ULLong}.

\item \tt{-Wcast-qual} warns on removal of
qualifiers from the target type of a pointer.
For example, non-modifiable types named without a trailing underscore are
implemented using \tt{const} qualifier, so warning would be generated on
converting a \tt{Char *} to \tt{Char_ *} (such as by assignment or type cast).
It is worth mentioning that there are known workarounds to circumvent
this artificial limitation; for instance, the reference implementation
provides \tt{unqual__}/\tt{unqual_}* in the header \tt{<pointer._>}
by removing qualifiers via type punning using \tt{union}.

\item \tt{-Wswitch-enum}, as the name suggests, applies when the
controlling expression of a \tt{switch} statement has an enumeration type.
Warnings are generated if any constant of that enumeration type has been omitted
as a case label, or if a case label is not a constant of that enumeration.
A \tt{default} case does not affect this option.

\item \tt{-Wwrite-strings} changes the type of string literals to array
of \tt{Char} instead of \tt{Char_}, and consequently, warnings are
generated when string literals are converted as pointer to \tt{Char_}.
In a sense, this option changes the rules of C language to a small extent,
and can generate warnings even for strictly conforming programs:
this is because the C standard specifies the type of string literals as array
of \tt{Char_}, even though updating that array causes undefined behavior.
Another point of concern is that \tt{-Wwrite-strings} can silently change the
behavior of type-sensitive code; for example, consider a \tt{_Generic} expression
whose controlling expression is a string literal, and there are two selection
expressions: one associated with \tt{Char *} and another with \tt{Char_ *}.

\note\qquad The minor inconsistency between the type of string literals
and their non-modifiability dates back before the \tt{const}
qualifier was standardized by the ANSI committee in C89/C90;
``fixing'' this rule in the language standard at a late stage can be
counterproductive and cause constraint violations for legacy codebases.

\end{itemize}

The options starting with \tt{-Wno-} are used to disable
specific warnings, most of which are considered harmless.

\begin{itemize}

\item \tt{-Wno-override-init} disables warnings when designated initializers
use multiple expressions to initialize a structure or union member.
Disabling the warning is necessary because the reference
implementation uses designated initializers to provide the feature
of named arguments for method invocations using \tt{call_} family.

\item \tt{-Wno-missing-field-initializers} is used to disable warnings
about the absence of an explicit initialization of structure members,
which by default are initialized as if with the integer constant \tt{0}
(null pointer for members with pointer type).
This warning has been disabled to permit default arguments for method
invocations when named arguments are not used; as mentioned before,
the latter is implemented using designated initializers.

\end{itemize}

\enlargethispage*{\baselineskip}
\pagebreak

\subsection{\tt{gcc} options}
The following additional options are used
for invoking \tt{gcc} in our build script.

\begin{itemize}

\item \tt{-ftrack-macro-expansion} is an option for the preprocessor
\tt{cpp} that controls location tracking of preprocessing tokens
that undergo nested macro invocations, and this information is shown
in diagnostic messages when a macro invocation causes an error.
As the foundation of the C\_ reference implementation is
built upon the preprocessor, even a minor typographic mistake
in code can trigger an avalanche of preprocessing errors.
A detailed stack trace report of macro expansions can be
beneficial for finding bugs in the reference implementation itself;
however, they are of limited interest to most programmers, and excessive
verbosity can overwhelm beginners about the precise cause of an error,
which can be as minor as an extra comma in a macro invocation.
Setting this option to zero disables it, thereby
limiting the depth of preprocessing error messages.

\item \tt{-Wduplicated-branches} warns when two blocks of
a conditional expression or statement have identical code.

\item \tt{-Wduplicated-cond} warns when two mutually exclusive branches,
such as \tt{if} and \tt{elif} (short for \tt{else if}), have conditions
whose values can be statically determined to be identical.
This makes the code guarded by the \tt{elif} unreachable:
when the condition is satisfied for \tt{if} branch, then \tt{elif} branch
will not execute due to mutual exclusion, and when the condition fails
for \tt{if} branch, then it will also fail for \tt{elif} branch as well.

This programming fallacy be demonstrated with a concrete example,
as shown in this contrived code snippet.

\code{Build/Compiler flags/unreachable.c_}

The above function cannot execute \tt{return 1} due to the guarding
\tt{elif} having the same condition as its preceding \tt{if}.
However, declaring the parameter as \tt{volatile Int n} suppresses the warning,
as it tells the compiler that the value of \tt{n} can possibly change from zero
to non-zero between the branches, due to external sources of mutation (such as
another concurrent thread of execution updating this value via a pointer to it).

\item \tt{-Winit-self} warns about initializing an uninitialized
variable with its indeterminate value, such as \tt{Int n = n;}

\item \tt{-Wnull-dereference} warns about execution paths
that can lead to dereferencing a possibly null pointer.
This is an example of a warning that is enabled by optimizations: in this case,
\tt{-Wnull-dereference} comes into effect when \tt{-fdelete-null-pointer-checks}
is active, which is enabled by \tt{-O2} and higher optimizations.

\item \tt{-Wshift-overflow=2} warns about integer overflow for bitwise left shift
operations; in particular, setting it to \tt{2} enables warning about shifting
a \tt{1} to sign bit position, when the promoted left operand has a signed type.

\item \tt{-Wstrict-overflow=2} warns about signed overflows
in cases where \tt{gcc} assumes that overflow will not occur.

\end{itemize}

The following options starting with \tt{-Wno-} are used to
disable certain warnings turned on by \tt{-Wall} or \tt{-Wextra}.

\begin{itemize}

\item \tt{-Wno-parentheses} is used to forgo redundant parentheses
in some expressions, notably with \tt{iff} and \tt{implies}.

\item \tt{-Wno-tautological-compare} is used to disable few
false positive warnings for the reference implementation.

\item \tt{-Wno-type-limits} disables warnings on range
checking conditions that are always true on most environments.

\end{itemize}


\subsection{\tt{clang} options}
The following additional options are used
for invoking \tt{clang} in our build script.

\begin{itemize}

\item \tt{-fmacro-backtrace-limit} determines the maximum backtrace depth for
the preprocessing call stack that is implicitly updated for macro invocations.
Setting it to \tt{1} reduces the verbosity of preprocessing error messages,
which is analogous (though not entirely equivalent) to
the option \tt{-ftrack-macro-expansion=0} for \tt{gcc}.

\item \tt{-Wassign-enum} when an enumeration type lvalue
is assigned a value other than its enumeration constants.

\item \tt{-Wshift-sign-overflow} warns when a left shift operation with a signed
left operand (after type promotion) might shift a \tt{1} to the sign bit position;
this option is analogous to the setting \tt{-Wshift-overflow=2} in \tt{gcc}.

\item \tt{-Wunreachable-code-aggressive} activates several diagnostic
options for detecting code points that cannot be reached during execution.
Removing such code reduces object file size without changing functional behavior.

\item \tt{-Wno-pointer-arith} is used to disable some false
positive warnings for address arithmetic on pointer to VLA.

\end{itemize}

