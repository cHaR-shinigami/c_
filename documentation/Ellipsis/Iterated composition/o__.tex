The fundamental design technique behind iteration composition of function-like
macros is to separate text generation from macro invocation, which is postponed
with the help of deferred expansion; an actual function can also be used.

\head{Syntax}

\tt{o__ (} \it{function} \tt{,} \it{exponent} \tt{, (} \it{argument-list} \tt{) )}

\head{Constraints}

\it{exponent} shall be a non-negative decimal integer constant
not exceeding \tt{PP_MAX}, or expand to such a constant.

\head{Semantics}

The macro \tt{o__} produces an iterated function composition
text of the form $f(f(\cdots f(arguments)\cdots))$;
if \it{function} is a function-like macro, then it is not invoked.
The number of iterations for the composition is given by \it{exponent}:
if it is zero, then \it{function} is ignored, and the text produced
contains only \it{argument-list} ($f^0$ is identity function).

\note The macro \tt{o__} is named after the mathematical
symbol $\circ$ for function composition, such as $f \circ g$.

\example The macro \tt{turn_} performs a single left
rotation on its expanded argument list, and is defined as:

\begin{center}
\tt{#define turn_(...) pop_(__VA_ARGS__, top_(__VA_ARGS__,))}
\end{center}

It is generalized by the macro \tt{left_}, which rotates
its expanded argument sequence by $n$ positions on the left.

\begin{center}
\tt{#define left_(n, ...) echo_(o__(turn_, n, (__VA_ARGS__)))}
\end{center}

The invocation of \tt{o__} merely generates the iterated function
composition, without invoking the macro \tt{turn_}; for instance,
\tt{o__(turn_, 3, (e, u, r, e, k, a))} expands to
\tt{turn_(turn_(turn_(e, u, r, e, k, a)))}.
To complete each invocation of \tt{turn_}, we need to pass it through
another function-like macro that is not used by \tt{turn_} itself:
we have chosen \tt{echo_} for this purpose.
Recall the two phases during a function-like macro invocation:

\begin{itemize}[nosep]

\item In the first phase, the \tt{echo_} argument
\tt{o__(turn_, 3, (e, u, r, e, k, a))} is expanded to\\
\tt{turn_(turn_(turn_(e, u, r, e, k, a)))}, and this
is substituted in the replacement text of \tt{echo_}.

\item After expanded argument substitution,
\tt{echo_} is moved to \it{\tt{PASSIVE}} state, and the replacement
text is scanned left to right, during which \tt{turn_} is invoked:
each invocation works as expected because it is performed during the argument
substitution phase of \tt{turn_}, when it remains in \it{\tt{ACTIVE}} state.

\end{itemize}

Due to the above steps, an invocation \tt{left_(3, e, u, r, e, k, a)}
would produce the text \tt{e, k, a, e, u, r}.
What makes iterated composition the crown jewel of this dialect is that the
function being composed can itself be a function-like macro that uses iterated
composition: for example, a function-like macro \tt{f} can be the iterated
composition of another function-like macro \tt{g}, whose definition itself can
involve iterated composition of another function-like macro \tt{h}, and so on;
the nesting of iterated composition can continue up to any depth, as long as
the entire expansion does not exhaust resources in the translation environment
(memory footprint is a major concern).

Nesting of iterated composition within the function%
-like macro being iterated works due to two factors:
two-phase scanning and separation of text generation from macro invocation.
For the given example, text generation occurs in the argument substitution phase
of \tt{echo_}, during which the \tt{o__} macro remains in \it{\tt{PASSIVE}}
state; once that is done, \tt{o__} returns to \it{\tt{ACTIVE}} state,
so when the replacement text of \tt{echo_} is scanned in the second phase,
\tt{o__} is already live and gets expanded if it is used within the
function-like macro whose iterated composition is currently being invoked.
The same mechanism works well for further nesting of the depth of
\tt{o__} in the full expansion tree of the outermost macro invocation,
as opposed to the number of iterations at any particular depth.

A crucial observation is that we need a macro to complete the invocations
which were deferred during text generation, and this expander macro itself
cannot be used either directly or indirectly by the function being composed.
This is because the function-like macro supplied to \tt{o__} as the
first argument is invoked in the second phase of the expander macro,
during which the latter remains in \it{\tt{PASSIVE}} state, so if
it is invoked by the iterated macro, the expansion fails and each
occurrence of the expander macro is retained verbatim; furthermore,
all such instances are marked as \it{\tt{DEAD}}, so they cannot be invoked
even if rescanned after the expander macro returns to \it{\tt{ACTIVE}} state.

\head{Recommended practice}

Our example used \tt{echo_} as the expander macro,
as it is not used by the iterated macro \tt{turn_},
which only relies on the primitives \tt{pop_} and \tt{top_}.
In general, it may be considered a good practice to define a new macro
created only to expand the text generated by an invocation of \tt{o__};
such an expander macro can have only a single parameter
that occurs as the sole entity in the replacement text,
since an invocation of \tt{o__} itself acts as one single argument.
This avoids the burden of having to know how the iterated macro works;
more precisely, using an existing macro to complete invocation of the
iterated macro requires knowledge of the complete expansion tree for
the latter, all the way down to the primitives, as one must ensure
that the expander macro is not used either directly or indirectly.

Another disadvantage of using an existing macro is tight coupling,
as the expander macro can no longer be used by the iterated
macro or any of its dependencies in a future refactoring:
the text for iterated composition would still be generated correctly,
but its expansion would not fully work in the second phase.
A workaround is that if an existing macro needs to be introduced
while refactoring another macro, it cannot be used directly;
instead, its functionality can be achieved by using a newly
created macro with the same definition as the existing macro.

\note One of the design goals for the reference implementation is
economy of names: iterated composition is used to generate preprocessing
``microcode'' for almost every non-trivial feature of the C\_ dialect,
and creating a new macro to expand each iterated composition would quickly
cause an unnecessary proliferation of macros, almost doubling in number
(one extra macro for each use of \tt{o__}).
As the definitions of macros are already known,
their iterated composition can be invoked with a function-like macro
that is not used in the resulting expansion tree.

Knowledge about implementation details can be tempting for minimalists,
but many developers may find it beneficial in the long run to
prefer our earlier suggestion of creating a separate expander
macro for each new feature that is to be implemented using iterated composition;
an extra macro can save a lot of hassle in the long game of code maintenance.
Before moving on to the next section, we shall complete this discussion
by revisiting our previous example: instead of \tt{echo_}, we could
have used a separate expander macro \tt{left_c_} as shown below.

\table{l}
\tt{#define left_(n, ...) left_c_(o__(turn_, n, (__VA_ARGS__)))}\\
\tt{#define left_c_(o) o}\\
\elbat

Now we need not worry about the macros used by \tt{turn_},
which can safely use \tt{echo_} without breaking \tt{left_}.
