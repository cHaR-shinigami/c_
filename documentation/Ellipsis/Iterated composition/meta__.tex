\tt{meta__} is another macro for iterated composition, and if the iterated function
happens to be a function-like macro, then the invocations are expanded as well;
ironically, this extra convenience makes \tt{meta__} less useful than \tt{o__}.

\head{Syntax}

\tt{meta__ (} \it{function} \tt{,} \it{exponent} \tt{, (} \it{argument-list} \tt{) )}

\head{Constraints}

\it{exponent} shall be a non-negative decimal integer constant
not exceeding \tt{PP_MAX}, or expand to such a constant.

\head{Semantics}

The macro \tt{meta__} produces an iterated function composition text
as generated by \tt{o__}; if \it{function} is a function-like macro,
then it is invoked when \tt{meta__} is in \it{\tt{PASSIVE}} state.
The number of iterations for the composition is given by \it{exponent}:
if it is zero, then \it{function} is ignored,
and the text produced contains only \it{argument-list}.

\note \tt{meta__} has its own expander macro, and if \it{function}
 uses \tt{meta__}, those instances are marked as \it{\tt{DEAD}}.
