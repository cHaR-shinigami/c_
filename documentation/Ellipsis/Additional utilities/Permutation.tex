\head{Syntax}

\idx{permute__} \tt{( (} \it{argument-list} \tt{) ,} \it{index-list} \tt{)}

\head{Constraints}

The number of elements in the expanded
\it{argument-list} shall be less than \tt{PP_MAX}.

If \it{argument-list} expands to $N$ elements and \it{index-list} expands
to $I$ elements, then for each non-negative integer $i$ less than $N$,
the element at index $i \% I$ in \it{index-list} shall be
a non-negative decimal integer not exceeding \tt{PP_MAX}.

\head{Semantics}

Let the expanded \it{argument-list} be labeled as
$l_0$ \tt{,} $\cdots$ \tt{,} $l_{N - 1}$; similarly the expanded
\it{index-list} can be labeled as $i_0$ \tt{,} $\cdots$ \tt{,} $i_{I - 1}$.
If elements in the \it{index-list} at indices less than
$N$ have values smaller than $N$, then the output is:

\centerline
{$l_{i_0}$ \tt{,} $l_{i_{1\%I}}$ \tt{,} $\cdots$ \tt{,} $l_{i_{(N - 1)\%I}}$}

Equivalently, the output is a list of $N$ elements, where the
element at index $j$ is of the form $l_{i_{j\%I}}$ ($0 \le j < N$);
in other words, the output element at index $j$ is an input
element from the expanded \it{argument-list} at index $i_{j\%I}$.

If $i_{j\%I}$ is not less than $N$, it indicates an invalid index,
and the existing element $l_j$ is retained at index $j$.

\note The name \tt{permute__} is a misnomer, as some elements can be repeated,
and some may not occur at all; a true permutation is obtained if \it{index-list}
is a shuffling of \tt{RANGE_(}$N$\tt{)}, where every index occurs exactly once.
