\head{Syntax}

\idx{sort_} \tt{(} $key$ \tt{,} \it{argument-list} \tt{)}

\head{Constraints}

$key$ shall be a function-like macro that can accept a single argument;
whenever $key$ is invoked with an element of \it{argument-list}, its outcome
shall be a non-negative decimal integer constant not exceeding \tt{PP_MAX}.

The number of elements in the expanded
\it{argument-list} shall be less than \tt{PP_MAX}.

\head{Semantics}

Each element of the expanded \it{argument-list} is mapped to the integer
generated on invoking $key$ with that element, and the expanded list
is sorted as per non-decreasing order of the mapping of its elements.

\textsc{example i}\indent
A plain list of integers can be sorted using \tt{echo_} as $key$; for instance,

\tt{sort_(echo_, RANGE_(0, 9, 2), RANGE_(9, 0, 2))}
expands to \tt{0, 1, 2, 3, 4, 5, 6, 7, 8, 9}.

\textsc{example ii}\indent
A list of triplets can be sorted on the basis of
middle elements using the following macro as $key$:

\centerline{\tt{#define mid_(_args_) echo_(top_ C_(pop_ _args_))}}

For instance, \tt{sort_(mid_, (C, 3, c), (A, 1, a), (B, 2, b))}
gives \tt{(A, 1, a), (B, 2, b), (C, 3, c)}.

\textsc{example iii}\indent
\tt{sort_(INC_, 125, 126, 127)} expands to \tt{127, 125, 126},
as its mapping list is \tt{0, 126, 127}.
