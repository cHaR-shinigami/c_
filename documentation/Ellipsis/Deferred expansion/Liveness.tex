Between argument substitution in first phase and scanning in second phase,
two important operations take place:

\begin{itemize}[nosep]

\item For each occurrence of a macro parameter that is subjected to
the preprocessing operator \tt{#} or \tt{##} in the replacement text,
the corresponding argument is substituted verbatim without scanning for macro
expansion: the \tt{#} operator ``stringifies'' the original text of the argument
when the macro was invoked, and the \tt{##} operator pastes its left and right
tokens as they appeared when the invocation took place (it may have been deferred).

\item The other operation is more relevant to our discussion.
At a conceptual level, the preprocessor maintains an internal
stack of macros whose replacement text is being scanned.
An object-like macro does not have the first phase of argument substitution,
so it is pushed to the stack as soon as it is scanned for replacement.
During the invocation of a function-like macro, it is pushed to the stack
after the first phase is over, and before commencing the second phase.
A macro is popped from the stack after completing its ongoing substitution.

\end{itemize}

This conceptual stack is analogous to the runtime call stack of
activation records; for the preprocessor, this stack maintains the
sequence of active macro invocations that are yet to be completed,
and its significance is that all macros in this stack are rendered ``passive''.
We have used the term ``\it{\tt{ACTIVE}} macros'' while describing the
primitives, and to understand the distinction between active and passive,
we shall introduce the concept of ``liveness'' states.

A macro is said to be ``live'' or in \it{\tt{ACTIVE}} state if it can be
expanded: an object-like macro is live before and after its substitution,
and a function-like macro remains live throughout the first phase of
an invocation, so if the same macro is invoked in one of the arguments,
that expansion works as expected (except with \tt{#} and \tt{##}).
A macro is said to be ``dormant'' or in \it{\tt{PASSIVE}} state if it cannot be
expanded: an object-like macro remains dormant throughout its substitution, and
a function-like macro is becomes dormant when its invocation enters the second phase.

The C preprocessor does not natively support recursion, so unlike a runtime call
stack, the conceptual preprocessor stack always contains unique records of macros
whose expansion is underway, and it can be considered as a ``set'' of dormant macros.
At any stage of macro substitution, if a dormant macro is found while scanning,
then that occurrence of the macro is marked as \it{\tt{DEAD}},
meaning that even if the text is rescanned once again after the
macro becomes live, an instance marked as \it{\tt{DEAD}} still
cannot be expanded; in other words, the text is retained verbatim.

\note \it{\tt{DEAD}} state is applicable to occurrence of a macro
if it is scanned when the macro is in \it{\tt{PASSIVE}} state.

\example The concept of liveness is important to understand iterated composition
with deferred expansion, so we shall clarify it with a concrete example
that covers all the state transitions. Consider the following invocation:

\centerline{\tt{top_(echo_(cat_ C_(echo, _)(0), cat_ top_ C_(,)(echo, _)(0)),)}}

We shall analyze this elaborate macro invocation
from the perspective of a C preprocessor.
Assuming this is directly part of the source text and not in the
replacement text of another macro, initially the stack is empty.

\begin{itemize}[nosep]

\item When \tt{top_} is invoked, its first argument
\tt{echo_(cat_ C_(echo, _)(0), cat_ top_ C_(,)(echo, _)(0))} is scanned
for replacement (recall that \tt{top_} discards rest of the arguments).
\tt{echo_} is then invoked, and its arguments \tt{cat_ C_(echo, _)(0)}
and \tt{cat_ top_ C_(,)(echo, _)(0)} are expanded before substitution.
Note that the stack is still empty, as both \tt{top_} and
\tt{echo_} invocations are currently in their first phase.

\item Both occurrences of \tt{C_} are erased (replaced with empty text),
and the resulting replacement text of \tt{echo_} is
\tt{cat_ (echo, _)(0), cat_ top_ (,)(echo, _)(0)}.
The first phase is complete for \tt{echo_}, so it is pushed onto the stack
and its second phase begins; \tt{echo_} is now in \it{\tt{PASSIVE}} state,
while \tt{top_} is still in \it{\tt{ACTIVE}} state.

\item While scanning the replacement text of  \tt{echo_},
\tt{cat_} is invoked which produces the token \tt{echo_}
(assuming \tt{echo} and \tt{_} are not defined as non-trivial object-like
macros lest they be replaced in the first phase, just like the \tt{C_} macro).
Now \tt{cat_} is pushed onto the stack, so we have two
macros in \it{\tt{PASSIVE}} state: \tt{echo_} and \tt{cat_}.

\item While scanning the replacement text of \tt{cat_},
the pasted token \tt{echo_} is found to be in \it{\tt{PASSIVE}} state,
so its occurrence is marked as \it{\tt{DEAD}};
this completes the expansion of \tt{cat_}, and it is popped off the stack.

\item Coming back to the second phase of \tt{echo_},
scanning the replacement text continues,
and \tt{top_} is invoked next, which simply expands to the empty text.
At the end of its second phase,
outcome of the \tt{echo_} invocation is \tt{echo_(0), cat_(echo, _)(0)}.
\tt{echo_} is popped and stack becomes empty:
all macros are in \it{\tt{ACTIVE}} state.

\item The outcome of \tt{echo_} is nothing but the expanded first
argument of \tt{top_}, which is then substituted in its replacement text.
The first phase being over, \tt{top_} is moved to
\it{\tt{PASSIVE}} state and its second phase commences.

\item Even though \tt{echo_} is currently in \it{\tt{ACTIVE}} state,
the occurrence of \tt{echo_} that was marked as
\it{\tt{DEAD}} earlier is not expanded, and left as it is.
\tt{cat_} is invoked next which again produces the token
\tt{echo_}, but this time \tt{echo_} is live, so this instance
can be invoked later (note that \tt{cat_} itself is pushed and then
popped from the stack, with its second phase happening in between,
when both \tt{top_} and \tt{cat_} are in \it{\tt{PASSIVE}} state).

\item The expansion of \tt{cat_} is followed by an opening
parenthesis, so the pasted token \tt{echo_} gets invoked.
The final outcome is the text \tt{echo_(0), 0} (one instance of
\tt{echo_} is unexpanded as it had been marked as \it{\tt{DEAD}}).

\end{itemize}

\note This example uses doubly deferred expansion for the
invocation of \tt{cat_} in the second argument of \tt{echo_}.
For the sake of brevity, we shall not elaborate on higher levels of deferring,
as these are increasingly complex and found to be of limited use; the basic
deferred expansion proves insufficient only in very rare circumstances.
