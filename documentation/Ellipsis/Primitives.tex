The ellipsis framework forms the core of the reference implementation,
and its architecture is strongly influenced by the microprogramming
approach to processor design; in a sense, function-like macros for
arithmetic, relational, and logical operations are analogous to micro-programs,
which are executed by the preprocessor that acts as a ``microcode engine''.
Each non-trivial computation is performed as a sequence
of primitive operations that act as micro-instructions.
The elementary operations on macro arguments are done with macros
defined in the header \idx{<primitives._>}, which is the topmost root
ancestor of the entire header hierarchy for the reference implementation.

\note \tt{cat_} and \tt{echo_} are named after UNIX commands;
\tt{pop_} and \tt{top_} are named after stack operations.

\subsection{\idx{C_}}
\input Ellipsis/Primitives/C_

\subsection{\idx{COMMA}}
\input Ellipsis/Primitives/COMMA

\subsection{\idx{cat_}}
\input Ellipsis/Primitives/cat_

\subsection{\idx{echo_}}
\input Ellipsis/Primitives/echo_

\subsection{\idx{pop_}}
\input Ellipsis/Primitives/pop_

\subsection{\idx{top_}}
\input Ellipsis/Primitives/top_
