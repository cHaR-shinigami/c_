\head{Syntax}

\tt{FILTER_ (} \it{predicate} \tt{,} \it{arg2} \tt{,} \it{argument-list} \tt{)}

\head{Constraints}

Each element in \it{predicate} shall be a function-like macro that can be
invoked with two arguments, and its outcome shall be \tt{0} or \tt{1} when
the first argument is from \it{argument-list} and the second from \it{arg2},
as described in semantics.

If \it{arg2} is a list with multiple elements after expansion,
the resulting element count shall be less than \tt{PP_MAX}.

The number of arguments in the expanded
\it{argument-list} shall be less than \tt{PP_MAX}.

\head{Semantics}

For each element in the expanded \it{argument-list}, a function-like macro
from \it{predicate} is invoked with that element as the first argument, and
the second argument is an element from \it{arg2} after peeling and expansion
of \it{arg2}: if outcome is \tt{1}, then the element from \it{argument-list}
is included; otherwise outcome is \tt{0} and the element is excluded.

If \it{predicate} or \it{arg2} expands to a list with multiple elements, the
elements are used in a round robin sequence, moving back to the first element each
time the list ends; extra elements beyond the number of iterations are unused.
\it{arg2} is also subjected to \tt{peel_}, which removes the
outermost parentheses from an optionally parenthesized prefix.

\example The invocation \tt{FILTER_(echo_(MOD_, LT_), (2, 10), 2, 3, 5, 7, 11, 13)}
extracts all odd integers at even indices, and single digits at odd indices;
the resulting expansion generates the text \tt{3, 5, 7, 11}.

\note As per our naming conventions, a function-like macro named
in uppercase suggests that its outcome is a list of constants.
The naming of \tt{FILTER_} is an exception,
since its outcome need not be a list of constants.
The name \tt{filter_} is already used for a similar iterator that works with
runtime arrays, so the metaprogramming variant was named as \tt{FILTER_};
moreover, \tt{FILTER_} is closely associated with \tt{SEARCH_},
so the naming is justified.
