\tt{omni__} is perhaps the most versatile higher order function in C\_, as the
other iterator macros can be implemented with it; however, it is not as powerful
as iterated composition, since a function-like macro given to \tt{omni__}
cannot itself invoke \tt{omni__} or any other macro implemented with it.
The superiority of \tt{o__} is due to its support of nested composition,
where a function-like macro being iterated can itself iterate over another
function-like macro using \tt{o__}, and this pattern can continue up to
any depth that does not exhaust resources of the translation environment.

\pagebreak

\head{Syntax}

\tt{omni__ (} \it{left-arg} \tt{,} $f$ \tt{,} \it{right-list} \tt{)}

\head{Constraints}

Number of elements in the expanded \it{right-list} shall be
less than \tt{PP_MAX}, after removal of optional parentheses.

If \it{left-arg} expands to a list with multiple elements,
the same constraint shall also apply to its element count.

\head{Semantics}

The \tt{peel_} macro is applied on \it{left-arg}
and \it{right-list}, and their results are expanded.
Both \it{left-arg} and $f$ are single arguments that can expand to a list with
multiple elements; the third argument onward are part of \it{right-list}.
\it{left-arg} and \it{right-list} can be optionally parenthesized:
each parenthesized list acts as a single argument,
and the outermost parentheses are removed by \tt{peel_}.
Length of the longer expanded list determines number of iterations.

The lists are scanned left to right starting from their first elements.
In each iteration, $f$ is invoked with two arguments:
current element of \it{left-arg} is the first argument,
and current element of \it{right-arg} is the second argument.
Current element of \it{left-arg} is replaced with the
outcome of current invocation, and the index moves
one step forward in each list for the next iteration.
Whenever the shorter list ends, its index is moved back to the beginning;
in other words, the shorter list is logically circular.
If $f$ expands to a list of functions, then that list is also considered
circular, and each function is invoked in a round robin sequence starting from
the left; if the number of functions is more than the number of iterations,
then the extra functions remain unused, but all of them are scanned as usual.

Formally, let the elements of \it{left-arg} be
labeled as $l_0$ \tt{,} $\cdots$ \tt{,} $l_{L-1}$;
similarly, the elements of \it{right-list} are
labeled as $r_0$ \tt{,} $\cdots$ \tt{,} $r_{R-1}$.
$L$ and $R$ denote the number of elements,
and let $N$ denote maximum of the two.
Considering $f$ to be a list, let its elements be labeled
in the same way as $f_0$ \tt{,} $\cdots$ \tt{,} $f_{F-1}$,
where $F$ denotes the number of functions.
With these notations, we can denote an invocation of \tt{omni__}
with the following form (\tt{echo_} is only an example):

\centerline
{
\tt{omni__ ( (}  $l_0$ \tt{,} $\cdots$ \tt{,} $l_{L-1}$
\tt{) , echo_ (} $f_0$ \tt{,} $\cdots$ \tt{,} $f_{F-1}$
\tt{) ,}  $r_0$ \tt{,} $\cdots$ \tt{,} $r_{R-1}$ \tt{)}
}

An invocation of the above form performs $N$ iterations,
generating a comma-separated list of the following form:

\centerline
{
$f_0$ \tt{(} $l_0$ \tt{,} $r_0$ \tt{) ,}
$f_{1\%F}$ \tt{(} $l'_{1\%L}$ \tt{,} $r_{1\%R}$ \tt{) ,}
$\cdots$ \tt{,}
$f_{(N - 1)\%F}$ \tt{(} $l'_{(N - 1)\%L}$ \tt{,} $r_{(N - 1)\%R}$ \tt{)}
}

Only the rightmost $L$ elements of the above list are retained
in the final output list: each element is of the form
$f_{i\%F}$ \tt{(} $l'_{i\%L}$ \tt{,} $r_{i\%R}$ \tt{)}, where $i$ ranges
from $N - L$ through $N - 1$, and $\%$ gives the remainder after division.
$l'$ denotes the current element in \it{left-arg}, which is updated
in each iteration: initially, each $l'$ is same as the original
element in \it{left-arg}, and whenever it is used in an iteration,
it is replaced with the element generated in that iteration.
In each case, if $f_{i\%F}$ is a function-like macro other than \tt{omni__},
it is invoked while \tt{omni__} remains in \it{\tt{PASSIVE}} state.

\note If \it{left-arg} contains fewer elements than \it{right-list},
intermediate results get used in subsequent iterations.
If a function-like macro in $f$ produces the name \tt{omni__},
that instance is marked as \it{\tt{DEAD}} and cannot be invoked.

\textsc{example i}\indent
When both sides have equal number of elements, \tt{omni__} pairs each element
of \it{left-arg} with the index-wise corresponding element in \it{right-list},
and invokes (an element of) $f$ with that pair of arguments.
For instance, \tt{omni_((di, sa), cat_, (na, ur))} expands to
\tt{cat_(di, na), cat_(sa, ur)}, and then \tt{dina, saur}.
Deferred expansion works as usual,
so \tt{echo_(cat_ C_(omni__((di, sa), cat_, (na, ur))))} produces \tt{dinasaur}.

\textsc{example ii}\indent
All elements of \it{right-list} can be
aggregated with a single element of \it{left-arg}.
For instance, \tt{omni__(0, ADD_, RANGE_(11, 20, 2))}
computes the sum of all odd numbers from 11 to 20, which is 75.

\textsc{example iii}\indent
The element in a singleton \it{right-list} can
be applied to each element of \it{left-arg}.
For instance, \tt{omni__(RANGE_(10), echo_(ADD_, MUL_), 2)} iterates over
integers from 0 through 9, adding 2 to each even number and multiplying each
odd number by 2: the resulting text is \tt{2, 2, 4, 6, 6, 10, 8, 14, 10, 18}.

\head{Recommended practice}

We shall reiterate our earlier cautionary note on the use of \tt{peel_} macro:
if the argument to \tt{peel_} is not fully
parenthesized but contains a parenthesized proper prefix,
then \tt{peel_} can silently change the meaning of a code.
For example, the text \tt{(a + b) * c} should never be used as \it{left-arg}
or \it{right-list}, as \tt{peel_} will alter it to \tt{a + b * c}.

A good practice is to fully enclose the actual list within a single
pair of parentheses, which is removed by \tt{omni__} using \tt{peel_};
often it will be redundant, but this habit can help
to avert potentially disastrous changes by \tt{peel_}.
