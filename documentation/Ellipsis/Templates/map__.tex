\tt{map__} implements the higher order function
\it{map} for element-wise transformation of a list.

\head{Syntax}

\tt{map__ (} $f$ \tt{,} \it{argument-list} \tt{)}

\head{Constraints}

The number of arguments in the expanded
\it{argument-list} shall be less than \tt{PP_MAX}.

\head{Semantics}

Let the expanded form of $f$ be labeled as $f_0$ \tt{,} $\cdots$ \tt{,} $f_{F-1}$;
similarly, the expanded form of \it{argument-list}
is denoted as $a_0$ \tt{,} $\cdots$ \tt{,} $a_{N-1}$.
With this notation, the output of \tt{map__}
is represented as a list of the following form:

\centerline
{
$f_0$ \tt{(} $a_0$ \tt{) ,}
$f_{1\%F}$ \tt{(} $a_1$ \tt{) ,}
$\cdots$ \tt{,}
$f_{(N - 1)\%F}$ \tt{(} $a_{N - 1}$ \tt{)}
}

Each element of the output list has the form $f_{i\%F}$ \tt{(} $a_i$ \tt{)};
if $f_{i\%F}$ is a function-like macro, then it is expanded.

\note \tt{map__} can be used to parenthesize elements of
a list simply by leaving $f$ as blank (comma is required).

\example \tt{map__(echo_(INC_, DEC_), RANGE_(10))} increments each integer
at even indices, and decrements those at odd indices; for the given list
\tt{RANGE_(10)}, the generated text is \tt{1, 0, 3, 2, 5, 4, 7, 6, 9, 8}.
