\tt{fold__} implements the higher order function \it{fold} for combining
multiple elements of a list into a single result using an aggregation
function that can accept two arguments: if the list contains more elements,
the function is called again with the result of its previous invocation
as one argument, the other argument being the next input element.

\head{Syntax}

\tt{fold__ (} $f$, \it{argument-list} \tt{)}

\head{Constraints}

The number of arguments in the expanded
\it{argument-list} shall be less than \tt{PP_MAX}.

\head{Semantics}

Let the expanded form of $f$ be labeled as $f_0$ \tt{,} $\cdots$ \tt{,} $f_{F-1}$;
similarly, the expanded form of \it{argument-list}
is denoted as $a_0$ \tt{,} $\cdots$ \tt{,} $a_{N-1}$.
With this notation, the output of \tt{fold__}
can be represented with the following form:

\centerline
{
$f_{(N - 2)\%F}$ \tt{(} $f_{(N - 3)\%F}$ \tt{(} $\cdots$ $f_{2\%F}$ \tt{(}
$f_{1\%F}$ \tt{(} $f_0$ \tt{(} $a_0$ \tt{,} $a_1$ \tt{) ,} $a_2$ \tt{) ,}
$a_3$ \tt{)} $\cdots$ \tt{,} $a_{N - 2}$ \tt{) ,} $a_{N - 1}$ \tt{)}
}

In other words, if \it{argument-list} expands to a list of $N$ elements
($N > 1$), then there are $N - 1$ iterations: intermediate results are used
in subsequent stages until all elements are combined into a single result.
The first invocation is of the form $f_0$ \tt{(} $a_0$ \tt{,} $a_1$ \tt{)};
if the list has more than two elements, then the result of each
invocation is used as the first argument in the next invocation,
and the next unprocessed element is used as the second argument.

\note \tt{fold__} aggregates the expanded \it{argument-list}
by processing its elements from left to right.

\example \tt{fold__(echo_(ADD_, SUB_), RANGE_(10))} alternately adds and
subtracts non-negative integers below 10; in effect, it subtracts the sum of
odd integers from the sum of even integers in given range, resulting in \tt{5}.
