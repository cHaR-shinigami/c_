\tt{reduce__} is also an aggregating macro similar to \tt{fold__},
except that the elements are combined right to left.

\head{Syntax}

\tt{reduce__ (} $f$, \it{argument-list} \tt{)}

\head{Constraints}

The number of arguments in the expanded
\it{argument-list} shall be less than \tt{PP_MAX}.

\head{Semantics}

Let the expanded form of $f$ be labeled as $f_0$ \tt{,} $\cdots$ \tt{,} $f_{F-1}$;
similarly, the expanded form of \it{argument-list}
is denoted as $a_0$ \tt{,} $\cdots$ \tt{,} $a_{N-1}$.
With this notation, the output of \tt{fold__}
can be represented with the following form:

\centerline
{
$f_{(N - 2)\%F}$ \tt{(} $a_0$ \tt{,} $f_{(N - 3)\%F}$ \tt{(}$a_1$ \tt{,} $\cdots$
$f_{2\%F}$ \tt{(} $a_{N - 4}$ \tt{,} $f_{1\%F}$ \tt{(} $a_{N - 3}$ \tt{,} $f_0$
\tt{(} $a_{N - 2}$ \tt{,} $a_{N - 1}$ \tt{)} \tt{)} \tt{)} $\cdots$ \tt{)} \tt{)}
}

The first invocation is of the form
$f_0$ \tt{(} $a_{N - 2}$ \tt{,} $a_{N - 1}$ \tt{)}; thereafter the result
of each invocation is used as the second argument in the next invocation,
and the previous unprocessed element is used as the first argument.

\note Even though \tt{reduce__} processes its \it{argument-list}
right to left, the function list $f$ is used left to right.

\example \tt{reduce__(SUB_, 3, 2, 1)} performs
\tt{SUB_(3, SUB_(2, 1))}, whose outcome is \tt{2}.
On the contrary, \tt{fold__(SUB_, 3, 2, 1)} works left to
right as \tt{SUB_(SUB_(3, 2), 1)}, and the result is \tt{0}.
