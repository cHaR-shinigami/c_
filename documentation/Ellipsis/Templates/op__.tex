The \tt{op__} family is modeled after the semantics of \tt{omni__},
except that \tt{op__} is intended to be used with operators.
Each element in the operand list(s) is parenthesized,
and each element in the output list is also fully parenthesized.

\head{Syntax}

 \tt{op__}\s\s\s\s\tt{(} \it{left-arg} \tt{,} $op$ \tt{,} \it{right-list} \tt{)}

 \tt{op__}\s\s\s\s\tt{(} $op$ \tt{,} \it{list-arg} \tt{)}

\idx{op__0__}\s\tt{(} \it{left-arg} \tt{,} $op$ \tt{,} \it{right-list} \tt{)}

\idx{op__2__}\s\tt{(} $op$ \tt{,} \it{list-arg} \tt{)}

\head{Constraints}

Number of elements in \it{list-arg} or \it{right-list} shall be
less than \tt{PP_MAX}, after removal of optional parentheses.

If \it{left-arg} expands to a list with multiple elements,
the same constraint shall also apply to its element count.

$op$ shall be a single argument that expands to an optionally parenthesized
list with less than \tt{PP_MAX} elements; $op$ needs to be parenthesized for
\tt{op__} if the unparenthesized form of $op$ expands to more than one element.

\note \it{list-arg} means a single argument that expands to
a list after removing optional parentheses with \tt{peel_}.

\head{Semantics}

Arguments to \tt{op__} are counted after excluding the first
argument in its unexpanded form, and rest of the arguments
are expanded: if the count is 1, then \tt{op__2__} is invoked;
otherwise the count is more than one, and  \tt{op__0__} is invoked.
\tt{op__} considers $op$ to be a singleton list: it should not expand to
more than one element, as rest of them are considered to be initial elements
of \it{right-list} ($op$ should be parenthesized if it has multiple elements).

\it{left-arg}, \it{right-list}, and \it{list-arg} are peeled and
expanded in the way described in the semantics of \tt{omni__}.

$op$ is also subjected to the \tt{peel_} macro, and let
$P$ be the number of elements after peeling and expansion.

\begin{itemize}
\item \tt{op__0__ (} \it{left-arg} \tt{,} $op$ \tt{,} \it{right-list} \tt{)}
\end{itemize}

After peeling \it{left-arg}, each of its elements is parenthesized, and let
the resulting list be labeled as $l_0$ \tt{,} $\cdots$ \tt{,} $l_{L - 1}$;
similar steps are also performed for \it{right-list}, whose parenthesized
elements can be labeled as $r_0$ \tt{,} $\cdots$ \tt{,} $r_{L - 1}$.
Note that $l_i$ and $r_i$ represent elements
after they are parenthesized by \tt{op__0__}.
Let $N$ be the maximum of $L$ and $R$.
\tt{op__0__} generates a list of the following form:
its rightmost $L$ elements are parenthesized and retained in output.

\begin{center}
$l_0\ op_0\ r_0$ \tt{,}
$l'_{1\%L}\ op_{1\%P}\ r_{1\%R}$ \tt{,}
$\cdots$ \tt{,}
$l'_{(N - 1)\%L}\ op_{(N - 1)\%P}\ r_{(N - 1)\%R}$
\end{center}

$l'$ denotes the current element in \it{left-arg}, which is
updated in each iteration: initially, each $l'$ is same as
the parenthesized element from \it{left-arg}, and whenever it
is used in an iteration, it is replaced with the new element.

\note A newly generated element is not immediately parenthesized to avoid
a proliferation of excessive \mbox{parentheses} in intermediate steps,
whenever \it{left-arg} is shorter than \it{right-list}
and generated elements are used in subsequent iterations.
Only after the entire output list has been generated,
its last $L$ elements are parenthesized.

\begin{itemize}
\item \tt{op__2__ (} $op$ \tt{,} \it{list-arg} \tt{)}
\end{itemize}

After peeling \it{list-arg}, let its expanded elements be labeled as
$l_0$ \tt{,} $\cdots$ \tt{,} $l_{N - 1}$; output of \tt{op__2__} has the form:

\begin{center}
\tt{(} $op_0$ \tt{(} $l_0$ \tt{) ) ,}
\tt{(} $op_{1\%P}$ \tt{(} $l_1$ \tt{) ) ,}
$\cdots$ \tt{,}
\tt{(} $op_{(N - 1)\%P}$ \tt{(} $l_{N - 1}$ \tt{) )}
\end{center}

Equivalently, the output is a list of $N$ parenthesized elements,
each having the form \tt{(} $op_{i\%P}$ \tt{(} $l_i$ \tt{) )} ($0 \le i < N$).\\

\textsc{example i}\indent
When both sides have equal number of elements, \tt{op__2__} pairs each element
of \it{left-arg} with the index-wise corresponding element in \it{right-list},
and places (an element of) $op$ in between them.
For instance, \tt{op__((0 + 1, 2 + 3), (*, /), (4 + 5, 6 + 7))}
expands to \tt{((0 + 1) * (4 + 5)), ((2 + 3) / (6 + 7))}.

\textsc{example ii}\indent
All elements of \it{right-list} can be
aggregated with a single element of \it{left-arg}.
For instance, \tt{op__(0, +, a, b, c)} generates the
parenthesized expression \tt{((0) + (a) + (b) + (c))}.

\textsc{example iii}\indent
The element in a singleton \it{right-list} can
be applied to each element of \it{left-arg}.
For instance, \tt{op__((i, j, k), *, p + q)} generates the
list \tt{((i) * (p + q)), ((j) * (p + q)), ((k) * (p + q))}.

\textsc{example iv}\indent
\tt{op__2__} can be used to apply a unary operator to each element of a list.
For instance, \tt{op__(-, (x + 1, y + 2, z + 3))} gives the
negation of each element as \tt{(-(x + 1)), (-(y + 2)), (-(z + 3))}.
