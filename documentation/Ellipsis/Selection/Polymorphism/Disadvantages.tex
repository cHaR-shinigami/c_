Generalization achieved with the use of polymorphic macros
has certain fundamental limitations in the design itself.

\begin{itemize}[nosep]

\item The preprocessing ``microcode'' of \tt{COUNT_} uses an
incrementing strategy to count the arguments one by one,\\
which incurs a small transpilation overhead; in fact, any other implementation
of static polymorphism based on argument counting will have some name
resolution overhead, even though it may be considered insignificant.

\item If \it{function-identifier} is defined as a non-trivial object-like macro,
then pasting the expanded arguments would not produce the required name.
For example, if a polymorphic macro \tt{fn_(...)} is meant to invoke \tt{fn2}
or \tt{fn3}, the definition \tt{poly_(fn, COUNT_(__VA_ARGS__))(__VA_ARGS__)}
has a subtle issue: if \tt{fn} is defined as an object-like
macro that expands to \tt{func}, then \tt{fn_(0, 1)} would
expand to \tt{func2(0, 1)} instead of \tt{fn2(0, 1)}.

\item The same issue also affects \it{suffix}: for example, the reference
implementation does not use \tt{__} directly as \it{suffix} with \tt{poly_3_},
as \tt{__} is a reserved identifier in C, and compilers can define it
as a non-trivial object-like macro, which would result in an incorrect
name mangling when \tt{__} expands to something else before token pasting.

\item When a polymorphic macro invokes another macro
obtained with name mangling, the latter receives expanded
arguments that may look unrecognizable when stringified.
Many debuggable features of C\_ are polymorphic in nature (such as \tt{assert_}),
and it is desirable that the original text is printed in a diagnostic message.
The reference implementation takes extra precautions to
retain the original argument text in a diagnostic message.

\end{itemize}

Due to these subtleties, the reference implementation only provides, but rarely
makes use of polymorphic macros: avoiding argument counting overhead in
the source code leads to a noticeable improvement in preprocessing time.
