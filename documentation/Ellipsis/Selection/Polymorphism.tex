\def\Subsubsection#1{\subsubsection{#1}\input{Ellipsis/Selection/Polymorphism/#1}}

The \tt{poly_} family of macros can be used to emulate static polymorphism via
name mangling on the basis of argument count in an invocation, which can be
determined with the \tt{COUNT_} macro (arguments are expanded before counting).

\note \tt{poly_} is itself a polymorphic macro
based on the same working principle: name mangling.

\head{Syntax}

\idx{poly_}\s\s\s\tt{(}
\it{function-identifier}
  \tt{,} \it{argument-count}
 [\tt{,} \it{suffix}
 [\tt{,} \it{upper-bound}]]
  \tt{)}

\idx{poly_2_}\s\tt{(}
\it{function-identifier}
  \tt{,} \it{argument-count}
  \tt{)}

\idx{poly_3_}\s\tt{(}
\it{function-identifier}
  \tt{,} \it{argument-count}
\l\tt{,} \it{suffix}\r
\ \tt{)}

\idx{poly_4_}\s\tt{(} \it{function-identifier}
  \tt{,} \it{argument-count}
\l\tt{,} \it{suffix}
\l\tt{,} \it{upper-bound}\r\r
\ \tt{)}

\head{Constraints}

Pasting the expanded forms of \it{function-identifier}, \it{argument-count},
and \it{suffix} (in that order) shall not produce an invalid token.
\it{upper-bound} shall expand to a decimal integer constant more than 1 and not
exceeding \tt{PP_MAX}; if\\\it{upper-bound} is given, \it{argument-count} shall
expand to a non-negative decimal integer constant not exceeding \tt{PP_MAX}.

\head{Semantics}

\tt{poly_} invokes \tt{poly_}$n$\_ if the
expanded argument sequence contains $n$ arguments.
\tt{poly_2_} and \tt{poly_3_} paste their expanded arguments in the given order.
For \tt{poly_4_}, if \it{argument-count} is less than \it{upper-bound},
the outcome is same as \tt{poly_3_}; otherwise \it{argument-count} is
replaced with \tt{0}, and pasting is done the same way as \tt{poly_3_}.

\note The \tt{poly_} family merely generates a
mangled name that has to be invoked separately.
\it{argument-count} is typically supplied via
\tt{COUNT_}, whose outcome is always non-zero.
For \tt{poly_4_}, the special argument count of zero is used to
designate a default function/macro meant to be invoked when argument
count exceeds a certain value; in other words, a function/macro should
be available for each supported argument count below \it{upper-bound}.

\example \tt{RANGE_(...)} can be implemented as
\tt{poly_(RANGE_, COUNT_(__VA_ARGS__), _)(__VA_ARGS__)}.

\Subsubsection{Disadvantages}
