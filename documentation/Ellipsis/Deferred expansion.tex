\def\Subsection#1{\subsection{#1}\input{Ellipsis/Deferred expansion/#1}}

Deferred expansion is main working principle behind the ellipsis framework.
A function-like macro is invoked only when the macro name is followed by a left
parenthesis (with optional whitespace in between); the invocation is suppressed
if this condition is not satisfied, such as by parenthesizing the macro name.
As opposed to not invoking a function-like macro at all,
deferred expansion simply postpones the macro invocation:
one possible approach is to place a transient token between the
macro name and the opening parenthesis before the argument sequence.

To illustrate the idea behind deferred expansion, let us consider the \tt{cat_}
macro which pastes its two arguments without scanning them for macro expansions.
\tt{cat_(cat_(0, 0), 7)} does not work as expected because
pasting the arguments verbatim produces \tt{)7} as a bad token.
We can get the desired outcome by deferring the outer call as:

\begin{center}
\tt{echo_(cat_ C_(cat_(0, 0), 7))}
\end{center}

The above expansion works due to a relatively obscure preprocessing rule:
the substitution of a function-like macro invocation involves two phases
of scanning for macro expansions in the arguments and the replacement text.

\begin{itemize}[nosep]

\item In the first phase, each occurrence of a macro parameter is replaced by its
corresponding argument, and if it is not subjected to the operators \tt{#} and
\tt{##}, then the argument itself is scanned for macros prior to substitution.

\item After substituting the arguments (expanded when applicable), \tt{#} and
\tt{##} operations are performed (if present), and the resulting replacement
text is scanned from left to right for macro expansions in the second phase.

\end{itemize}

In our example, the first phase expands the argument
\tt{cat_ C_(cat_(0, 0), 7)}: the outer invocation of \tt{cat_} is
deferred by the macro \tt{C_}, which expands to the empty text;
the second invocation of \tt{cat_} works as usual, producing the token \tt{00}.
The expanded argument is substituted in the replacement
text of \tt{echo_}, producing the text \tt{cat_(00, 7)}.
This text is scanned for macro expansions in the second phase: as the
obstruction \tt{C_} was erased during the first phase, the outer invocation
of \tt{cat_} takes place as usual, producing the expected outcome \tt{007}.

In summary, our intent in this example was to expand the argument of \tt{cat_}
prior to token pasting, as the expansion is suppressed by the \tt{##} operator;
to accomplish this, we simply deferred the outer invocation of \tt{cat_} until
its arguments were expanded during the first phase (argument substitution).
The purpose of \tt{echo_} here is to complete the invocation deferred by \tt{C_};
without that we would end up with a partially expanded text \tt{cat_(00, 7)}.
\tt{echo_} and \tt{C_} are nothing special; in fact, we could have used
any function-like macro other than \tt{cat_} to perform the full expansion.
For instance, \tt{top_(cat_ C_(cat_(0, 0), 7),)}
expands to the same text as before: \tt{007}.

\note Deferred expansion came to be realized through a
slightly different approach with two macros \tt{L} and \tt{R}.

\centerline{\tt{#define L (}}
\centerline{\tt{#define R )}}

With this alternative technique, the earlier example
would be written as \tt{echo_(cat_ L cat_(0, 0), 7 R)}.

\Subsection{Liveness}
