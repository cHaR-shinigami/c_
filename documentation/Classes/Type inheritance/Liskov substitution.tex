C\_ supports Liskov substitution, allowing a derived class object to be used
wherever its base class object is expected: this rule works recursively,
so the instance of any class can be treated as an instance of each
of its ancestor classes, up to the root ancestor \tt{Object} class.
The common practice is to operate on instances via pointers, and since each
class structure implicitly contains its base structure as the first member, the
pointer representation can be directly interpreted as pointing to an instance
of the base structure, or recursively to an instance of any ancestor class.

The function \tt{is_type} only checks whether a
pointer refers to a valid \tt{Type} structure or not.
To check whether an object can be considered as a valid instance of
a given class, the required conditions are tested by the \tt{validate}
method provided by that class (or inherited from its base class).
However, for a C\_ program to work correctly with Liskov substitution,
it is necessary that each valid instance of a derived class is also a valid
instance of all of its ancestor classes; this is done by the \tt{validate}
function, whose precise behavior is described in a later section.

\enlargethispage*{\baselineskip}
\pagebreak
