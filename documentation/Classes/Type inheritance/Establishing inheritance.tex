In most object-oriented programming languages, the term ``inheritance'' broadly
means that members of a base type are also implicitly members of any of its
derived types, which includes both member attributes and member methods.
The derived type can suppress an inherited method with its own implementation,
which is usually known as ``method overriding''; this technique does
not forbid accessing the base class method via other mechanisms.

C\_ relies on macros to establish inheritance between two
object-oriented types: this is not merely limited to the
reference implementation, but is part of the specification.
The macro and replacement text need to be defined as:

\begin{center}

\tt{# define} $Derived$\tt{_EXTENDS} $Base$ [, \textit{override-list}]

\end{center}

In the above syntax, a type named as $Derived$
inherits from another  type named as $Base$.
The optional \textit{override-list} can only specify a comma-separated
list of method names that are part of the \tt{Type} structure.

\note C\_ does not specify any mechanism to disallow
inheritance, though implementations can support it.
