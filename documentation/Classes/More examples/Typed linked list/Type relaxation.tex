The method \tt{species} is used to obtain and possibly relax the
existing \tt{species} of a \tt{List} instance to an ancestor type.
We have left the protocol definition as an exercise, which shall
be implemented in the next chapter using \tt{Collection} interface.
Following are the basic pre-conditions that can be established by the protocol:

\begin{itemize}

\item \tt{this} must point to a valid instance of \tt{List} class.

\item If \tt{species} is not null, then it
must point to a valid \tt{Type} structure.

\end{itemize}

The following procedure code is available in the source file
\src{compile/class/List/species.c_}

If the parameter \tt{species} is a null pointer, then
the existing value of \tt{this->species} is returned.

If the list is empty, then \tt{this->species} is directly updated to \tt{species};
otherwise \tt{this->species} is updated to the nearest common ancestor
of \tt{species} and the existing value of \tt{this->species}.
It ensures that \tt{this->species} is always updated to an ancestor type,
and due to Liskov substitution principle, each existing element
in the list would be a valid instance of the super type.
So a non-empty list would remain valid after relaxing the type.
