In this chapter, we have discussed only the implementation of \tt{init} for the
\tt{List} class; some other procedures will be implemented in the next chapter,
demonstrating how inheritance and abstraction promote code reusability.
We conclude this chapter with an overview of how these
procedures can be implemented (can be done as an exercise).

\begin{itemize}

\item \tt{validate} should check that \tt{is_type(species)}
is \tt{true}, and the \tt{data} member of each node in
the list points to a valid instance of \tt{species}.
Recall that the general \tt{validate} function does validation with
respect to the base type first, so we need not check that the list
structure is valid, as it is already done by the \tt{Chain} class.
To allow instances of any object-oriented type in a list,
its \tt{species} can be initialized to \tt{type_(Object)}.

\item \tt{compare} should adopt a similar rule as \tt{strcmp}: for a pair
of elements, one from each list in identical positions, find their super
type and invoke its \tt{compare} procedure with the instances as arguments.
If the outcome is non-zero, then that is
also the return value of comparing the lists.
Otherwise the same is repeated for successive pairs until one of
the lists ends: if the other list ends as well, then the lists
are considered to be equal, and the return value should be zero;
otherwise the shorter list is considered to be ``less than'' the longer list.

\item \tt{copy} should perform a ``shallow copy'' of the source list: only
the data pointers are copied to nodes of the destination list (allocated as
required), as opposed to invoking \tt{copy} procedure for each instance.
The primary reason behind this is to avoid issues of memory
allocation failure while creating a ``deep copy'' of each instance.

\item \tt{read}, \tt{parse}, and \tt{decode} should each invoke its
corresponding procedure from the \tt{Type} specified by \tt{species}.

\item \tt{write}, \tt{text}, and \tt{encode} should each invoke its
corresponding procedure from the \tt{Type} given by each instance.

\item \tt{add} should concatenate the two lists: elements of
the left list are to be followed by elements of the right list.

\end{itemize}
