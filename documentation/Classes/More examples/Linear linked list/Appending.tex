The protocol for \tt{append} method is left as an exercise in this chapter.
The following conditions can be established:

\begin{itemize}

\item \it{Pre-condition}:
The parameter \tt{this} must point to a valid \tt{Chain} object.

\item \it{Post-condition}:
\tt{this} must point to a valid \tt{Chain} object after appending.
If the procedure returns \tt{false}, then the operation failed,
and all members of \tt{this} must remain unchanged; otherwise a
new node was appended, whose member \tt{data} must be same as the
parameter \tt{data}, and \tt{this->length} must have been incremented.

\end{itemize}

In the next chapter, we shall design the \tt{append} protocol
for \tt{Chain} class with the help of \tt{Iterable} interface.
For now, we shall discuss the following procedure available in the source file
\src{compile/class/Chain/append.c_}

If a new node cannot be allocated, then zero is returned;
otherwise the member \tt{data} is set to the parameter
\tt{data}, and the member \tt{next} is set to a null pointer.
The length is updated, and if its existing value is non-zero,
then the new node is linked to the \tt{next} member of current tail node;
otherwise the chain is currently empty, and the new node becomes the head node.
In any case, the newly appended node always becomes the tail node.
