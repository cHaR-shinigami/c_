\def\Subsubsection#1{\subsubsection{#1}
\input{Classes/More examples/Linear linked list/#1}}

A linked list is a concrete data structure
for representing an ordered sequence of data.
Unlike arrays, a linked list is not required to support direct access of elements:
to access an element, each element that precedes it needs to be visited as well.
This is because the elements are not required to be stored in physically
contiguous locations, but each element stores a pointer to its next element.
Each element is called a node and has two primary components:
a data value, and a pointer to its successor; the latter
establishes a logical continuation between successive elements.
The first node of a list is called the head node, which is a starting point to
reach any node in that list; similarly, the last node is called the tail node,
which does not have any successor for a non-circular linked list implementation.

\note The tail node is useful for appending an element
in constant time, which is often a frequent operation.

\Subsubsection{Declaration}

\enlargethispage*{\baselineskip}
\enlargethispage*{\baselineskip}
\Subsubsection{Definition}
\pagebreak

\Subsubsection{Validation}

\Subsubsection{Constructor}

\Subsubsection{Destructor}

\Subsubsection{Appending}
