\head{Syntax}

\idx{super}\s\s\tt{(} [\tt{. this =}] $this$ \tt{,} [\tt{. that =}] $that$ \tt{)}

\idx{super_}\s\tt{(} \it{oo-identifier-list} \tt{)}

\head{Constraints}

Both the arguments $this$ and $that$ shall be expressions
of type \tt{Type}, optionally specifying parameter names.

\it{oo-identifier-list} shall be a comma-separated list of
names for object-oriented types whose declarations are visible.

The number of elements in \it{oo-identifier-list}
shall be less than \tt{PP_MAX} after macro expansions.

\head{Semantics}

\tt{super (}$this$ \tt{,} $that$\tt{)} is equivalent to the more verbose
invocation \tt{call_((Type, super)} \tt{,} $this$ \tt{,} $that$\tt{)}.

\tt{super_} accepts a list of object-oriented type names, and returns their
nearest common ancestor type; it essentially provides a combination of map and
fold/reduce, by applying \tt{type_} on each name in \it{oo-identifier-list},
and then using fold/reduce with \tt{super} as the aggregator.
If \it{oo-identifier-list} is a singleton, the
outcome is \tt{type_(}\it{oo-identifier-list}\tt{)}.

\note \tt{super} is commutative, so the order of arguments does not affect the
return value; however, the order of evaluation of arguments is unspecified,
so side-effects should generally be avoided in argument expressions.

\example The outcome of \tt{super (type_(Programmer), type_(Phone))}
is same as \tt{type_(Object)}.

The outcome of \tt{super_(Landline, FeaturePhone, SmartPhone)}
is same as \tt{type_(Phone)}.
