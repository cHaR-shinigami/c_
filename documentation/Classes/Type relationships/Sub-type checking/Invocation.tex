The \tt{is} family is used to check for ``is-a''
relationship among object-oriented types.

\head{Syntax}

\idx{is}\s\s\s\tt{(} [\tt{. descendant =}] $descendant$
\tt{,} [\tt{. ancestor =}] $ancestor$ \tt{)}

\idx{is_}\s\s\tt{(} \it{oo-identifier-list} \tt{)}

\idx{is__}\s\tt{(} \it{oo-type-list} \tt{)}

\head{Constraints}

Both $descendant$ and $ancestor$ shall be expressions
of type \tt{Type}, optionally specifying parameter names.

\it{oo-identifier-list} shall be a comma-separated list of
names for object-oriented types whose declarations are visible.

\it{oo-type-list} shall be a comma-separated list of
expressions, each of which shall be of type \tt{Type}.

The number of elements in \it{oo-identifier-list} or \it{oo-type-list}
shall be less than \tt{PP_MAX} after macro expansions.

\head{Semantics}

\tt{is (}$descendant$ \tt{,} $ancestor$\tt{)} is equivalent to the more
verbose \tt{call_((Type, is)} \tt{,} $descendant$ \tt{,} $ancestor$\tt{)}.

\tt{is_} accepts a list of object-oriented type names,
and checks if each type is a descendant of the type after it.

\tt{is__} accepts a list of object-oriented type expressions,
checking if each type is a descendant of the type after it.

Both \tt{is} and \tt{is_} evaluate each argument exactly once; for \tt{is__},
if the number of expanded arguments in \it{oo-type-list} is more than two,
then all arguments except the first and last can be evaluated more than once.

In each case, the outcome in an expression
of type \tt{Bool_}, and it is not an lvalue.

\note ``is-a'' relationship establishes a partial ordering,
being reflexive, anti-symmetric, and transitive.

\example \tt{is(type_(Person), type_(Object))} is \tt{true},
but \tt{is(type_(Object), type_(Person))} is \tt{false}.

\tt{is_(FeaturePhone, Mobile, Phone, Object)} is \tt{true},
which is equivalent to the following conjunction:

\centerline
{\tt{is_(FeaturePhone, Mobile) && is_(Mobile, Phone) && is_(Phone, Object)}}
