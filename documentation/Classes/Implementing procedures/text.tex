The first invocation of \tt{output__} counts the number of characters
that would be present in the decimal form of \tt{this->value};
\tt{1U} is added to make room for a terminating null byte.
If \tt{out} is not a null pointer, it is expected to be a character array having
a capacity of (at least) \tt{*length} elements; if \tt{*length} is not less than
the required buffer length, then \tt{out} is used to store the text representation.
Otherwise a new character array of length \tt{buflen} is allocated;
if the allocation fails, then a null pointer is returned.
However, \tt{*length} is always updated to \tt{buflen}
regardless of whether the allocation succeeds or not; if it fails,
then \tt{*length} indicates the capacity that would be required.
The second invocation of \tt{output__} actually writes the text representation
of \tt{this->value} to the array, and the latter is returned by the function
(pointer to an array has the same address as base element of the array).
