\tt{in} points to an array of \tt{UByte}, whose data is interpreted
as the little-endian representation of an unsigned integer.
The initial value of \tt{*length} is expected to be length of the
encoding array; only the first eight bytes are interpreted, and if
the array is shorter than that, then a null pointer is returned.
Otherwise \tt{*length} is set to eight,
indicating the number of bytes that were interpreted.
The bitwise-AND with \tt{255} is meant for systems where \tt{CHAR_BIT} is
more than eight; for most execution environments, a byte is an octet of bits,
and the bitwise-AND is likely to be optimized away by compilers.
However, the suffixes \tt{UL} and \tt{ULL} are significant, as
they promote the left operand of the shift to (possibly) wider type,
ensuring that there is enough room to accommodate the shifted bits.
