Our implementation of \tt{parse} expects \tt{in} to be a valid string;
the initial value of \tt{*length} is expected to be capacity of the
character array, and if none of its elements match the null byte,
then the input is assumed to be invalid, and a null pointer is returned.
Otherwise the string is interpreted as an unsigned integer using the
standard library function \tt{strtoull} (declared in \tt{<stdlib.h>});
the third argument corresponding to base is specified as zero,
which supports octal (prefixed with \tt{0}) and hexadecimal
(prefixed with \tt{0X} or \tt{0x}) notations along with decimal form.

The modifiable pointer \tt{endptr_} records the address of the first
invalid character that could not be converted; consequently, its offset
relative to the base pointer \tt{in} gives the number of valid characters
that were correctly interpreted, which is stored in \tt{*length}.
\tt{errnum_} (an alternative name for \tt{errno} defined in the
C\_ extension header \tt{<errno._>}) is set to zero before calling
\tt{strtoull}, and if it becomes non-zero after the call, or if zero
characters were read, or the converted value exceeds \tt{UNSIGNED_MAX},
then \tt{errnum_} is reset to zero and a null pointer is returned.
Otherwise the converted value is assigned to \tt{this->value},
and a pointer to the instance is returned.
