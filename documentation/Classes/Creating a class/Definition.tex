Declarations of object-oriented types are typically placed in header files,
as they are required in multiple translation units (wherever the type is used).
On the contrary, every such type must have exactly one definition that can be
compiled: this is because every object-oriented type has an associated \tt{Type}
structure declared as a singleton array (non-modifiable), which must be defined
in exactly one translation unit (source file that is given to the compiler).

\enlargethispage*{\baselineskip}
\pagebreak

\head{Syntax}

\idx{define_}\s\s\s\tt{(} \it{name} [\tt{,} \it{interface-list}] \tt{)}

\idx{define_0_}\s\tt{(} \it{class-name} \tt{,} \it{interface-list} \tt{)}

\idx{define_1_}\s\tt{(} \it{oo-type-name} \tt{)}

\head{Constraints}

\it{interface-list} shall be a comma-separated list of interface
names, declared prior to the declaration of \it{class-name}.

The number of elements in the expanded
\it{interface-list} shall be less than \tt{PP_MAX}.

An object-like macro named as \it{class-name}\tt{_EXTENDS} or
\it{oo-type-name}\tt{_EXTENDS} shall remain defined, and for each
$ancestor$ type up to the \tt{Object} class, a similarly named macro
$ancestor$\tt{_EXTENDS} shall also remain defined.

For each \it{interface-name} in \it{interface-list}, an
object-like macro shall be defined in the following form:

\begin{center}
\tt{# define} \it{class-name}\tt{_IMPLEMENTS_}\it{interface-name}
\it{base-implementation} \tt{,} \it{methods-list}
\end{center}

In the replacement text, \it{base-implementation} shall be name
of the class that implements the base type of \it{interface-name},
as specified by the macro named as \it{interface-name}\tt{_EXTENDS};
as a special rule, if \it{interface-name} directly extends the \tt{Abstract}
type, then \it{base-implementation} shall be specified as \tt{SELF}.
If the macro \it{interface-name}\tt{_EXTENDS} specifies \it{base-interface} as
the base type and \it{base-interface} is not \tt{Abstract}, then an object-like
macro named \it{base-implementation}\tt{_IMPLEMENTS_}\it{base-interface}
shall be defined in the same form as described above.

\it{methods-list} shall be a list of method names that are specified in the
extended type structure declared as \tt{struct Typex (}\it{interface-name}\tt{)};
for each \it{method-name} in \it{methods-list}, an external function named
as \tt{solver_(}\it{class-name}\tt{,} \it{method-name}\tt{)} shall be declared,
and this also applies for the implementation of \it{base-interface}.

\note \tt{is(type_(}\it{class-name}\tt{), type_(}\it{base-implementor}\tt{))}
should be \tt{true}, but not enforced as a constraint.

\head{Semantics}

\tt{define_} invokes \tt{define_0_} if \it{interface-list} is specified;
else \it{interface-list} is omitted and \tt{define_1_} is invoked.

\tt{define_1_} defines a non-modifiable external array
named \tt{type_(}\it{oo-type-name}\tt{)}:
this array is a singleton whose sole element is a \tt{Type} structure,
containing function pointers for the basic procedures of an object-oriented type.

For each procedure name specified in the optional \it{override-list}
in the replacement text of the macro \it{oo-type-name}\tt{_EXTENDS},
the function pointer is used to initialize the corresponding member
in \tt{type_(}\it{oo-type-name}\tt{)}; otherwise the process is
repeated for each ancestor type, and procedures defined by the nearest
ancestor are used to initialize remaining function pointers in the
non-modifiable \tt{Type} structure, defined with static storage duration.

In addition to the facilities provided by \tt{define_1_},
\tt{define_0_} also defines a non-modifiable external
array for each interface specified in \it{interface-list}.
For an interface identified by \it{interface-name}, name of the external array
is given by \tt{typex_(}\it{interface-name}\tt{,} \it{class-name}\tt{)}:
this array is a singleton whose element type is an extended \tt{Type}
structure identified as \tt{struct Typex (}\it{interface-name}\tt{)},
having function pointers for each method of that interface.

\example The wrapper class \tt{Unsigned} is defined in the file
\tt{examples/compile/class/Unsigned.c_}

\code{../compile/class/Unsigned.c_}
