\head{Syntax}

\tt{# define} \it{class-name}\tt{_EXTENDS}
\it{base-class} [\tt{,} \it{override-list}]

\idx{class_}\s\s\s\tt{(} \it{class-name} [\tt{,}
\it{interface-list}] \tt{)} \it{member-declarations} \idx{fin}

\idx{class_0_}\s\tt{(} \it{class-name} \phantom{[}\tt{,}
\it{interface-list}\phantom{]}  \tt{)} \it{member-declarations} \tt{fin}

\idx{class_1_}\s\tt{(} \it{class-name} \tt{)} \it{member-declarations} \tt{fin}

\head{Constraints}

\it{base-class} shall be the name of another class that has
 been  declared prior to the declaration of \it{class-name}.

\it{interface-list} shall be a comma-separated list of interface
names, declared prior to the declaration of \it{class-name}.

The number of elements in the expanded
\it{interface-list} shall be less than \tt{PP_MAX}.

The identifiers \tt{base} and \tt{type} shall not be
used as member names in \it{member-declarations}.

The optional \it{override-list} shall be a comma-separated
list of method names from the \tt{Type} structure.

\head{Semantics}

\tt{class_} invokes \tt{class_0_} if \it{interface-list} is specified;
otherwise \it{interface-list} is omitted and \tt{class_1_} is invoked.

\tt{class_1_} declaration provides member details
of a class structure whose tag is \it{class-name}.
\tt{class_1_} declares two synonyms: \it{class-name}\_ is a synonym for
\tt{struct} \it{class-name}, and \it{class-name} is its non-modifiable counterpart.
The first implicit member of a class is named as \tt{base}, which is
an instance of \it{base-class} as specified in the replacement text
of \it{class-name}\tt{_EXTENDS}; more precisely, the member \tt{base}
is declared before \it{member-declarations} as a singleton array whose
element type is \tt{struct} \it{base-class} (or simply \it{base-class}\_).
Each class structure also has a member named \tt{type} that is inherited from
the \tt{Object} class: it is a pointer to a non-modifiable \tt{Type} structure.

\tt{class_1_} also declares procedures for the \tt{Type} methods
prefixed with \it{class-name}, along with a non-modifiable external
array whose name is given by \tt{type_(}\it{class-name}\tt{)}:
the array is a singleton whose element is a \tt{Type} structure.

In addition to the facilities provided by \tt{class_1_},
\tt{class_0_} also declares a non-modifiable external
array for each interface specified in \it{interface-list}.
For an interface identified by \it{interface-name}, name of the external array
is given by \tt{typex_(}\it{interface-name}\tt{,} \it{class-name}\tt{)}:
this array is a singleton whose element type is an extended \tt{Type}
structure identified as \tt{struct Typex (}\it{interface-name}\tt{)},
having function pointers for each method of that interface.

\note Interfaces provide a mechanism to extend the basic
\tt{Type} structure, discussed in the next chapter.

\example A wrapper class provides structural encapsulation
of a basic data type, and describes fundamental operations
supported on it with the help of \tt{Type} methods.
As our first example, we shall declare a simple wrapper class named
\tt{Unsigned}, whose only non-trivial member is \tt{ULLong_ value}
(\tt{base} and \tt{type} are trivially present).

\code{../include/class/Unsigned._}

The above class declaration is available in the file
\tt{Unsigned._} located in \tt{examples/include/class} directory.
The macro \tt{Unsigned_EXTENDS} is required by the \tt{class_}
declaration, which establishes an inheritance relationship
between \tt{Unsigned} class and \tt{Object} class; more precisely,
the member \tt{base} of \tt{struct Unsigned} is of type \tt{Object_}.
Successive names in the replacement text of \tt{Unsigned_EXTENDS}
comprise the optional \it{override-list}, specifying the basic
\tt{Type} methods that are implemented by the \tt{Unsigned} class (more
precisely, it is the procedures that are being implemented, not the protocols).
Note that \tt{free} is missing in the list, which means that \tt{Unsigned}
uses the same procedure that is inherited from \tt{Object} class
(recall that the \tt{free} procedure implemented by \tt{Object} class
calls the standard library function \tt{free} for memory deallocation).
The word \tt{fin} marks the end of the class declaration.

The macro \tt{UNSIGNED_MAX} is later used in validation;
for maximum portability, the value of $2^{64} - 1$ is used.

\note For naming structure and union members, we shall
relax the C\_ convention of a trailing underscore.
