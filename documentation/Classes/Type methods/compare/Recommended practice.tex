The result of comparison between two distinct instances should follow the same
convention as prescribed for the comparator function required by standard library
functions \tt{bsearch} and \tt{qsort} (both are declared in \tt{<stdlib.h>}).

\begin{itemize}[nosep]

\item If the first instance is considered to be less than
the second instance, then the outcome should be negative.

\item If both instances are considered to be equal (though
not necessarily same), then the outcome should be zero.

\item If the first instance is considered to be more than
the second instance, then the outcome should be positive.

\item Otherwise both the instances are considered to be
incomparable, and the outcome can be any positive value.

\end{itemize}

\vspace{\baselineskip}

If a derived type overrides \tt{compare} procedure of its base type,
it should be a refinement in the following sense:

\begin{itemize}[nosep]

\item If two instances can be compared with the base type,
they should remain comparable for the derived type.

\item If two instances are considered unequal by the base type,
they should remain unequal for the derived type.

\end{itemize}
