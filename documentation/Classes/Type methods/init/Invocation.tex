\head{Syntax}

\idx{init_}\s\s\s\s\s\tt{(} \it{oo-type-name} \tt{)}

\idx{init_1_}\s\s\s\tt{(} \it{oo-type-name} \tt{)}

\idx{init_}\s\s\s\s\s\tt{(} [\tt{. type =}]
$type$ \tt{,} [\tt{. tape =}] $tape$ \tt{)}

\idx{init_2_}\s\s\s\tt{(}   [\tt{. type =}]
$type$ \tt{,} [\tt{. tape =}] $tape$ \tt{)}\\

\idx{init__}\s\s\s\s\tt{(} [\tt{. type =}] $type$ \tt{,} \it{argument-list} \tt{)}

\idx{init__0__}\s\tt{(} [\tt{. type =}] $type$ \tt{,} \it{argument-list} \tt{)}

\idx{init__}\s\s\s\s\tt{(} \it{oo-type-name} \tt{)}

\idx{init__1_}\s\s\tt{(} \it{oo-type-name} \tt{)}

\enlargethispage*{\baselineskip}
\pagebreak

\head{Semantics}

\tt{init_} invokes \tt{init_}$n$\_ if the expanded
argument sequence contains $n$ arguments.
\tt{init_1_(}\it{oo-type-name}\tt{)} is equivalent
to the verbose call \tt{((}\it{oo-type-name}\_ \tt{*)
call_((Type, init), type_(}\it{oo-type-name}\tt{), (Tape)\{NULL\}))}.

\tt{init_2_(}$type$\tt{,} $tape$\tt{)} is equivalent
to \tt{call_((Type, init),} $type$\tt{,} $tape$\tt{)};
the outcome is of type \tt{Void_ *}.

\tt{init__} invokes \tt{init__1_} if the expanded argument
sequence is a singleton; otherwise it invokes \tt{init__0__}.

\tt{init__1_} is equivalent to \tt{init_1_}.

\tt{init__0__(}$type$\tt{,} \it{argument-list}\tt{)} is same as \tt{init_2_(}%
$type$\tt{,} \tt{(Tape)\{map__(&lvalue__,} \it{argument-list}\tt{), NULL\})}.
In other words, each value in \it{argument-list} is converted into an lvalue
(as if with \tt{lvalue__}), pointers to these lvalues are packed into an array,
and a terminating null pointer is appended to it (to mark the end);
each lvalue and the array itself have automatic storage duration,
whose lifetime ends after the innermost block where the invocation is performed.
Each value in \it{argument-list} can be evaluated
more than once only if it has a variably modified type.
