\tt{validate} returns \tt{true} if and only if
all of the following conditions are satisfied:

\begin{itemize}

\item \tt{this} is not null, and
\tt{is_type(((Object *) this)->type)} is \tt{true}.

\item If \tt{type} is not null, then \tt{is_type(type)}
and \tt{is(((Object *) this)->type, type)} are \tt{true};
in other words, \tt{this} should point to an instance whose \tt{type}
attribute is a descendant or sub-type of the \tt{type} parameter.

\item The \tt{validate} procedure is invoked for each \tt{Type} in the
lineage from \tt{Object} class to the \tt{type} attribute of \tt{this}
instance, $i.e.$ starting from the \tt{Object} class and ending with
\tt{((Object *) this)->type)}; if any of them returns \tt{false},
then that is the outcome, and successive calls are not performed.
Also, if a derived type inherits \tt{validate} procedure from its base
type without overriding it, then that procedure is called only once.

\end{itemize}

\note The order of invoking \tt{validate} procedures is intended to be aligned
with Liskov substitution principle: every valid instance of a derived type is
required to be a valid instance for its base type, and this applies transitively.
