\head{Syntax}

\idx{text__}\s\s\s\s\tt{(} [\tt{. this =}] $this$ [[\tt{,}
[\tt{. length =}] $length$] \tt{,} [\tt{. out =}] $out$] \tt{)}

\idx{text__1_}\s\s\tt{(} [\tt{. this =}] $this$ \tt{)}

\idx{text__2__}\s\tt{(}  [\tt{. this =}] $this$
\l\l\tt{,} [\tt{. out =}]\r $out$\r \tt{)}

\idx{text__3_}\s\s\tt{(} [\tt{. this =}] $this$ \l\l\tt{,}
[\tt{. length =}] $length$\r\ \tt{,} [\tt{. out =}] $out$\r\ \tt{)}\\

\idX{text_}\s\s\s\s\s\tt{(} [\tt{. this =}] $this$ [[\tt{,}
[\tt{. length =}] $length$] \tt{,} [\tt{. out =}] $out$] \tt{)}

\idx{text_1_}\s\s\s\tt{(} [\tt{. this =}] $this$ \tt{)}

\idX{text_2_}\s\s\s\tt{(} [\tt{. this =}] $this$
\l\l\tt{,} [\tt{. out =}]\r $out$\r \tt{)}

\idx{text_3_}\s\s\s\tt{(} [\tt{. this =}] $this$ \l\l\tt{,}
[\tt{. length =}] $length$\r\ \tt{,} [\tt{. out =}] $out$\r\ \tt{)}

\head{Semantics}

\tt{text__} invokes \tt{text__1_} if the argument sequence expands to a singleton.
If the argument sequence expands to two elements, then \tt{text__2__} is invoked;
otherwise there shall be three arguments, and \tt{text__3_} is invoked.

\tt{text__1_ (}$this$\tt{)} is equivalent to
\tt{text__3_(}$this$\tt{, &(Size_)\{1\}, NULL)}.

\tt{text__2__(}$this$\tt{,} $out$\tt{)} is equivalent to
\tt{text__3_(}$this$\tt{, &lvalue__(length__(}$out$\tt{)),} $out$\tt{)}.

\tt{text__2__} can evaluate the argument $out$ more
than once only if it has a variably modified type.

\tt{text__3_(}$this$\tt{,} $length$\tt{,} $out$\tt{)} is equivalent to the verbose
invocation \tt{call_((Type, text),}  $this$\tt{,} $length$\tt{,} $out$\tt{)}.\\

The \tt{text_}* family evaluates each argument exactly once;
rest of the semantics are identical to \tt{text__} family.
So \tt{text_1_} is equivalent to \tt{text__1_}, \tt{text_2_}* is similar
to \tt{text__2__}, and \tt{text_3_} is equivalent to \tt{text__3_}.
