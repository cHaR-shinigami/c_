\fbox{\tt{
examples/include/class/Chain/append._}}
\code{../include/class/Chain/append._}

Before calling the \tt{append} protocol of \tt{Iterable} interface,
the concrete type is set to \tt{typex_(Iterable, Chain)},
which contains the function pointer for the \tt{append} procedure
of \tt{Chain} class (and other concretizing procedures as well).
The concrete instance is wrapped into an instance of \tt{Iterable} using
\tt{abstract_}, which is passed to the \tt{append} protocol of \tt{Iterable}.
The first argument is given as the parameter \tt{_site} instead of \tt{SITE};
the latter would give the source code coordinates of the current call site,
whereas \tt{_site} ensures that the original call site of the current
invocation is used in diagnostic messages for any pre-condition violation.
Calling the \tt{append} protocol of \tt{Iterable} ensures that the concrete
instance is checked against its pre-conditions and post-conditions.

In addition to that, the concrete protocol also
ensures that the following conditions are satisfied:

\begin{itemize}

\item \it{Pre-condition}: \tt{this} must be a valid instance of
\tt{Chain} class, or one of its sub-classes (Liskov substitution).

\item \it{Post-condition}: If \tt{data} was successfully
appended to \tt{this}, then it must be at the \tt{tail} node.

\end{itemize}

Before returning the success status, \tt{this->type} is
restored to its former value that was stored in a local variable.
