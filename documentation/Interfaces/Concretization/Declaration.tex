\head{Syntax}

\tt{# define} \it{class-name}\tt{_EXTENDS}
\it{base-class} [\tt{,} \it{override-list}]

\tt{# define} \it{class-name}\tt{_IMPLEMENTS_}\it{interface-list}
\it{base-implementation} \tt{,} \it{methods-list}

\tt{class_}\s\s\s\tt{(} \it{class-name} [\tt{implements}
\it{interface-list}]   \tt{)} \it{member-declarations} \idx{fin}

\tt{class_0_}\s\tt{(}  \it{class-name} \l\tt{implements}
\it{interface-list}\r\ \tt{)} \it{member-declarations}  \tt{fin}

\tt{prototype_ (} [\it{return-type} \tt{,}]\s\tt{(} \it{class-name}
\tt{,} \it{member-name} \tt{)} [\tt{,} \it{parameter-tuples}]\s\tt{)}\\

\head{Constraints}

\it{base-implementation} shall be name of the class that implements
the base type of \it{interface-name}, as specified by another macro
named as \it{interface-name}\tt{_EXTENDS}; as a special constraint,
if \it{interface-name} directly extends the \tt{Abstract} type,
then \it{base-implementation} shall be specified as the word \idx{SELF}.
If the macro \it{interface-name}\tt{_EXTENDS} specifies \it{base-interface} as
the base type and \it{base-interface} is not \tt{Abstract}, then an object-like
macro named \it{base-implementation}\tt{_IMPLEMENTS_}\it{base-interface}
shall be defined in the same form as described in the syntax.

\it{methods-list} shall be a comma-separated list of member names declared
in \tt{struct Typex (}\it{interface-name}\tt{)}, and for each such member
a corresponding prototype shall be declared, prefixed with \it{class-name}.
Each element in \it{methods-list} shall be a
\it{member-name} corresponding to a function pointer in
\tt{struct Typex (}\it{interface-name}\tt{)}, or a parenthesized pair of the form
\tt{(} \it{typex-member} \tt{,} \it{class-method} \tt{)}, where \it{typex-member}
is a member name declared in \tt{struct Typex (}\it{interface-name}\tt{)}, and
\it{class-method} is the unprefixed name declared in prototype of a class method.

Other constraints are precisely the same as documented
for class declarations in \S 7.8.1 of the previous chapter.

\note \it{member-name},
\tt{(}\it{member-name}\tt{)}, and \tt{(}\it{member-name}\tt{,} \it{member-name}\tt{)}
are equivalent in \it{members-list}.

\head{Semantics}

\idx{implements} is defined as an object-like macro that expands to a single comma.
When the argument sequence of \tt{class_} expands to multiple arguments,
it invokes \tt{class_0_}: the first argument is considered as name of a
concretizing class, and subsequent arguments are the names of interfaces.
For each \it{interface-name} specified in \it{interface-list},
\tt{class_0_} declares a non-modifiable external array named
\idx{typex_}\tt{(}\it{interface-name}\tt{,} \it{class-name}\tt{)},
whose element type is \tt{const struct Typex (}\it{interface-name}\tt{)}.
Rest of the semantics are identical to that of \tt{class_1_} (see \S 7.8.1).

\enlargethispage*{\baselineskip}
\enlargethispage*{\baselineskip}
\enlargethispage*{\baselineskip}

\note The macros named as \it{class-name}\tt{_IMPLEMENTS_}\it{interface-list}
are required only for class definitions; they are placed in header files because
another macro \it{base-implementation}\tt{_IMPLEMENTS_}\it{base-interface}
is required by the class definition (recursively for each ancestor);
the latter is made available in the header of \it{base-implementation}.

\example The \tt{Chain} class implementing a linked list is
a good candidate for concretizing \tt{Iterable} interface.

The following class declaration is available in the header file
\src{include/class/Chain._}

We are already familiar with most of this code,
as it is almost entirely similar to the version in previous chapter.

\pagebreak

New additions are inclusion of the header \tt{"Iterable._"},
along with the macro \tt{Chain_IMPLEMENTS_Iterable} that specifies
the members of \tt{struct Typex (Iterable)}, and for each member,
we have also declared a prototype for the class method
whose procedure provides the concrete implementation.
\tt{SELF} is used as \it{base-implementation} because the
\tt{Iterable} interface directly extends \tt{Abstract},
for which we do not have a separate concretizing class.

Recall that the header \tt{"Iterable._"} declares \tt{Iterator}
as a synonym for the incomplete type \tt{struct Iterator},
whose pointer type is used in the \tt{Iterable} prototypes for
\tt{iterator}, \tt{has_next}, \tt{get_next}, and \tt{duplicate}.
The opaque type \tt{Iterator} for traversing through an \tt{Iterable}
is concretized by \tt{Node} for traversing through a \tt{Chain}.
