The \tt{abstract_} family can be used to wrap a class
instance within a new instance of an interface that is
concretized by the class, or by one of its ancestor classes;
this wrapped class instance can be used with methods of the interface.

\note Since all interfaces inherit from the \tt{Abstract} type
(either directly or indirectly), the concretizing class of any
interface must have an ancestor class that concretizes a direct
descendant of \tt{Abstract} (it can be the same class itself).
Therefore, the instance of any class can be wrapped as
the \tt{concrete} member of an \tt{Abstract} instance.

\head{Syntax}

\idx{abstract_}\s\s\s\tt{(} [\it{interface-name}\tt{=Abstract} \tt{,}] \it{expression} \tt{)}

\idx{abstract_1_}\s\tt{(} \it{expression} \tt{)}

\idx{abstract_2_}\s\tt{(} \phantom{[}\it{interface-name} \tt{,}\phantom{]} \it{expression} \tt{)}

\head{Constraints}

\it{expression} shall be pointer to an object type.

\head{Semantics}

\tt{abstract_} invokes \tt{abstract_}$n$\_ if the
expanded argument sequence contains  $n$ arguments.

\tt{abstract_1_(}\it{expression}\tt{)} is equivalent to
\tt{abstract_2_(Abstract,} \it{expression}\tt{)}.

\tt{abstract_2_} creates an lvalue of type \tt{struct} \it{interface-name},
whose \tt{type} member is initialized with \tt{type_(}\\\it{interface-name}\tt{)},
and \tt{concrete} member is initialized with $expression$.
This lvalue has automatic storage duration, and its lifetime
is limited to the nearest enclosing block where it is created.

\note If \it{expression} points to an instance of class \it{class-name},
then the wrapped instance can be correctly used with methods of
\it{interface-name} only if \tt{(}\it{expression}\tt{)->type} points to
\tt{typex_(}\it{interface-name}\tt{,} \it{class-name}\tt{)},
which provides the necessary concrete procedures (as function pointers);
otherwise the behavior is undefined.
