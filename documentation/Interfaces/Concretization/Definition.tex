For a concretizing class \it{class-name}, the translation unit that
defines the array \tt{type_(}\it{class-name}\tt{)} also defines
\tt{typex_(}\it{interface-name}\tt{,} \it{class-name}\tt{)} for each
\it{interface-name} specified in the class definition;
it should be precisely the same set of instances that are specified in
the class declaration (sequence of interface names does not matter).

\head{Syntax}

\tt{define_}\s\s\s\tt{(} \it{class-name}
 [\tt{implements} \it{interface-list}]   \tt{)}

\tt{define_0_}\s\tt{(} \it{class-name}
\l\tt{implements} \it{interface-list}\r\ \tt{)}

\tt{define_1_}\s\tt{(} \it{class-name}   \tt{)}

\head{Constraints}

The constraints are precisely the same as documented for
class definitions in \S 7.8.2 of the previous chapter.

\head{Semantics}

The semantics are precisely the same as documented for class
definitions in \S 7.8.2 (\tt{implements} expands to a comma).
In particular, for each \it{interface-name} in \it{interface-list},
\tt{define_0_} defines the non-modifiable external array named as
\tt{typex_(}\it{interface-name}\tt{,} \it{class-name}\tt{)} that
is also declared (but not defined) by the class declaration.

Additionally, a single definition line of the form
\tt{define_(}\it{class-name}\tt{,} \it{interface-A}\tt{,} \it{interface-B}\tt{)}
can be split as \tt{define_(}\it{class-name}\tt{,} \it{interface-A}\tt{)}
\tt{define_(}\it{class-name}\tt{,} \it{interface-B}\tt{)},
since both forms are equivalent.

\tt{define_(}\it{class-name}\tt{,} \it{interface-name}\tt{)}
initializes members of the sole element of the array
\tt{typex_(}\it{interface-name}, \it{class-name}\tt{)},
which is of type \tt{struct Typex (}\it{interface-name}\tt{)}.
This initialization requires an object-like macro named as
\it{class-name}\tt{_IMPLEMENTS_}\it{interface-name}, whose replacement text
specifies name of the \it{base-implementation} class that concretizes the base
interface of \it{interface-name}, and \it{members-list} specifies the member
names of \tt{struct Typex (}\it{interface-name}\tt{)} to be initialized.
For each expanded element in \it{members-list}:

\begin{itemize}

\item If the element is an identifier of the form \it{member-name}
(optionally parenthesized), then the structure member \tt{.}\it{member-name}
is initialized with function pointer for the procedure
\tt{solver_(}\it{class-name}\tt{,} \it{member-name}\tt{)}.

\item Otherwise it is a parenthesized pair of identifiers of the form
\tt{(}\it{typex-member}\tt{,} \it{class-method}\tt{)}, which initializes
the member \tt{.}\it{typex-member} with function pointer for the procedure
\tt{solver_(}\it{class-name}\tt{,} \it{class-method}\tt{)}.

\end{itemize}

The characterizing distinction between \tt{type_(}\it{class-name}\tt{)} and
\tt{typex_(}\it{interface-name}\tt{,} \it{class-name}\tt{)} is that for the
latter, the member \tt{self} points to \tt{type_(}\it{class-name}\tt{)},
whereas \tt{type_(}\it{class-name}\tt{)->self} points to itself.

\note Required procedure names are internally declared by the \tt{define_}
family, using the function types specified by \tt{prototype_} declarations.
The parenthesized form \tt{(}\it{typex-member}\tt{,} \it{class-method}\tt{)}
is useful when the implementing class method is declared with a
different name than name of the interface method being implemented.
This situation is inevitable when a single class concretizes multiple interfaces
and at least one member name is common between their respective \tt{Typex}
structures, but declared with incompatible function pointer types.

\enlargethispage*{\baselineskip}
\enlargethispage*{\baselineskip}
\enlargethispage*{\baselineskip}

\example The following definitions are available in the source file
\src{compile/class/Chain.c_}
