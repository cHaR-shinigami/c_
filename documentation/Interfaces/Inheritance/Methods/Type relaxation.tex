To store instances of any object-oriented type,
the argument for \tt{species} can be specified as \tt{type_(Object)}.

\head{Protocol}

\noindent\fbox{\tt{
examples/include/interface/Collection/species._}}
\code{../include/interface/Collection/species._}

\enlargethispage*{\baselineskip}
\enlargethispage*{\baselineskip}

The pre-conditions ensure \tt{this} must be a valid instance of \tt{Collection},
and if \tt{species} is not null, then it must be a valid \tt{Type}.
The post-conditions ensure that the updated type is also valid,
and if the parameter \tt{species} was null,
then the collection's type must remain same as before.
For an empty collection, a non-null \tt{species} parameter
must be directly used to update the collection's type.
Otherwise the collection is non-empty and its updated type must be
nearest common ancestor of the prior type and \tt{species} parameter;
rationale of using the super type is to avoid a non-empty collection from getting
invalidated on account of some element not being an instance of \tt{species}.

\head{Procedure}

\noindent\fbox{\tt{
examples/compile/interface/Collection/species.c_}}
\code{../compile/interface/Collection/species.c_}
