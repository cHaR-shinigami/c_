The following declaration is available in the header file
\src{include/interface/Collection._}

\tt{Collection} introduces an additional method \tt{species},
whose purpose is to associate a \tt{Type} with every instance of \tt{Collection}.
Each data pointer stored in a collection must refer to a valid instance of
the \tt{Type} associated with that collection; this is also a validation
condition in the overriding \tt{validate} procedure of \tt{Collection}.

The benefit of associating a type with every instance of \tt{Collection} is that
it allows us to override most of the basic \tt{Type} procedures inherited from the
\tt{Iterable} interface, which in turn inherits them from the \tt{Abstract} type.
For example, the \tt{write} procedure iterates over a \tt{Collection} instance,
and since each element has its own \tt{type} member, the \tt{write} procedure of
that \tt{type} can be invoked to write the instance data on a given output stream.
Similar abstract designs have been adopted for implementing most of the other
\tt{Type} methods, and concretizing classes can use the dependency inversion
technique to utilize the interface procedures without any non-trivial logic.
