Before running this program, we need to compile the source files of all
classes and interfaces used in this code, along with their base types.
To avoid recompiling all source files for any change we make in our program,
it is suggested that the dependencies should be compiled only once,
and their resulting object files should be stored in \tt{object/} directory:
this task can be automated with the shell script \tt{examples/build.sh}
that needs to be executed once to populate the \tt{object/} directory.
Once that is done, we can compile our program and link the object files as:

\begin{center}

\tt{cc_ compile/merry.c_ -xnone object/lib.o object/*/*.o object/*/*/*.o}

\end{center}

The option \tt{-xnone} (contraction of \tt{-x none}) is used before the object
files to undo the effect of \tt{-xc} option that is part of the \tt{cc_} alias:
\tt{-x c} tells the compiler (\tt{gcc} or \tt{clang} in our tests) to consider
subsequent files with any extension as \tt{C} programs, which is not the case
for object files (otherwise recognized by their filename extension \tt{.o}).

If things go well with the compilation,
executing the program (as \tt{./a.out}) should print the following output.

\code{Interfaces/More examples/Re-concretization/output.txt}
