\def\Subsubsection#1{\subsubsection{#1}
\input{Interfaces/More examples/Re-concretization/#1}}

We conclude our discussion on interfaces with a program for testing various
classes and interfaces that are provided as examples in this documentation.
To test the \tt{Collection} interface, we instantiate different wrapper
classes and append their instances to a collection concretized
by the \tt{List} class, which specifies the format of storage.

The abstract nature of the \tt{copy} procedure of \tt{Collection} naturally
facilitates a format conversion from one concrete type to another, which we refer
to as ``re-concretization'' of an existing instance (a shallow copy is created).

The following program is available in the source file \src{compile/merry.c_}

The numerous guard clauses throughout this program ensure
that the program terminates if any allocation fails;
readers may ignore this cautious bit of defensive programming.
We instantiate the wrapper classes \tt{Unsigned}, \tt{Signed}, and \tt{Rational}
by calling their respective constructors with an initializer data; for the
wrapper class \tt{Text}, we first create an empty instance, and then invoke the
\tt{parse} method (via \tt{parse__}) to copy the given string to the instance
(note that it expects pointer to an array, so the address-of
operator \tt{&} has been used on the string literal).

\enlargethispage*{\baselineskip}
\pagebreak

If all of the above operations are successful, we instantiate the
\tt{Collection} interface with \tt{List} as the concretizing class; note
that the second argument to \tt{init__} is made into an lvalue, and a pointer
to it becomes the first element of \tt{tape} parameter for \tt{init} method.
Since a \tt{species} has not been specified, the \tt{List} constructor
assigns \tt{type_(Object)} as the default value, which allows the four
instances of diverse wrapper classes to be stored in the same collection.
If all the \tt{append} operations are successful, we print the name of the
concretizing class, followed by writing the collection data to \tt{stdout}
(the default output stream).
The \tt{write} procedure of \tt{Collection} does not use the \tt{species}
type to print the elements: since each element is an instance, it has its own
\tt{type} member, whose \tt{write} procedure is invoked for printing each
particular instance; successive elements are separated by a newline.\\

Next we create another empty instance of \tt{Collection},
but this time we use \tt{Vector} as the concretizing class.
Notice the additional argument \tt{type_(Object)} which explicitly states
\tt{species} of the new collection; this is required because the \tt{Vector}
class implements both \tt{Collection} as well as its base interface
\tt{Iterable} (which is untyped), so in the absence of an explicit argument
for \tt{species}, it would be set to null, making it an invalid instance
of \tt{Collection} (as per its \tt{validate} procedure), thereby causing
a post-condition violation for \tt{init} protocol of \tt{Type}.\\

Our task of re-concretizing the \tt{List}-based collection to a
\tt{Vector}-based one is accomplished by simply calling the \tt{copy} method,
with \tt{vector} as the destination and \tt{list} as the source.
To confirm that a successful re-concretization has indeed occurred, we print the
concrete type name of the \tt{Vector}-based instance, and then print its elements.
The outcome would be precisely the same as before,
because even though \tt{Collection} itself does not impose any ordering, both
\tt{List} and \tt{Vector} store their elements as per the order of insertion,
so the sequence is preserved.

\Subsubsection{Compilation}
