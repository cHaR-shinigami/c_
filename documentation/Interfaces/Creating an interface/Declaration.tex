\head{Syntax}

\tt{# define} \it{interface-name}\tt{_EXTENDS}
\it{base-interface} [\tt{,} \it{override-list}]

\idx{Interface_} \tt{(} \it{interface-name} \tt{)}

\tt{prototype_} \tt{(} [\it{return-type} \tt{,}]\s\tt{(} \it{interface-name}
\tt{,} \it{member-name} \tt{)} [\tt{,} \it{parameter-tuples}]\s\tt{)}

\idx{interface_} \tt{(} \it{interface-name} \tt{,} \it{members-list} \tt{)}

\head{Constraints}

\it{base-interface} shall be the name of another interface that
has been declared prior to the declaration of \it{class-name}.

\it{members-list} shall be a comma-separated list of names, and for each
\it{member-name} in \it{member-list}, a \tt{prototype_} declaration with the
identifier tuple \tt{(}\it{interface-name}\tt{,} \it{member-name}\tt{)}
shall be visible prior to the use of \tt{interface_}.

\it{members-list} shall be non-empty and the number of elements
after macro expansions shall be less than \tt{PP_MAX}.

The optional \it{override-list} shall be a comma-separated
list of method names from the \tt{Type} structure.

\head{Semantics}

Declaring an interface involves the following four
steps that are required to be done in the given order:

\enlargethispage*{\baselineskip}
\enlargethispage*{\baselineskip}

\begin{enumerate}

\item An object-like macro named as \it{interface-name}\tt{_EXTENDS} is
defined first, which establishes an inheritance relationship with another
interface named as \it{base-interface}, and if \it{interface-name} overrides
any of the \tt{Type} procedures inherited from \it{base-interface},
then those are required to be specified in \it{override-list}.

\item The next step is to declare the interface name with
\tt{Interface_}, which itself declares types and synonyms:

\begin{itemize}

\item It creates a forward declaration of
\tt{struct Typex (}\it{interface-name}\tt{)}, along with a pair of type synonyms:
\tt{Typex (}\it{interface-name}\tt{)} is a synonym for a non-modifiable
pointer to \tt{const struct Typex (}\it{interface-name}\tt{)}, whereas
\tt{Typex_(}\it{interface-name}\tt{)} is its modifiable twin
(the dereferenced object is non-modifiable).

\item It defines the type \tt{struct} \it{interface-name} along with a
pair of synonyms: \it{interface-name}\_ is declared a synonym for \tt{struct}
\it{interface-name}, and \tt{interface-name} is its non-modifiable counterpart.
The structure definition contains precisely the same member names as
\tt{struct Abstract}, with the following members having different types:
\tt{base} is declared as a singleton array whose element type is
\it{base-interface}\_ (which is \tt{struct} \it{base-interface}), and the member
\tt{typex} of \tt{_concrete} is of type \tt{Typex_(}\it{interface-name}\tt{)}.

\end{itemize}

\item The third step is to declare prototypes for the methods associated with
the \it{interface-name} being declared; the purpose of declaring the type
names earlier is to be able to use them as parameters and return types.

\item After declaring the prototypes, \tt{interface_} defines
the members of \tt{struct Typex (}\it{interface-name}\tt{)},
whose forward declaration was done by \tt{Interface_}.
Similar to a \tt{class_} declaration, \tt{interface_} also declares
procedures for the \tt{Type} methods prefixed with \it{interface-name},
along with a non-modifiable external array whose name is given by
\tt{type_(}\it{interface-name}\tt{)}: the array is
a singleton whose element is a \tt{Type} structure.

\end{enumerate}

\example As with classes, interfaces are typically declared in header files that
are included in other headers and translation units that require the interface.
As our first example in this chapter,
we shall declare an interface named \tt{Iterable}.
The following interface declaration is available in the header
\src{include/interface/Iterable._}

The macro \tt{Iterable_EXTENDS} sets \tt{Abstract} as the base
interface of \tt{Iterable}, and specifies the procedures that are
overridden by \tt{Iterable}, namely \tt{validate}, \tt{compare},
\tt{copy}, and \tt{add}; the rest are inherited from the \tt{Abstract} type.
\tt{Interface_(Iterable)} declares structures and
type synonyms for use in subsequent prototypes.

\enlargethispage*{\baselineskip}
\enlargethispage*{\baselineskip}

Every instance of \tt{Iterable} is an abstract data structure that
acts as a container for data objects; as their concrete representation
is not specified, we need a mechanism to iterate over the elements.
\tt{Iterator} is a synonym for an opaque structure that contains the
necessary information for iterating over an instance of \tt{Iterable};
the exact details depend on how the data is actually stored,
which is decided by the concretizing class.

The \tt{iterator} method is used to obtain an \tt{Iterator}
that is initialized to traverse through the instance
of \tt{Iterable} referred by the pointer \tt{this}.
Each \tt{iterator} maintains an internal state about the current position in
the \tt{Iterable}: \tt{has_next} checks whether the \tt{iterator} is exhausted,
and if not, then \tt{get_next} can be used to fetch the next element
(and also update the internal state of \tt{iterator}).
As \tt{Iterator} is an opaque structure, the concretizing class needs
to implement the required procedures for operating with iterators;
the necessary function pointers for these procedures
are supplied via the extended type \tt{typex}.
The \tt{duplicate} method is used clone the internal
state of an \tt{Iterator} into another object,
so that iterating through one of them does not affect the other.

As their names suggest, \tt{append} is used to add a new element
to an \tt{Iterable} instance (though not necessarily at the end),
and \tt{count} is used to determine the number of elements
currently present in an instance of \tt{Iterable}.

\note \tt{Iterable} is not required to be a sequence,
so a concretizing class need not preserve the insertion order.
