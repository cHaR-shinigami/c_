One of the blessings of inheritance in object-oriented
programming is that it facilitates code reuse.
We have already seen how the pre-conditions and
post-conditions established by interface protocols can be
utilized by concretizing classes without re-writing them.
This workflow is illustrated below, where each arrow denotes a function call.

\begin{center}
\fbox{Class protocol}$\longrightarrow$%
\fbox{Interface protocol}$\longrightarrow$%
\fbox{Interface procedure}$\longrightarrow$%
\fbox{Class procedure}
\end{center}

The above diagram leads to another interpretation of interfaces:
an interface can act as an additional layer of abstraction
between a class protocol and its corresponding procedure.
If a class method provides the concrete implementation of an interface method,
then the class protocol can indirectly invoke its procedure via an interface,
which validates pre-conditions on arguments and verifies
post-conditions on return value (and possible side effects).

So far we have seen that each interface requires some concretizing class,
which can be summarily stated as ``abstract depends on concrete'':
this is the conventional dependency.
To further leverage the benefits of interfaces, we can also write class procedures
that depend on an interface procedure, which inverts the dependency relationship.

For example, we have seen in \S 8.4.2 how the \tt{compare}
procedure of \tt{Iterable} can be implemented by comparing the
count of elements in each iterable, and if found to be equal,
then checking if both of them contain precisely the same set of elements.
Iteration and appending require knowledge of concrete representation,
and must be facilitated by the class; once these are available, other
operations such as \tt{compare} and \tt{copy} can be implemented using them.

This approach of comparing two iterables is abstract in nature,
as it does not make any assumptions about how the data is actually stored.
Abstract designs are highly reusable: each concretizing class
need not implement its own custom logic for comparing instances,
but can simply use the generic approach implemented by the interface.
This is how the \tt{compare} procedure of \tt{Chain} class
is implemented in \src{compile/class/Chain/compare.c_}

The high-level workflow shown below is quite
opposite to the conventional one we saw earlier.

\begin{center}
\fbox{Class procedure}$\longrightarrow$\fbox{Interface procedure}
\end{center}

We refer to this as ``dependency inversion'', since the concretizing
class depends on the interface for providing certain functionalities.
The coordination and interdependence between an interface and its concretizing
class is an important aspect in object-oriented design, and the technique
of dependency inversion is also used to implement several other procedures
associated with \tt{Chain}, \tt{List}, and \tt{Vector} classes, for which the
interface itself implements abstract algorithms not tied to any particular
representation, by using the primitives of iteration and appending.

\note A basic tenet of abstract design is that instead of operating directly
on data, we implement representation-agnostic higher-order functions that work
with other lower-level functions which are ``closer'' to the concrete data.
