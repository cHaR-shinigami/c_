The names \tt{concrete} and \tt{concrete_} pointers to
\tt{Void} and \tt{Void_} (respectively), whereas \tt{_concrete}
has a different type, which is pointer to an untagged union.
The ISO C standard does not disallow union pointers to have a different
representation than \tt{void} pointer; as a purely artificial example,
consider a hypothetical execution environment where pointers to union
are represented with flipped bits, and the actual memory address can
be obtained by toggling each bit, which is equivalent to taking one's
complement of the pointer data interpreted as a binary integer.
Therefore, at least in theory, a \tt{void} pointer representation need not
be identical to a union pointer, and interpreting one pointer type's binary
representation as another pointer type would cause undefined behavior.

In practice, virtually all existing environments have identical representations
for all object pointers; in fact, the POSIX standard has a stronger requirement
that function pointers must also be representable as \tt{void} pointers.
This implies that if pointer to a concrete instance is stored using the
names \tt{concrete} or \tt{concrete_}, the stored representation is
that of a \tt{void} pointer, and it can be safely interpreted as union
pointer due to their identical representation (on real-world systems).
However, out of purely pedantic concerns, it is recommended that
the name \tt{_concrete} should not be directly accessed, and the
name \tt{concrete} should be used for storing a concrete instance.

A non-lvalue expression having the same type as \tt{_concrete}
can be obtained with the macro \tt{concrete_}.

\head{Syntax}

\tt{concrete_ (} $expression$ \tt{)}

\head{Constraints}

Type of $expression$ shall be pointer to \tt{Abstract}, or
pointer to an interface type that inherits from \tt{Abstract}.
Equivalently, $expression$ shall be of type ``pointer to $T$'' or
``pointer to $T$\_'', and \tt{is_(}$T$\tt{, Abstract)} shall be \tt{true}.

\note \tt{is(Abstract, Abstract)} is trivially satisfied
due to the reflexive nature of ``is-a'' relationship.

\head{Semantics}

\tt{concrete_} dereferences $expression$ to read the member
name \tt{concrete_} and casts it to the type of \tt{_concrete}.

\tt {concrete_(}$expression$\tt{)} is equivalent to
\tt{((typeof ((}$expression$\tt{)->_concrete)) (}$expression$\tt{)->concrete_)}.

\note The macro \tt{concrete_} uses the name \tt{_concrete}
only to get its data type, not for reading the pointer.
Nevertheless, the cast is a no-op for practically all systems,
and the outcome is same as \tt{(}$expression$\tt{)->_concrete}.
