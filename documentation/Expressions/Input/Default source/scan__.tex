\head{Syntax}

\tt{# include <input._>}

\tt{scan__ (} \it{expression-list} \tt{)}

\head{Constraints}

\it{expression-list} shall be a comma-separated list of expressions,
and each expression shall be an unqualified lvalue for which
a pointer can be obtained with address-of operator \tt{&}.
Each lvalue shall have one of the following types:\\

\noindent
\begin{tabular}{@{}rrrrrrrr}

\tt{UByte_} &\tt{UShort_} &  \tt{UInt_} & \tt{ULLong_} &\tt{UInt_least8_} &\tt{UInt_least16_} &\tt{UInt_least32_} &\tt{UInt_least64_}\\
 \tt{Byte_} & \tt{Short_} &   \tt{Int_} &  \tt{LLong_} & \tt{Int_least8_} & \tt{Int_least16_} & \tt{Int_least32_} & \tt{Int_least64_}\\

\tt{UInt8_} &\tt{UInt16_} &\tt{UInt32_} & \tt{UInt64_} & \tt{UInt_fast8_} & \tt{UInt_fast16_} & \tt{UInt_fast32_} & \tt{UInt_fast64_}\\
 \tt{Int8_} & \tt{Int16_} & \tt{Int32_} &  \tt{Int64_} &  \tt{Int_fast8_} &  \tt{Int_fast16_} &  \tt{Int_fast32_} &  \tt{Int_fast64_}\\

            &\tt{Dec128_} & \tt{Dec32_} &  \tt{Dec64_} &       \tt{Size_} &     \tt{UIntptr_} &       \tt{ULong_} &     \tt{UIntmax_}\\
 \tt{Char_} & \tt{Float_} &\tt{Double_} &\tt{LDouble_} &    \tt{Ptrdiff_} &      \tt{Intptr_} &        \tt{Long_} &      \tt{Intmax_}\\

&& \tt{Void *} & \tt{Void_ *} & \tt{Char_ [}$n$\tt{]} & \tt{WChar_ [}$n$\tt{]} &&

\end{tabular}\\

\note \tt{Char_ [}$n$\tt{]} and \tt{WChar_ [}$n$\tt{]}
denote complete array types having $n$ elements.
Plain pointers to \tt{Char_} or \tt{WChar_} are not supported for storing text
as string or wide string because length cannot be inferred from the type,
and a sufficiently long input text can get written past the buffer end,
thereby corrupting adjacent memory.

Types named as \tt{UInt}$n$\_ and \tt{Int}$n$\_
are synonyms for exact-width types;
types named as \tt{UInt_least}$n$\_ and \tt{Int_least}$n$\_
are synonyms for minimum-width types;
types named as \tt{UInt_fast}$n$\_ and \tt{Int_fast}$n$\_
are synonyms for fastest minimum-width types.
These synonyms are provided by the header \tt{<stdint._>},
discussed in chapter 9.

Most implementations of C define these types as synonyms for the basic types;
however, the standard allows them to be extended integer types,
which can be separately provided by compilers in addition
to and incompatible with the basic integer types.
Moreover, the exact-width types along with \tt{UIntptr_} and \tt{Intptr_}
are optional; these and the conditionally supported decimal floating-point
types can be used with \tt{scan__} subject to their availability.

Variables defined with \tt{register}, \tt{let}, \tt{Var}, or \tt{Var_}
are unsuitable for \tt{scan__}, as their address cannot be taken.

\head{Semantics}

\tt{scan__} reads from \tt{stdin} and interprets the text
as a value in the range of an lvalue in \it{expression-list}.
Lvalues are assigned from left to right, and all expressions
are evaluated in an unspecified order before reading starts.
Input is consumed as long as the unconverted text read so far can
be interpreted as a valid value as per the specifiers listed below.
The first non-matching character acts as a delimiter and the unconverted input
read till then is represented as a value that is stored in the first lvalue.
If there are subsequent lvalues in \it{expression-list},
then input text is interpreted and represented as per their types.
For each lvalue, any whitespace character (as identified by \tt{isspace}
function) found before the first matching character is consumed and discarded$.$
An assignment is unsuccessful for an lvalue if a valid value
cannot be constructed from the matching characters found (if any):
this can happen if a delimiter is found too early, or if the input
ends due to end of file (\tt{EOF}), or due to an error or interruption.
When an lvalue cannot be assigned, reading stops there and subsequent lvalues
(if any) are also left unassigned; if the interpretation fails due to the early
occurrence of a non-matching character, then that character is not consumed.

For each lvalue in \it{expression-list}, input text is interpreted as if
\tt{scanf} is called with a pointer to that lvalue as the second argument,
the first argument being a format string with a percent symbol \tt{"\%"}
followed by one of the following format specifiers as per the non-array lvalue
type; for the array types, \tt{\%} occurs after an opening backtick (\tt{`}).\\

\noindent
\begin{tabular*}{\textwidth}{@{\extracolsep{\fill}}rl|rl|rl}

       \tt{UByte_} & \tt{"hhu"}       &        \tt{Byte_} & \tt{"hhi"}  & \tt{Char_}                  & \tt{" c"}\\

      \tt{UShort_} & \tt{"hu"}        &       \tt{Short_} & \tt{"hi"}   & \tt{Float_}                 & \tt{"g"}\\

        \tt{UInt_} & \tt{"u"}         &         \tt{Int_} & \tt{"i"}    & \tt{Double_}                & \tt{"lg"}\\

       \tt{ULong_} & \tt{"lu"}        &        \tt{Long_} & \tt{"li"}   & \tt{LDouble_}               & \tt{"Lg"}\\

      \tt{ULLong_} & \tt{"llu"}       &       \tt{LLong_} & \tt{"lli"}  & \tt{Dec32_}                 & \tt{"Hg"}\\

        \tt{Size_} & \tt{"zu"}        &     \tt{Ptrdiff_} & \tt{"ti"}   & \tt{Dec64_}                 & \tt{"Dg"}\\

     \tt{UIntmax_} & \tt{"ju"}        &      \tt{Intmax_} & \tt{"ji"}   & \tt{Dec128_}                & \tt{"DDg"}\\

       \tt{UInt8_} & \tt{SCNu8}       &        \tt{Int8_} & \tt{SCNi8}  & \tt{Void}\s\s\tt{*}         & \tt{"p"}\\

      \tt{UInt16_} & \tt{SCNu16}      &       \tt{Int16_} & \tt{SCNi16} & \tt{Void_ *}                & \tt{"p"}\\

      \tt{UInt32_} & \tt{SCNu32}      &       \tt{Int32_} & \tt{SCNi32} & \tt{\ Char_ [}$n + 1$\tt{]} & \tt{"`\%}$n$\tt{[^`]`"}\\

      \tt{UInt64_} & \tt{SCNu64}      &       \tt{Int64_} & \tt{SCNi64} & \tt {WChar_ [}$n + 1$\tt{]} & \tt{"`\%}$n$\tt{l[^`]`"}\\

 \tt{UInt8_least_} & \tt{SCNuLEAST8}  &  \tt{Int8_least_} & \tt{SCNiLEAST8}\\

\tt{UInt16_least_} & \tt{SCNuLEAST16} & \tt{Int16_least_} & \tt{SCNiLEAST16}\\

\tt{UInt32_least_} & \tt{SCNuLEAST32} & \tt{Int32_least_} & \tt{SCNiLEAST32}\\

\tt{UInt64_least_} & \tt{SCNuLEAST64} & \tt{Int64_least_} & \tt{SCNiLEAST64}\\

  \tt{UInt8_fast_} & \tt{SCNuFAST8}   &   \tt{Int8_fast_} & \tt{SCNiFAST8}\\

 \tt{UInt16_fast_} & \tt{SCNuFAST16}  &  \tt{Int16_fast_} & \tt{SCNiFAST16}\\

 \tt{UInt32_fast_} & \tt{SCNuFAST32}  &  \tt{Int32_fast_} & \tt{SCNiFAST32}\\

 \tt{UInt64_fast_} & \tt{SCNuFAST64}  &  \tt{Int64_fast_} & \tt{SCNiFAST64}\\

     \tt{UIntptr_} & \tt{SCNuPTR}     &      \tt{Intptr_} & \tt{SCNiPTR}\\

\end{tabular*}\\

If an lvalue is of type \tt{Char_ [}$n$\tt{]} or \tt{WChar_ [}$n$\tt{]},
its matching input text is enclosed within backticks,
optionally preceded by whitespace.
For that lvalue, at most $n - 1$ non-backtick characters after the
opening backtick are stored in the array, and a null character is
appended (wide null character is used for \tt{WChar_} arrays).
If a closing backtick is found among the first $n$ characters
after the opening backtick, it is consumed but not stored;
otherwise the $n^{th}$ character (if any) is not consumed, and matching
failure occurs for any subsequent lvalue in the same \it{expression-list}.

An lvalue can be evaluated more than once only if it is a variable length array,
namely \tt{Char_ [}$n$\tt{]} or \tt{WChar_ [}$n$\tt{]}
with $n$ not being an integer constant expression.
Outcome of \tt{scan__} is of type \tt{Int_}: a positive outcome
indicates number of lvalues assigned in sequence; otherwise
none were assigned, and a negative outcome indicates read error.

If \tt{INPUT__} is \tt{1}, pointers to lvalues are asserted to be not null;
otherwise \tt{INPUT__} shall be \tt{0}, and if pointer to a non-array
lvalue is null, \tt{scan__} proceeds as if it was not null:
matching text is consumed and counted in outcome.
