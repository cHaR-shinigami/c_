\head{Syntax}

\tt{# include <input._>}

\idx{input__}\s\s\s\s\tt{(} [\it{source}\tt{=stdin} \tt{,}
[\it{separator}\tt{=""} \tt{,}]] \it{expression-list} \tt{)}

\idx{input__0__}\s\tt{(} \it{source} \tt{,}
\it{separator} \tt{,} \it{expression-list} \tt{)}

\idx{input__1__}\s\tt{(} \it{expression} \tt{)}

\idx{input__2__}\s\tt{(} \it{source} \tt{,} \it{expression} \tt{)}

\idx{input__3__}\s\tt{(} \it{source} \tt{,} \it{separator} \tt{,} \it{expression} \tt{)}

\head{Constraints}

\it{source} shall be a stream (\tt{File_ *}), a string, or a wide string.
\it{separator} shall be a string or a wide string:
if \it{separator} is a string, then \it{source} shall not be a wide string;
otherwise \it{separator} is a wide string, and \it{source} shall not be a string.
\it{expression} shall be a single lvalue expression and \it{expression-list}
shall be a comma-separated list of lvalue expressions, each of which shall
have precisely the same constraints as those applicable for \tt{scan__}.

\head{Semantics}

\tt{input__} invokes \tt{input__0__} if the expanded argument sequence contains
more than three arguments; otherwise it invokes \tt{input__}$n$\tt{__} if the
expanded argument sequence contains $n$ arguments, with $n$ not exceeding three.
\tt{input__1__} is equivalent to \tt{scan__} with a single expression.
\tt{input__2__} reads from \it{source}, terminated by a null byte
or wide null character for string and wide string respectively;
rest of the semantics are similar to \tt{input__1__},
and if \it{source} is an input stream, it is assumed to be byte-oriented.
\tt{input__3__} is similar to \tt{input__2__} with some constraints between
\it{source} and \it{separator}: if these are not violated and \it{source}
is an input stream, then \it{separator} decides how the text is processed:
if \it{separator} is a string, then \it{source} is assumed to be byte-oriented;
otherwise \it{separator} is a wide string and
\it{source} is assumed to be wide-oriented.
\tt{input__0__} is a customizable variation of \tt{scan__}:
the first argument specifies a \it{source}, and the second argument
specifies a pattern to be matched between every two consecutive inputs.
\it{separator} can be any pattern that is supported by \tt{scanf}, but it should
not contain any format specifier: more precisely, any percent symbol occurring
in \it{separator} should be immediately followed by another percent symbol, so
that each pairing of \tt{\%\%} matches a literal percent symbol in the input.
If \it{separator} contains a percent symbol not paired with another percent
symbol just after that, then \it{separator} is dereferenced but not utilized, and
an empty string \tt{""} or wide string \tt{L""} is used as the separator instead.
The behavior is undefined if \it{source} is a stream that is not an input stream,
or its orientation is different from what is indicated by \it{separator}.

\it{source} and \it{separator} are evaluated exactly once, and an lvalue
in \it{expression} or \it{expression-list} can be evaluated more than once
only if it is a variable length array, namely \tt{Char_ [}$n$\tt{]} or
\tt{WChar_ [}$n$\tt{]} with $n$ not being an integer constant expression.
Outcome of the \tt{input__} family is of type \tt{Int_}: a positive
outcome indicates the number of lvalues assigned in sequence; otherwise
none of them were assigned, and a negative outcome indicates read error.

If \tt{INPUT__} expands to \tt{1}, then \it{source}, \it{separator} for
\tt{input__0__}, and each pointer to an lvalue in \it{expression} or
\it{expression-list} is asserted to be not null; otherwise \tt{INPUT__}
shall be \tt{0}, and null pointers are handled as follows:

\begin{itemize}[nosep]

\item If \it{source} is null, then the outcome is equal to \tt{EOF},
and none of the lvalues are assigned.

\item If \it{separator} for \tt{input__0__} is null, it is evaluated
and replaced by a pattern with one space (\tt{" "} or \tt{L" "}).

\item If pointer to a non-array lvalue in \it{expression} or \it{expression-list}
is null, the \tt{input__} family proceeds as if that pointer was not null: a
matching text is consumed and counted in outcome, but the interpreted value is lost.

\item If pointer to an array lvalue is null,
then attempting to store a matching text causes undefined behavior.

\end{itemize}

\note For \tt{input__3__}, only the data type of \it{separator} is used
to indicate the input stream orientation, and the value of \it{separator}
is not required at all; however, it is still evaluated and the result
is discarded without dereferencing, so \it{separator} can also be a null
pointer for \tt{input__3__} (the type is relevant, the value is not).
