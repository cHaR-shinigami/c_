\head{Syntax}

\idx{has_qualifier_}\s\s\s\tt{(} [\it{qualifier} \tt{,}] \it{expression} \tt{)}

 \tt{has_qualifier_}\s\s\s\tt{(} [\it{qualifier} \tt{,}] \it{type-name}  \tt{)}

\idx{has_qualifier_1_}\s\tt{(} \it{expression} \tt{)}

 \tt{has_qualifier_1_}\s\tt{(} \it{type-name}  \tt{)}

\idx{has_qualifier_2_}\s\tt{(}
\phantom{[}\it{qualifier} \tt{,}\phantom{]} \it{expression} \tt{)}

 \tt{has_qualifier_2_}\s\tt{(}
\phantom{[}\it{qualifier} \tt{,}\phantom{]} \it{type-name}  \tt{)}

\head{Constraints}

\it{expression} shall have object type and shall not designate a bit-field.
\it{type-name}  shall not specify a function type.

\head{Semantics}

\tt{has_qualifier_} invokes \tt{has_qualifier_}$n$\_ if
the expanded argument sequence contains $n$ arguments.
The outcome of \tt{has_qualifier_1_} is \tt{FALSE} if \it{expression}
or \it{type-name} is unqualified; otherwise the outcome is \tt{TRUE}.

\tt{has_qualifier_2_} checks if \it{expression} or \it{type-name} is qualified
with \it{qualifier}, which can specify multiple qualifiers separated by
whitespace: the outcome is \tt{TRUE} if \it{qualifier} is detected in the type,
and \tt{FALSE} otherwise.

\example \tt{has_qualifier_(Char_)} is \tt{FALSE}.
\tt{has_qualifier_(const volatile, volatile Char)} is \tt{TRUE}.

\pagebreak
