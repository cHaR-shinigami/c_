\def\Subsection#1{\subsection{#1}\input{Expressions/Generic selections/#1}}

Generic selections use the converted type of an expression to
select another expression; the former expression is not evaluated.
They are analogous to \tt{switch}, except that \tt{switch} selections are done at
runtime using value of the controlling expression, whereas generic selections are
performed during translation, based on the type of the controlling expression.
Generic selections were standardized in C11, and they have the following syntax:

\tt{_Generic (} \it{controlling-expression}
[\tt{,} \it{type-name} \tt{:} \it{expression}] $\cdots$
[\tt{, default :} \it{expression}]
\tt{)}

A generic selection consists of a controlling expression
followed by a comma-separated list of generic associations:

\begin{tabular}{r@{\hskip 0pt}l}

\it{type-name} & \s\tt{:} \it{expression}\\

\tt{default} & \s\tt{:} \it{expression}\\

\end{tabular}

A \tt{_Generic} selection requires at least one generic association having
one of the above forms; at most one \tt{default} association is permitted.
As comma is used to separate different parts of a generic selection,
none of the expressions can be an unparenthesized comma expression.
Type for the value of \it{controlling-expression} is determined as if with
\tt{typeof_(}\it{controlling-expression}\tt{)}, which is itself equivalent to
\tt{typeof (value_(}\it{controlling-expression}\tt{))}.

If the resulting type is compatible with \it{type-name} of a generic association,
then outcome of the generic selection is the
\it{expression} for that matching association.
If the optional \tt{default} association is present, then its \it{expression}
is selected iff no other generic association matches the resulting type
of \it{controlling-expression}; it need not be the last in sequence.
Exactly one generic association must match, and type of a \tt{_Generic}
expression is same as type of its selected expression, as if the entire generic
selection is replaced by an implicitly parenthesized form of that expression.
None of the other expressions are evaluated;
however, they must all be semantically valid C expressions.

\note Recall that \tt{typeof_} and \tt{value_} perform lvalue conversion,
array to pointer decay, and function to function pointer conversion;
if \it{type-name} of an association is qualified,
then its \it{expression} will never be selected.

\example \tt{_Generic (*"", Char : "error")} causes a compilation error as the
controlling expression \tt{*""} undergoes lvalue conversion, and its resulting
type \tt{Char_} does not match the \it{type-name} of any generic association.

\Subsection{Generalized generic}

\Subsection{Qualifier sensitivity}
