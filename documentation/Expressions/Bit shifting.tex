\def\Subsection#1{\subsection{#1}\input{Expressions/Bit shifting/#1}}

Certain semantics of shift operators are implementation-defined or undefined,
particularly with negative integers.
C\_ fills these gaps with bit shifting features that have
well-defined behavior for all integers when compiled in production mode;
however, operands that cause undefined behavior for the built-in shift operators
are caught at runtime when compiled in debugging mode, as if with assertions.
C\_ broadens the set of operands for which shifting by
$s$ bits is same as multiplication or division by $2^s$,
even with one's complement and sign-magnitude representations.

The following subsections describe features that have a more
uniform semantics for bitwise shift operations on negative integers;
these features are representation agnostic and their behavior is defined
in terms of value of the outcome for both unsigned and signed integers.
For the different representations of signed integers,
there can be two distinct notions of uniformity in outcome
when a bitwise operation is performed on a negative value:

\begin{itemize}[nosep]

\item Outcomes are specified in terms of bit patterns,
but their corresponding values depend on the representation.

\item Outcomes are specified in terms of values,
but their corresponding bit patterns depend on the representation.

\end{itemize}

C\_ prioritizes the latter kind of uniformity for signed types,
as bit pattern manipulations are mostly done with unsigned types.
For each of these features, the left expression is converted
to the widest integer type that is not a bit-precise type:
\tt{UIntmax} is used for \tt{ulsh_} and \tt{ursh_};
\tt{Intmax} is used for \tt{lsh_} and \tt{rsh_}.
The converted left expression is shifted by the number of bits given
by the right expression; the latter is converted to an unsigned type
and it should be less than width of the left expression's type.
The outcome has the same type as the left expression.

The bit shifting features are grouped into the
headers \tt{<lshift._>} and \tt{<rshift._>};
their behavior can be configured during compilation with the \tt{DEBUG} macro.
These two headers are aggregated by \tt{<shift._>};
the purpose of including \tt{<shift._>} in a source file is to ensure
a common configuration for both \tt{<lshift._>} and \tt{<rshift._>}.

\note Both \tt{UIntmax} and \tt{Intmax} are synonyms defined in \tt{<stdint._>};
they can be extended integer types.
With the ubiquity of two's complement representation that is also mandated
by C23, the bit shifting features described here are of limited use,
and most programmers are likely to prefer the basic
shift operators for simple bit pattern manipulations.
Chapter 5 presents dynamically resizable bit
arrays that serve a wider variety of purposes.

\Subsection{Left shift}

\Subsection{Right shift}
