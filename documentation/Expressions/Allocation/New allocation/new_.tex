\head{Syntax}

\idX{new_}\s\s\s\tt{(} \it{expression}
[\tt{,} \it{array-length}
[\tt{,} \it{initializer}]] \tt{)}

 \tt{new_}\s\s\s\tt{(} \it{type-name}
[\tt{,} \it{array-length}
[\tt{,} \it{initializer}]] \tt{)}

\idX{new_1_}\s\tt{(} \it{expression} \tt{)}

 \tt{new_1_}\s\tt{(} \it{type-name}  \tt{)}

\idX{new_2_}\s\tt{(} \it{expression}
\l\tt{,} \it{array-length}\r\  \tt{)}

 \tt{new_2_}\s\tt{(} \it{type-name}
\l\tt{,} \it{array-length}\r\  \tt{)}

\idX{new_3_}\s\tt{(} \it{expression}
\l\tt{,} \it{array-length}
\l\tt{,} \it{initializer}\r\r\ \tt{)}

 \tt{new_3_}\s\tt{(} \it{type-name}
\l\tt{,} \it{array-length}
\l\tt{,} \it{initializer}\r\r\ \tt{)}

\head{Constraints}

The \tt{new_} family shall have precisely the same
constraints as those applicable for the \tt{new__} family.

\head{Semantics}

\tt{new_} family evaluates each expression and \it{type-name} only once;
rest of the semantics are same as \tt{new__} family.
