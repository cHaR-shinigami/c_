\head{Syntax}

\idx{new__}\s\s\s\s\tt{(} \it{expression}
[\tt{,} \it{array-length}
[\tt{,} \it{initializer}]] \tt{)}

 \tt{new__}\s\s\s\s\tt{(} \it{type-name}
[\tt{,} \it{array-length}
[\tt{,} \it{initializer}]] \tt{)}

\idx{new__1__}\s\tt{(} \it{expression} \tt{)}

 \tt{new__1__}\s\tt{(} \it{type-name}  \tt{)}

\idx{new__2__}\s\tt{(} \it{expression}
\l\tt{,} \it{array-length}\r\  \tt{)}

 \tt{new__2__}\s\tt{(} \it{type-name}
\l\tt{,} \it{array-length}\r\  \tt{)}

\idx{new__3__}\s\tt{(} \it{expression}
\l\tt{,} \it{array-length}
\l\tt{,} \it{initializer}\r\r\ \tt{)}

 \tt{new__3__}\s\tt{(} \it{type-name}
\l\tt{,} \it{array-length}
\l\tt{,} \it{initializer}\r\r\ \tt{)}

\head{Constraints}

\it{expression} shall have a complete object type
and shall not designate a structure bit-field member.
\it{type-name}  shall specify a complete object type.
\it{array-length} shall have an integer type.
It shall be possible to use the value of \it{initializer} to
initialize a variable declared with the same type as \it{expression},
or the type specified by \it{type-name}.

\head{Semantics}

\tt{new__} invokes \tt{new__}$n$\_\_ if the
expanded argument sequence contains $n$ arguments.
In all cases, \it{expression} or \it{type-name} may not be evaluated
at all if they do not specify a variably modified type, or they can
be evaluated more than once only if their type is variably modified.
\it{array-length} can be evaluated multiple times
only if it is not an integer constant expression.
\it{initializer} is always evaluated only once.
The order of evaluation is unspecified.

A non-null outcome of \tt{new__1__} is pointer to an object
that has the unqualified type of \it{expression} or \it{type-name}.

\tt{new__2__} converts \it{array-length} to type \tt{Size}
and a non-null outcome is pointer to an array whose elements
have the unqualified type of \it{expression} or \it{type-name};
length of the array is equal to the converted value of \it{array-length}.

\tt{new__3__} allocates an array in the same way as \tt{new__2__},
and on success, \tt{new__3__} also initializes each element of
the array with the resulting value of \it{initializer} after it
is converted to the type of \it{expression} or \it{type-name}.

\note \tt{Size_} is a synonym for \tt{size_t},
which is defined as the type of a \tt{sizeof} expression.
For \tt{new__2__} and \tt{new__3__},
the outcome is of type ``pointer to array'' which encodes length information;
a similarly typed outcome is also given by \tt{new__1__} if its \it{expression}
or \it{type-name} specifies an array type, such as \tt{new__(Char_ [BUFSIZ])}.

\example \tt{new__(Char_ *, 10, 0)} is a portable approach to populate
a dynamically allocated array of ten elements with null pointers.
Note that this may not be equivalent to simply zeroing out the memory
with \tt{calloc} or \tt{memset}, as there are execution environments
(mostly archaic) where null pointer representation has non-zero bits.
