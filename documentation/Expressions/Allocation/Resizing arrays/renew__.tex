\head{Syntax}

\idx{renew__}\s\s\s\s\tt{(} \it{array-pointer} \tt{,} \it{array-length}
[\tt{,} \it{initializer}] \tt{)}

\idx{renew__2__}\s\tt{(} \it{array-pointer} \tt{,} \it{array-length} \tt{)}

\idx{renew__3__}\s\tt{(} \it{array-pointer} \tt{,} \it{array-length}
\l\tt{,} \it{initializer}\r\ \tt{)}

\head{Constraints}

\it{array-pointer} shall be pointer to unqualified complete array type.
\it{array-length} shall have integer type.
It shall be possible to assign \it{initializer} to element of the
array obtained on dereferencing \it{array-pointer}, without type cast.

\head{Semantics}

\tt{renew__} invokes \tt{renew__}$n$\_\_ if the
expanded argument sequence contains $n$ arguments.
In all cases, \it{array-pointer} shall be suitable for passing to \tt{free};
otherwise the behavior is undefined.
\it{array-pointer} can be evaluated more than once
only if element type of the array is variably modified.
\it{array-length} is converted to type \tt{Size}, and it can be evaluated
multiple times only if it is not an integer constant expression.
\it{initializer} is evaluated only once.

A non-null outcome of \tt{renew__2__} is pointer to an
array having \tt{(Size)(}\it{array-length}\tt{)} elements,
with the same element type as that of \it{array-pointer}.
If \it{array-pointer} is null,
then the behavior is identical to \tt{new__2__}, which allocates a new array.
If the outcome is null,
then resizing could not be done, and the original array is preserved.

\tt{renew__3__} resizes an array in the same way as \tt{renew__2__},
and if the array is expanded, then \tt{renew__3__} also initializes
each additional element of the resized array with the resulting value
of \it{initializer} after it is converted to the array element type.
For the purpose of initialization, the current array length
is determined solely from the type of \it{array-pointer},
which may or may not reflect true length of the array object.
For instance, \tt{renew__3__} also permits \it{array-pointer} to be null, but as
per the constraints, its type shall be pointer to an unqualified complete array
type: if the length encoded by that type is smaller than the converted value of
\it{array-length}, then initialization starts from the index equal to the
inferred old length, and array elements prior to that index remain uninitialized.

\note If the element type is not variably modified, then \it{array-pointer}
is evaluated only once, even if the array type itself is variably modified.
Practically speaking, shrinking an array should always be possible in-place,
but in theory, any resizing operation can relocate the array
and produce a new pointer, leaving the old pointer dangling.
