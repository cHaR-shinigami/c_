\def\Subsection#1{\subsection{#1}\input{Expressions/Allocation/#1}}

The features listed in the following subsections are used for dynamic
memory allocation: on success, they return a typed pointer to an object
whose lifetime is not limited by the lexical scope, and extends throughout
the process until that pointer is passed to the library function \tt{free}.
If the required allocation cannot be obtained in one contiguous memory block,
then the outcome is a null pointer.
The reference implementation provides these features as
function-like macros in the header \tt{<allocation._>},
which also declares the \tt{free} function as: \tt{Void_ free(Void_ *);}

\note It is the responsibility of a programmer to release
the acquired memory when it is no longer required,
as there is no automatic ``garbage collection'' for unreachable objects.
On most hosted environments, dynamically allocated memory is
released by the operating system when the process terminates,
so calling \tt{free} just before exiting the process is often a redundant
operation that causes a marginal increase in code size and execution time.

\Subsection{New allocation}

\Subsection{Resizing arrays}

\Subsection{Conditional allocation}
