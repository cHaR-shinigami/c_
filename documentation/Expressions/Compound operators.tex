C\_ has two unusual operators whose semantics can be achieved
by the combined use of unary, equality, and logical operators;
for this reason, we refer to them as compound operators.
Both are useful for expressing logical assertions;
they are frequently used for writing pre-conditions and
post-conditions within protocols (introduced in chapter 6).

Due to the way they are provided by the reference implementation,
both operators require parentheses as part of their syntax.
If the parenthesized expression is part of a larger expression,
then it should be doubly parenthesized, as precedence and associativity are
implementation defined when there are other operators outside the parentheses.

\subsection{\idx{iff}}
\head{Syntax}

\tt{(} \it{expression} \tt{iff} \it{expression} \tt{)}

\head{Constraints}

Both operands shall be scalar expressions.

\head{Semantics}

\tt{iff} checks logical equivalence of the two scalar expressions:
the result has value one if both operands compare equal to zero,
or both operands compare unequal to zero; otherwise the result is zero.
The result is of type \tt{Int_}.

\note The reference implementation provides \tt{iff} as an
object-like macro that expands to the text \tt{)==0 == !(}


\subsection{\idx{implies}}
\head{Syntax}

\tt{(} \it{implicant} \tt{implies} \it{implicand} \tt{)}

\head{Constraints}

Both \it{implicant} and \it{implicand} shall be scalar expressions.

\head{Semantics}

\tt{implies} checks if a logical implication exists between the two operands:
the result has value one if \it{implicant} is zero or
\it{implicand} is non-zero; otherwise the result is zero.
The result is of type \tt{Int_}, and the operation follows
short-circuit evaluation: \it{implicant} is evaluated first,
and if it is non-zero, only then \it{implicand} is evaluated.

\note The reference implementation provides \tt{implies} as
an object-like macro that expands to \tt{)==0 ||}\s\s\tt{(}

