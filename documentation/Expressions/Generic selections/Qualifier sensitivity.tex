Both \tt{_Generic} and its extension \tt{generic_} ignore type
qualifiers when checking type compatibility of a controlling
expression (list) with the type (list) of an association.
C\_ has another kind of generalization called \tt{generiq_},
which is a qualifier preserving variation of  \tt{generic_}:
both have the same syntax, but different constraints and semantics.

\head{Syntax}

\idx{generiq_} \tt{( (} \it{expression-list} \tt{)}
[\tt{, (} \it{type-list} \tt{,} \it{expression} \tt{)}] $\cdots$
[\tt{, (} \it{default-expression} \tt{)}]
\tt{)}

\head{Constraints}

\tt{generiq_} accepts a sequence of parenthesized lists, where the first list is
a tuple of controlling expressions, and each subsequent list is an association.
\it{expression-list} shall be a comma-separated list of expressions,
such as $expr_1$ \tt{,} $\cdots$ \tt{,} $expr_n$.
Each \it{type-list} shall be a comma-separated list of types,
such as \it{type-name}$_1$ \tt{,} $\cdots$ \tt{,} \it{type-name}$_k$.
Number of types in a \it{type-list} need not match
the count of expressions in \it{expression-list}.

An optional association without \it{type-list} is a default association,
which shall consist of a single \it{default-expression}; at most one
default association is permitted, and it need not be the last in sequence.
All associations shall be distinct, such that any two non-default association
lists differ in at least one \it{type-name}, after considering type qualifiers.
There shall be exactly one association list
that can be selected based on semantic rules.

\head{Semantics}

A type sequence is constructed from the controlling \it{expression-list}
retaining type qualifiers, and if it is compatible element-wise with
the type list of an association, then that association is selected,
and its expression becomes the outcome of the \tt{generiq_} expression;
otherwise a default association list is present,
having a single \it{default-expression} that is selected as the outcome.
Type of the \tt{generiq_} expression is same as type of the selected expression,
as if the entire \tt{generiq_} expression is replaced by an
implicitly parenthesized form of the selected expression.

Other than the expression from the selected association list,
none of the other expressions are evaluated.

\example Our \tt{cc_} alias for \tt{gcc} and \tt{clang} includes the option
\tt{-Wwrite-strings} which enforces the type of string literals to be
``array of \tt{Char}'', consistent with their non-modifiable nature;
we can use \tt{generiq_} to test this.

\centerline{\tt{generiq_((*""), (Char, FALSE), (Char_, TRUE))}}

The outcome would be \tt{FALSE} when compiled with \tt{cc_},
and \tt{TRUE} when compiled with \tt{cc_ -Wno-write-strings}.

\note Recall that \tt{FALSE} has type \tt{False_} and \tt{TRUE}
has type \tt{True_}, so type of the result indicates its value.
