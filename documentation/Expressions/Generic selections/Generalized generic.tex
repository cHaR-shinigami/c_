The basic \tt{_Generic} selection only allows a single controlling expression;
C\_ offers an extension that generalizes it to a tuple of controlling expressions,
and correspondingly, each generic association also permits a tuple of types.

\head{Syntax}

\idx{generic_} \tt{( (} \it{expression-list} \tt{)}
[\tt{, (} \it{type-list} \tt{,} \it{expression} \tt{)}] $\cdots$
[\tt{, (} \it{default-expression} \tt{)}]
\tt{)}

\head{Constraints}

\tt{generic_} accepts a sequence of parenthesized lists, where the first list is
a tuple of controlling expressions, and each subsequent list is an association.
\it{expression-list} shall be a comma-separated list of expressions,
such as $expr_1$ \tt{,} $\cdots$ \tt{,} $expr_n$.
Each \it{type-list} shall be a comma-separated list of types,
such as \it{type-name}$_1$ \tt{,} $\cdots$ \tt{,} \it{type-name}$_k$, and a
\it{type-name} shall not specify an incomplete aggregate type; a \it{type-name}
can be \tt{Void_} only if there are no other types in an association list.
Number of types in a \it{type-list} need not match
the count of expressions in \it{expression-list}.

An optional association without \it{type-list} is a default association,
which shall consist of a single \it{default-expression}; at most one
default association is permitted, and it need not be the last in sequence.
All associations shall be distinct, such that any two non-default
association lists differ in at least one \it{type-name},
after considering type adjustments described by the semantics.
There shall be exactly one association list
that can be selected based on semantic rules.

\head{Semantics}

A type sequence is constructed from the controlling \it{expression-list},
as if with \tt{typeof_}; more precisely, each expression in the first list
is subjected to lvalue conversion, array to pointer decay, and function
to function pointer conversion; then its resulting type is determined.
For each \it{type-name} in a non-default association list,
outermost type qualifiers are discarded, array types are adjusted to pointer
types, and function types are adjusted to function pointer types; the latter two
adjustments are identical to those which are performed for function parameters.

After performing these conversions, if the type list constructed from the
controlling \it{expression-list} is compatible element-wise with the type list of
an association, then that association is selected, and its expression becomes the
outcome of the \tt{generic_} expression; otherwise a default association list is
present, having a single \it{default-expression} that is selected as the outcome.
Type of the \tt{generic_} expression is same as type of the selected expression,
as if the entire \tt{generic_} expression is replaced by an
implicitly parenthesized form of the selected expression.

Other than the expression from the selected association list,
none of the other expressions are evaluated.

\example \tt{generic_} can be used to emulate ``function overloading'',
where a unified function call interface is
designed for a group of possibly related functions.
It is a form of static polymorphism because binding a call
to a specific function is decided during translation itself,
depending on the type of arguments given by that invocation.

\code{../compile/generic.c_}

\note C23 allows variable argument functions to be declared
without a named parameter before ellipsis (\tt{...}).
