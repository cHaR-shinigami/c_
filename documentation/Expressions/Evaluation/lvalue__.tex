\head{Syntax}

\tt{lvalue__ (} \it{expression} \tt{)}

\head{Semantics}

The result is an unqualified lvalue initialized to the value of \it{expression},
with the same type as \tt{value_(}\it{expression}\tt{)}.
In other words, the same type conversions are performed as done by \tt{value_}:
if \it{expression} is an array, then the lvalue is a pointer to base element;
if \it{expression} is a function,
then the lvalue is a corresponding function pointer.
The lvalue has automatic storage duration,
and its lifetime is limited to the innermost block where it is created.

\it{expression} can be evaluated more than
once only if it has a variably modified type.

\note As the outcome is an lvalue, the address-of operator \tt{&} can be
applied to it; however, the resulting pointer should not be used outside the
current block, as the object lifetime expires and the pointer becomes dangling.
