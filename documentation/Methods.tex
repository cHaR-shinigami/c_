\def\Section#1{\section{#1}\input{Methods/#1}}

A C\_ method is essentially a pair of function pointers: protocol and procedure.
The primary motivation behind this design is to separate behavior
from implementation: protocol describes ``$what$'' needs to be done,
whereas procedure specifies ``$how$'' it is actually accomplished.
As an analogy, let us consider the example of making a cake: a protocol would
describe only the externally observable features of the cake that are of
interest to buyers; a procedure would be a detailed step-by-step recipe to
prepare the cake from its ingredients, that is of concern to the baker.

On similar lines, the primary advantage of isolating behavior from
implementation is that a caller needs to be concerned only with the protocol,
and the exact implementation details of the procedure ``$should$ be'' irrelevant.
If a functionality or transformation is conceptually imagined as an opaque box,
then protocol is an abstract specification of the external behavior, whereas
procedure deals with the concrete machinery that operates inside the box.

\section {\idx{public} and \idx{private}}
\tt{public} and \tt{private} are object-like macros;
the reference implementation defines them in \tt{<specifiers._>} header.
\tt{public}  expands to the keyword \tt{inline}, and
\tt{private} expands to the keyword \tt{static} (for internal linkage).

\note These macros are nothing but alternative names,
and they do not add anything new to the language.
However, one advantage of macros is that they can be easily undefined,
which can be convenient for certain purposes; for example, the reference
implementation provides inline definitions for several functions in header
files, using the macro \tt{public} instead of the keyword \tt{inline}.
The benefit is that the inline definitions can be made visible
in multiple translation units, which can be used by compilers
for static analysis and optimizations of function calls.
An external definition is still required for each such function, so the
file \tt{lib.c_} first includes the file \tt{<specifiers._>}, then undefines
the macro \tt{public}, and redefines it with an empty replacement text.
This ensures that when the header files containing \tt{public} function
definition are included in \tt{lib.c_}, they are no longer \tt{inline} definitions.


\section {\idx{no_inline_}}
The \tt{inline} keyword is used to suggest that the code
generated for calling a function should be ``efficient''.
For most purposes, efficiency is desirable in terms of runtime,
and if the generated code for a function definition is small, the code can be
substituted at call site(s), thereby avoiding the overhead of a function call.
However, substituting the code of another function at multiple call
sites can increase size of the object file; also, too many inline
substitutions inside a function can bloat its code to an extent
that the resulting code itself becomes unsuitable for inlining.

\enlargethispage*{\baselineskip}
\pagebreak

The \tt{inline} keyword is only a hint to the compiler (similar
to the \tt{register} keyword); however, even in the absence of an
explicit hint from the programmer, translators can still perform
inline substitutions at their own discretion, if the function
definition is visible in the same translation unit as the function call.
Increasing code size to gain speed (space-time tradeoff)
may not always work as expected, and large code expansion
can actually degrade performance (depending on several factors
of the execution environment, such as instruction caching).
In particular, minimizing the size of an executable is an
important concern for memory constrained devices, even if that
comes at the cost of a tolerable increase in execution time.
In several contexts, it may be desirable to have a portable
mechanism to explicitly disable inline substitutions at
specific call sites, without the use of compiler-specific flags.

\head{Syntax}

\tt{no_inline_ (} $function$ \tt{)}

\head{Semantics}

If $function$ is a function name or a function pointer, then invoking
the outcome of \tt{no_inline_} ensures that the call is not inlined,
even if an inline definition is visible in the translation unit.
The outcome is an expression that compares equal with $function$,
and it has the same function type as $function$; if $function$ is
a function pointer, then type of the outcome is the corresponding
function type obtained on dereferencing the pointer.


\Section{Contracts}

\Section{Prototype}

\Section{Protocol}

\Section{Procedure}

\Section{Invocation}

\Section{Design strategies}
