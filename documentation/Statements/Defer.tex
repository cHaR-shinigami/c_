The defer family allows statements to be postponed, which are executed after
reaching the end of the innermost block that is lexically closed with \tt{refed}.
This feature is useful for registering cleanup or release of resources.

\example Consider a function that needs $n$ resources for some task;
if a resource is not available, then the acquired resources must be released.
With three resources, we can express this requirement with the following code.

\enlargethispage*{\baselineskip}

\code{Statements/Defer/quadratic.c_}

The crucial observation is that resource $i$ has
to be released if another resource is available.
A call to \tt{release} lexically occurs  $i$ times for the $i^{th}$ resource,
so for $n$ resources there will be $n * (n + 1) / 2$
instances of \tt{release} calls, even though at most $n$ such
calls will actually be made (one for each acquired resource).
This bloats the executable as code size grows at a
quadratic rate in terms of the number of resources.
\tt{defer_} offers a more elegant approach by postponing the release of acquired
resources; deferred statements are executed in reverse order of registration.

\note Before presenting a cleaner alternative with \tt{defer_},
we shall take another look at the earlier code,
and observe the use of brace-enclosed nested blocks: the intent is to make
a released resource unavailable for any subsequent use, which is done by
making the pointer variable go out of scope just after calling \tt{release}.
Some programmers advocate setting it to \tt{NULL},
so that defensive null pointer checks can catch an invalid use;
however, that means the variable has to be modifiable,
which opens up the possibility of unintended mutations due to bugs.

We prefer to follow the previous chapter's advice on enforcing
immutability as much as possible, so we rely on scoping rules instead:
every resource is associated with its own lexical scope that starts with
the resource is acquired, and ends when the resource is released; this
ensures that each pointer variable is no longer accessible once its use is over.
Plain nested scopes can prove beneficial in the
long run for functions that manage several resources.
We consider this practice as a part of structured
programming that has also been followed in later chapters.

\subsection{\idX{defer_}* and \idX{refed}*}
Without any further ado, we shall now re-implement our
earlier example with the help of \tt{defer_} and friends.

\code{Statements/Defer/linear.c_}

\tt{defer_} registers expressions and statements to be postponed
at the end of the nearest enclosing block that ends with \tt{refed}.
The name \tt{refed} alludes to the fact that \tt{defer_} statements get
executed in reverse order of reaching them; however, multiple statements
within a single \tt{defer_} are executed in their lexical order.
For example, if one writes \tt{defer_(}\it{expr1}\tt{;} \it{expr2}\tt{)}
followed by \tt{defer_(}\it{expr3}\tt{;} \it{expr4}\tt{)},
they get executed in due course in the order:
\it{expr3}\tt{;} \it{expr4}\tt{;} \it{expr1}\tt{;} \it{expr2}\tt{;} .
Reaching \tt{refed} via \tt{guard_1_} or \tt{stop_1_} executes \tt{defer_}
statements registered within that block; however, \tt{guard_2_} and \tt{stop_2_}
directly return from the function, ignoring deferred statements.

\note Due to a minor flaw in the reference implementation,
\tt{refed} can generate false warnings when it ends a function (other than
\tt{main}) whose return type is not \tt{Void_}; for \tt{gcc} and \tt{clang},
one can disable such warnings with the option \tt{-Wno-return-type}.
It was also observed during testing that \tt{gcc} generates false
warnings for the given example that is implemented using \tt{defer_};
these latter ones can be suppressed with \tt{-Wno-maybe-uninitialized}.


\enlargethispage*{\baselineskip}
\pagebreak

\subsection{\idX{deferrable}* and \idX{start}*}
A function that supports the use of \tt{defer_} is started with
\tt{deferrable} instead of \tt{begin} or a plain opening brace.
If statements are to be postponed to the end of an inner block,
then such a block is started with \tt{start};
reaching its corresponding \tt{refed} executes the statements
deferred within that block in reverse order of registering them.

\example The following code demonstrates execution
sequence of deferred statements within nested blocks.

\code{../compile/defer.c_}

\note Asterisk means that the reference implementation relies
on non-standard extensions for providing these features;
they are reported by \tt{-Wpedantic} (enabled in \tt{cc_}),
but we can suppress such warnings with \tt{-Wno-pedantic}.

\code{Statements/Defer/defer.txt}


\subsection{\idX{return_}* and \idX{yield}*}
A conventional \tt{return} is oblivious of \tt{defer_},
so the pending statements do not get executed.
\tt{return_} takes an expression that is compatible with the function
return type, that is to say, it should be possible to return that
value without an explicit type cast; the expression is first evaluated,
and then deferred statements are executed starting from the innermost block,
moving to outer blocks in the order of their ending.
When deferred statements of the function block started with \tt{deferrable}
has been executed, the expression that was evaluated earlier is returned.

\tt{return_} cannot be used if function return type is \tt{Void_};
as \tt{return} keyword is unaware of pending statements,
an early return should be written as \tt{yield;} which returns
from the function after executing the deferred statements.


\subsection{\idx{DEFER_MAX}}
\input Statements/Defer/DEFER_MAX
