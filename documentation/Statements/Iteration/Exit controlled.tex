We also have looping blocks whose condition is checked after each iteration;
such blocks are executed at least once.

\head{Syntax}

\begin{tabular}{l|l}

\tt{begin} & \tt{begin}\\

\s\s\s\s\it{declarations-and-statements}\opt
&
\s\s\s\s\it{declarations-and-statements}\opt\\

\idx{again_} \tt{(} \it{loop-condition} \tt{)}
&
\idx{end_}   \tt{(} \it{stop-condition} \tt{)}\\

\end{tabular}

\head{Constraints}

Both \it{loop-condition} and \it{stop-condition} shall be scalar expressions.

\head{Semantics}

\it{loop-condition} and \it{stop-condition} are evaluated on
reaching end of that block, or due to a \tt{continue} statement.
Execution of the block is repeated as long as \it{loop-condition} is non-zero,
or \it{stop-condition} is zero, after each iteration.

\note \tt{end} is equivalent to \tt{end_(1)}.
\tt{end_} is equivalent to \tt{again_} with the condition being logically negated.
