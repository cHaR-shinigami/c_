C\_ provides an additional kind of iteration block to
simplify looping over an arithmetic progression of integers.
The conventional approach of using relational operators may not work as expected
in rare cases, such as signed integer overflow on updating a control variable
after the final iteration, which causes implementation-defined behavior.

\example The following code is meant to print only three values,
but it ends up producing a lot more output than one might expect: it is actually
an infinite loop, assuming overflow causes the signed equivalent of wraparound.

\code{../compile/overflow.c_}

Integer loops take care of such fine-grained subtleties,
thereby easing the programmer to focus on the loop body.

\head{Syntax}

\idx{loop_}\s\s\s\tt{(} \it{range} \tt{)}

\idx{loop_}\s\s\s\tt{(} \it{start} \tt{,} \it{stop} [\tt{,} \it{step}] \tt{)}

\idx{loop_1_}\s\tt{(}   \it{range} \tt{)}

\idx{loop_2_}\s\tt{(}   \it{start} \tt{,} \it{stop} \tt{)}

\idx{loop_3_}\s\tt{(}   \it{start} \tt{,} \it{stop} \tt{,} \it{step} \tt{)}

\head{Constraints}

\it{range} shall be an expression of some \tt{Range} type; \it{start},
\it{stop}, and \it{step} shall be expressions of some integer type.

\note \it{range} is pointer to a triplet that encodes an integer sequence;
\tt{Range} types are presented in chapter 5.

\head{Semantics}

\tt{loop_} invokes \tt{loop_}$n$\_ if the
expanded argument sequence contains $n$ arguments.
These are used to iterate over the following arithmetic sequence, where $k$
is a non-negative integer determined from the given relational inequality.

\begin{quotation}
\textsc{relation}\indent
$|start + k*step| \le |stop| \le |start + (k+1)*step|$

\textsc{sequence}\indent
$start, start + step, start + 2*step, \cdots, start + k*step$
\end{quotation}

\tt{loop_1_} obtains \it{start}, \it{stop}, and \it{step} from \it{range}.
For \tt{loop_2_} and \tt{loop_3_}, all arguments are converted to a common type,
which is the type of the expression $start - stop$.
For \tt{loop_2_}, \it{step} is taken as -1 if \it{start} is greater than
\it{stop}, 1 if \it{start} is smaller than \it{stop}, and zero otherwise.
For \tt{loop_3_}, if $start - stop$ has an unsigned type and \it{start} is
greater than \it{step}, then \it{step} is also converted to the same unsigned
type, and \it{start} is moved backwards by the converted step value;
in other words, it considers the mathematical negation of the converted
step value, even though an unsigned type cannot represent negative values.
If \it{step} is specified as zero, two things can happen:
if \it{start} is equal to \it{stop}, then the sequence is a singleton,
and only one iteration is performed; otherwise, the sequence is a pair
and there are two iterations: first for \it{start}, then for \it{stop}.
If the result of adding \it{step} to \it{start} diverges away from \it{stop},
then the sequence is considered to be empty, and no iterations are performed;
more precisely, $start + step$ moves farther away from \it{stop} if
$start - stop$ has a signed type and one of the following conditions is met:

\begin{itemize}[nosep]
\item \it{start} is smaller than \it{stop} and \it{step} is negative.
\item \it{start} is greater than \it{stop} and \it{step} is positive.
\end{itemize}

Within the loop block, the modifiable variable \tt{_i_} stores the current
sequence value that can be updated, and the non-modifiable variables \tt{_omega}
and \tt{_delta} respectively store final sequence value and converted step value.

\pagebreak

\note \it{step} size zero is given a special meaning to specify singleton
sequences, and because sometimes it may not be possible to represent ``one
giant leap'' from \it{start} to \it{stop}; for example, \tt{loop_(min_(Int),
max_(Int), 0)} jumps directly from \tt{min_(Int)} to \tt{max_(Int)},
but the difference between them cannot be represented in type \tt{Int}.

\example The buggy code in our earlier example can be
fixed with \tt{loop_}, which correctly prints three values.

\code{../compile/loop.c_}
