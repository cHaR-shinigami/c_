A block is a sequence of declarations and statements, which are executed
in their lexical order until a branch or jump statement is encountered.
Syntactically, a block as a whole acts as one single statement,
which is why it is also called a compound statement.
A C\_ block is started with \idx{begin},
and its lexical scope extends till the occurrence of a matching \idx{end}.
C\_ blocks can be nested like ordinary blocks:
an inner block ends before the one that encloses it.

Unlike traditional blocks enclosed within curly braces,
C\_ blocks also support guard clauses, which are discussed in the next section.
The primary advantage is that if some condition is (un)satisfied,
then it is possible to bail out early by skipping rest of the statements
within the nearest enclosing block that supports jumping directly to its end.
We refer to this feature as ``early exit'' out of a C\_ block;
the term does not imply exiting the process itself.
