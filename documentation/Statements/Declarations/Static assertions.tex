Static assertions are used to specify pre-conditions for compiling a program.
Unlike runtime assertions which are expressions,
static assertions are syntactically classified as declarations,
so they can be placed anywhere a declaration can occur.
Static assertions are checked during compilation itself,
and they are evaluated in the lexical order as they appear in the source code,
including those within function definitions.
Static assertions are not part of the object code generated by the compiler,
and if they are not satisfied during translation,
it causes a constraint violation.

\head{Syntax}

\idx{static_assert_}\s\s\s\tt{(}   \it{constant-expression}
[\tt{,} \it{string-literal}\tt{="}\it{constant-expression}\tt{"}] \tt{)}

\idx{static_assert_1_}\s\tt{(} \it{constant-expression} \tt{)}

\idx{static_assert_2_}\s\tt{(} \it{constant-expression}
\l\tt{,} \it{string-literal}\r \tt{)}

\head{Constraints}

The first argument shall be a non-zero integer constant expression.

\head{Semantics}

\tt{static_assert_} invokes \tt{static_assert_}$n$\_ if the
expanded argument sequence contains $n$ separate arguments.
\it{constant-expression} is evaluated during compilation: if the result is zero,
the constraint violation is reported with a diagnostic message.
The optional \it{string-literal} is part of the message; if it is omitted,
then the text of \it{constant-expression} is used as the default value.
Translation continues only if the expression is non-zero,
and if it contains any type definition,
then that type is also created in the current scope,
so there cannot be any incompatible redefinition.

\note Static assertions are based on C11 \tt{_Static_assert}, which requires
both the arguments; C23 makes \it{string-literal} optional, but the reference
implementation does not depend on this for providing \tt{static_assert_1_}.

\example \tt{static_assert_(width_(Byte) == 8,
"POSIX requires exactly eight bits in a byte")}
