\def\Subsubsection#1{\subsubsection{#1}\input{Statements/Branching/switch_/#1}}

\head{Syntax}

\tt{switch_ (} \it{selection} \tt{)}

\s\s\s\s\it{declarations-and-statements}\opt

\tt{end}

\head{Constraints}

\it{selection} shall be an integer expression.

\head{Semantics}

\tt{switch_} blocks allow case labels of the following forms,
where each \it{constant-expression} shall be a distinct integer.\\

 \tt{case}\s\s\s\s\it{constant-expression} \tt{:} \it{statement}

\idx{CASE}\s\s\s\s\it{constant-expression} \tt{:} \it{statement}

\idx{case_}\s\tt{(}\s\it{constant-expression} \tt{)}
\it{declaration-or-statement}\opt\\

The next two forms are additionally permitted by C23,
but forbidden by older revisions of the C standard.\\

\tt{case}\s\s\s\s\it{constant-expression} \tt{:} \it{declaration}

\tt{CASE}\s\s\s\s\it{constant-expression} \tt{:} \it{declaration}\\

If the \it{selection} expression of \tt{switch_}
is equal to the \it{constant-expression} of a case,
then control jumps directly to that case statement,
skipping any preceding code within the \tt{switch_} block;
thereafter subsequent statements are executed in their lexical order,
until a branch or jump statement is encountered.
If none of the case \it{constant-expression} matches the \it{selection}
expression, then it acts as an ordinary block and executes from its beginning.

The basic \tt{case} is a plain labeled statement,
whereas the variants \tt{CASE} and \tt{case_} additionally
have an implicit unconditional jump immediately before them,
which prevents a fallthrough from the previous statement (if any).
If the control reaches just before a \tt{CASE} or \tt{case_},
then it jumps directly to the end of that \tt{switch_} block.

\note \tt{switch_} is somewhat different from \tt{switch}.
An explicit \tt{default} case is not mentioned in the semantics,
as it is disallowed.
\tt{continue} statements are permitted, but as \tt{switch_}
is not an iteration block, \tt{continue} acts like \tt{break}.
Also, \tt{switch_} works like an ordinary block if none of the cases match;
however, both \tt{CASE} and \tt{case_} have an implicit ``jump to the end''
immediately before them, so if a \tt{switch_} block starts with either of them,
then the non-matching behavior is identical to that of a
conventional \tt{switch} statement: the entire block is skipped.
Conversely, any code that occurs lexically before the
first case gets executed when a matching case is not found.

\example The following program selects a color based on
a random number (modulo 4); if the random selection does
not match any of the given colors, it prints ``Mixed''.
Bitwise AND with 3 gives an integer in the range [0, 3].

\code{../compile/colors.c_}

\note The selection expression is cast to type \tt{Color}
because some compilers can warn if a case is missing for an
enumeration constant of that type; with \tt{gcc} and \tt{clang},
the option \tt{-Wswitch-enum} enables this diagnostic.

\Subsubsection{Fallthrough}

\Subsubsection{Declaration after label}
