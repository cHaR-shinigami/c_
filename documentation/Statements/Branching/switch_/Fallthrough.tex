\tt{case} does not alter the sequential flow of control,
so in the absence of any branch or jump statements,
when the selecting expression is equal to a \tt{case} expression,
subsequent cases that lexically follow the selected case are
also executed in order; proceeding to the next \tt{case} statement
after executing the selected case is termed as fallthrough.

Fallthrough may or may not be desirable,
and often the requirement is to execute only the matching case.
In our previous example, the use of \tt{case_} ensures mutual exclusion:
both \tt{CASE} and \tt{case_} behave as if they are preceded by an
implicit jump statement to the end, so only one color is printed.
One important point to note is that the implicit jump applies to the
end of the nearest enclosing block that supports \tt{break} statement;
this small subtlety may be insignificant for most purposes,
but for the sake of completeness, we exemplify this with a concrete program.

\example The following code demonstrates fallthrough,
where both print statements will get executed.

\code{../compile/fallthrough.c_}
