The formal language grammar specified by older revisions of the C standard
did not allow a declaration to follow a label, which also applied to \tt{case}:
as per the earlier syntactic rule,
the colon after a label could only be followed by a statement.
C23 updates the grammar to allow both declarations and statements;
however, C\_ does not require complete C23 support, so for backward
compatibility with the older rule, the use of \tt{case_} may be preferred.

Both \tt{CASE} and \tt{case_} have an implicit jump preceding them,
but the difference is that \tt{case_} can be followed by a declaration or
statement even by the older C rule, but \tt{CASE} expression should be
followed by a colon and a statement; the latter can have a leading
declaration if the compiler follows the updated C23 grammar for labels.
