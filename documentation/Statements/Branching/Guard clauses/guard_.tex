\head{Syntax}

\idx{guard_}\s\s\s\tt{(} \it{condition} [\tt{,} \it{return-value}\opt]  \tt{)}

\idx{guard_1_}\s\tt{(} \it{condition} \tt{)}

\idx{guard_2_}\s\tt{(} \it{condition} \l\tt{,} \it{return-value}\opt\r\ \tt{)}

\head{Constraints}

\it{condition} shall be a scalar expression, and the optional \it{return-value}
shall be an expression that can be returned by the function without a type cast.
If the function return type is \tt{Void_}, then \it{return-value} shall be blank.

\head{Semantics}

\tt{guard_} invokes \tt{guard_}$n$\_ if the
expanded argument sequence has $n$ arguments.
If \it{condition} does not compare equal to zero,
then control flow is not altered and \it{return-value} is not evaluated;
otherwise the behavior of \tt{guard_1_} is same as that of \tt{break} statement,
and \tt{guard_2_} has the effect of a return statement with \it{return-value}.
