The noticeable slowness in the compilation of C\_ programs is primarily due to
the large scale preprocessing overhead incurred by the reference implementation,
which is by virtue of fundamental limitations in its design itself.
This appendix chapter explores an approach to quantify
performance in terms of number of preprocessing operations.

As discussed in the main chapters, the design of the ellipsis framework
is influenced by microprogramming architecture used in physical processors:
the function-like macros for arithmetic, logical, and relational
operations are implemented with the help of primitive macros
that perform preprocessing micro-operations on argument lists.
The total number of macro invocations required for a given
task can be a benchmark parameter, for which a lower bound can
be estimated by counting how many times the primitive macros
\tt{cat_}, \tt{echo_}, \tt{pop_}, and \tt{top_} are invoked.

A convenient way to do this is with the help of \tt{__COUNTER__} macro,
which is predefined as a GNU C extension (also supported by other compilers).
The unique property of \tt{__COUNTER__} is that it is incremented
every time it is used, with zero being the initial value.
We can modify the primitive macros to use \tt{__COUNTER__} in
such a way that their replacement text is not altered, so that
\tt{__COUNTER__} gets incremented each time a primitive macro is invoked.
Here we present one possible way of updating the primitive
macros without changing their functional behavior.

\code{Benchmarking/primitives._}

A technical subtlety in the incrementation of \tt{__COUNTER__} is that
it is considered to be ``used'' only when it is scanned and expanded.
For this reason, we have introduced two additional macros:
\tt{bean_c_} is invoked by the primitive macros, but as discussed in chapter 4,
macro arguments are not expanded at the call site, so \tt{bean_c_}
calls another macro \tt{beanie_c_}, and it is during this invocation
that \tt{__COUNTER__} is actually expanded and considered to be ``used'',
thereby causing its value to be incremented by one.
On the other hand, \tt{__COUNTER__} would not get updated if the primitive macros
directly invoke \tt{beanie_c_}, as it does not expand its arguments at all.

With this small modification in the header \tt{<primitives._>},
we can get a conservative estimate of macro invocations
required for high-level operations done with the preprocessor.
For example, if we run the preprocessor as \tt{cc_ -E} on the
text \tt{sort_(echo_, RANGE_(125, 1, 2), RANGE_(126, 2, 2))},
followed by \tt{__COUNTER__}, the preprocessed text is the sorted
list of positive integers up to 126, followed by 12157714 as the
value of \tt{__COUNTER__}, which indicates the number of times the
primitive macros were invoked by \tt{sort_} and its helper macros.

It is worth mentioning that this underestimated count is much lower than
the total number of macro invocations; if we consider that each invocation
of a primitive macro is initiated by some other macro (such as the
\tt{o}$n$\_\_ family), then the total count would be more than twice
the value reported by \tt{__COUNTER__}, going well above 24 million.

Such an astonishing scale of macro invocations remains a major
bottleneck in the transpilation of \tt{C_} programs to \tt{C}.
While it is hoped that future improvements on the reference
implementation can lower these numbers, it should be accepted
that a ``giant leap'' in performance can only be possible with
the help of a dedicated compiler frontend for the \tt{C\_} dialect.
Optimistically speaking, this can also open more opportunities for
a wider adoption of C\_ in real-world projects, beyond its humble
beginnings as a small recreational exercise in metaprogramming.
