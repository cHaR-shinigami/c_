\def\Section#1{\section{#1}\input{Statements/#1}}

This chapter discusses various types of statements in C\_.
The most common form of statements is expression statement,
which is an expression followed by a semicolon;
the semicolon acts as a sequence point, which means that pending side effects
before the semicolon are completed before moving on to the next statement.
For example, assuming the return values have arithmetic types in the expression
\tt{f() * g() + h()}, the multiplication is done before addition due to
precedence rules, but the order of evaluation of arguments is unsequenced,
so there are six possible orderings in which these functions can get called.
If we want to enforce a specific order of calling these functions,
we need to separate them out as statements,
and store their return values in temporary variables.\\

\tt{Var a = f();}

\tt{Var b = g();}

\tt{Var c = h();}

\tt{Var sum}\s\s\tt{=}\s\s\tt{a*b + c;}\\

There are several other forms of statements, which are categorized into
selection or branching statements, iteration statements, jump statements, and
compound statements; the latter is a sequence of declarations and statements.
It is important to note that C\_ declarations and statements do not
require a terminating semicolon, which is harmless in most cases,
but should be avoided as a general practice.
This is because an extra semicolon acts as a null statement by itself, which
can be problematic in certain contexts (these are syntax errors, not bugs).

\example The following code fails to compile as \tt{stop_} is itself a statement,
and the next semicolon is a null statement outside the \tt{if}  statement,
which creates a syntactic isolation of the subsequent \tt{else} statement.\\

\tt{Void_ queue(Int);}

\tt{Int_}\s\s\tt{flush();}

\tt{Var_}\s\s\tt{chr = 0;}

\tt{if ((chr = getchar()) == EOF)}

\s\s\s\s\tt{stop_(flush() == EOF);}

\tt{else}

\s\s\s\s\tt{queue(chr);}\\

Same problem occurs when an external C\_ declaration is followed by a semicolon;
the latter acts acts a null statement, which is disallowed outside functions.
For these reasons, an unnecessary semicolon should be avoided.

\Section{Declarations}

\Section{Blocks}

\Section{Branching}

\Section{Iteration}

\section{Defer*}
\input Statements/Defer.tex
