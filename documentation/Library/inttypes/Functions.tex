\tt{<inttypes._>} additionally provides inline
definitions for the following functions:

\table{l}

 \tt{UIntmax_} \idx{uimaxabs}\tt{(Intmax);}\\

 \tt{UIntmax_} \idx{gcd}\s\tt{(UIntmax, UIntmax);}\\

 \tt{UIntmax_} \idx{umax}\tt{(UIntmax, UIntmax);}\\

 \tt{UIntmax_} \idx{umin}\tt{(UIntmax, UIntmax);}\\

\s\tt{Intmax_} \idx{smax}\tt{( Intmax,}\s\s\tt{Intmax);}\\

\s\tt{Intmax_} \idx{smin}\tt{( Intmax,}\s\s\tt{Intmax);}\\

\elbat

\tt{uimaxabs} is similar to \tt{imaxabs}, except that the return type is
\tt{UIntmax_}: it returns the absolute value of an integer, and the behavior
is well-defined for all values in range of the parameter type \tt{Intmax}.

\tt{gcd} returns the greatest common divisor of two non-negative integers;
output is zero iff both inputs are zero.

\tt{umax} and \tt{umin} respectively return the maximum and minimum of two
unsigned integers; \tt{smax} and \tt{smin} are the signed integer counterparts.
