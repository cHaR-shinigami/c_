The \tt{assert_} family can be used to ensure
that some condition is satisfied at runtime.
Unlike the \tt{pre_} and \tt{post_} facilities,
the \tt{assert_} family can be customized during translation:
when compiled in debugging mode, failure of a given condition causes
a diagnostic message to be printed on the standard error stream and the
process terminates with exit status as one; when compiled in production mode,
the condition is evaluated but the result is ignored.

\note C\_ \tt{assert_} family is not a wrapper over
the C \tt{assert} macro defined in \tt{<assert.h>};
as with other extension headers, \tt{<assert.h>} is included by \tt{<assert._>},
and the \tt{assert} macro can be configured with \tt{NDEBUG}.

\head{Syntax}

\tt{# include <assert._>}\\

\idx{assert_}\s\s\s
 \tt{(} \it{condition}
[\tt{,} \it{text}\tt{="}\it{condition}\tt{"}
[\tt{,} \it{site}\tt{=SITE}
[\tt{,} \it{sink}\tt{=stderr}
[\tt{,} \it{DEBUG}\tt{=ASSERT__}]]]]
 \tt{)}

\idx{assert_1_}\s\tt{(}
\it{condition} \tt{)}

\idx{assert_2_}\s\tt{(}
\it{condition} \tt{,}
\it{text} \tt{)}

\idx{assert_3_}\s\tt{(}
\it{condition} \tt{,}
\it{text} \tt{,}
\it{site} \tt{)}

\idx{assert_4_}\s\tt{(}
\it{condition} \tt{,}
\it{text} \tt{,}
\it{site} \tt{,}
\it{sink} \tt{)}

\idx{assert_5_}\s\tt{(}
\it{condition} \tt{,}
\it{text} \tt{,}
\it{site} \tt{,}
\it{sink} \tt{,}
\it{DEBUG} \tt{)}

\head{Constraints}

\it{condition} shall be a scalar expression.
\it{text} shall be a string.
\it{site} shall be of type \tt{Site}.
\it{sink} shall be of type \tt{Stream}.
\it{DEBUG} shall expand to either \tt{0} or \tt{1}.

\head{Semantics}

\tt{assert_} invokes \tt{assert_}$n$\_ if the
expanded argument sequence contains $n$ arguments.
Outcome of the expression is of type \tt{Bool_}:
when compiled with \it{DEBUG} expanding to \tt{0},
the outcome is \tt{(Bool)(}\it{condition}\tt{)};
otherwise \it{DEBUG} expands to \tt{1} and the outcome is \tt{(Bool) 1}.

When compiled with \it{DEBUG} expanding to \tt{1}, if \it{condition} compares
equal to zero, the process prints a diagnostic message of the following form,
and then terminates by calling \tt{exit(1)}.\\

\tt{Assertion failed:} \it{condition}\tt{, function}
\it{function-identifier}\tt{, file}
\it{file-name}\tt{, line}
\it{line-number}\tt{.}\\

The stringified form of \it{condition} is used as default,
which can be customized with the argument \it{text};
an empty string is used if \it{text} is null.
\it{function-identifier}, \it{file-name}, and \it{line-number}
indicate the location where the assertion failed.
The message is written to the standard error stream \tt{stderr} by default,
which can be changed with the \it{sink}.
\tt{stderr} is also the fallback if \it{sink} is null; the behavior
of writing to \it{sink} is undefined if it is not an output stream.

All the arguments are always evaluated exactly once,
regardless of whether \it{DEBUG} expands to \tt{0} or \tt{1}.
