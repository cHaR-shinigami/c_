The ellipsis framework provides function-like macros for performing various
arithmetic, logical, and relational operations on non-negative integers.
These features are modeled after machine instructions in
a physical processing unit, and like any other translator,
the ellipsis framework can operate only on a limited set of values.
This limit is determined by the macro \tt{PP_MAX}, which the reference
implementation defines as 127; this value was chosen because C compilers can limit
the maximum number of arguments permitted in a function-like macro invocation,
and this upper limit must be at least 127 (for all conforming compilers).
The C standard does not encourage the use of fixed translation limits,
and most preprocessors have a reasonably larger limit on the number
of macro arguments (it should be acknowledged that an upper limit
is inevitable due to the finiteness of memory address space).

\tt{PP_MAX} is required to be at least 127, and for maximum
portability across implementations, its value should not be changed.
However, C (and thus C\_) programs can be non-portable, and this upper
limit can easily be raised: the only pre-requisite is that the underlying C
processor must support at least \tt{PP_MAX} arguments in a macro invocation.

The reference implementation of the ellipsis framework follows a modular design,
and it is quite trivial to support operations on larger integers simply
by updating a few macro constants and extending the \tt{o}$n$\_\_
family in an inductive manner (defined in the header \tt{<meta._>});
all in all, the entire process is fairly mechanical.

The following changes need to be made in the header \tt{<count._>},
located in \tt{.include/ellipsis/} directory:

\begin{itemize}

\item Define a value greater than 127 in the replacement text of \tt{PP_MAX}.

\item Change the replacement text  of  \tt{PP_MAW}
to one less than the value defined for \tt{PP_MAX}.

\item Update the replacement text of \tt{PP_LOG2} with
$\lfloor\log_2($\tt{PP_MAX}$)\rfloor$ (truncating any fractional part).

\item Compute the value of $\lfloor\sqrt{\verb|PP_MAX|}\rfloor$
(truncating any fractional part) and update \tt{PP_SQRT} with this result.

\item Update \tt{PP_INT} as the sequence of integers from
\tt{PP_MAW} - 1 through 1 (in reverse or decreasing order).

\end{itemize}

The following changes is required in the header \tt{<utilities._>},
also located in \tt{.include/ellipsis/} directory:

\begin{itemize}

\item Update \tt{PP_RANGE} as reverse of \tt{PP_INT}, $i.e.$
the increasing sequence of integers from 1 through \tt{PP_MAW} - 1.

\end{itemize}

Finally, for each integer $i$ greater than 127 and up to the
updated value of \tt{PP_MAX}, define a macro \tt{o}$i$\_\_ as:

\begin{center}
\tt{#define o}$i$\tt{__(F, f) F(f)o}$i-1$\tt{__(F, f)}
\end{center}

In other words, each \tt{o}$i$\_\_ macro invokes the one immediately before that.
The default \tt{o}$n$\_\_ family up to \tt{o127__} is
defined in the header \tt{<meta._>}, and additional macros
beyond \tt{o127__} should be defined in the same header.
