\def\Section#1{\section{#1}\input{Arrays/#1}}

An array is a contiguous sequence of objects having the same type.
An array having one element is allocated in the same way as a non-array
lvalue of the same type; if there are more elements, they are allocated
successively after the first element without any padding between two elements:
thus the byte offset between two elements is equal to size of the element type.
It is permissible to obtain a pointer to one past the last element:
it points to the address that is one byte after the end of the last element,
which is outside the array and should not be accessed; however, the pointer
can be used for comparison with another pointer to an element of the same array.

Elements of an array are indexed from zero (base element).
An element can be accessed by specifying its index with the subscript operator:
both $array$ \tt{[} $index$ \tt{]} and $index$ \tt{[} $array$ \tt{]} are equivalent
to \tt{* (} $array$ \tt{+} $index$ \tt{)}, all of which access the element at
the given $index$; the outcome is an lvalue, so the address-of operator can be
applied to it (assignment operators can be used only if the type is modifiable).
Note that when address arithmetic is performed directly on an array name,
it is implicitly ``decayed'' to a pointer for the base element.

An array definition must specify a complete type, as the length is required to
determine allocation; however, an array declaration can omit the length, making
it an incomplete array that cannot be used as the operand of \tt{sizeof}.\\
Recall that in C, function parameters of array types are adjusted as pointer
to the corresponding element type, so applying \tt{sizeof} to a parameter
declared with array type is permitted, though it gives only the size of
the adjusted pointer type (which may not be the intent of the programmer).
Also, declarations of external arrays can be incomplete.
However, the element type of an array cannot be incomplete, so if the
element type is also an array type (often called multi-dimensional arrays),
then it must have a complete type that contains the length information.
In other words, only the ``outermost'' length of a multi-dimensional array
can be omitted in declarations, but the inner dimension(s) must be present,
in order to determine the byte offset between two consecutive elements.

\note Zero length arrays are not allowed by the ISO C standard,
and they can cause undefined behavior.

\Section{Variably modified types}

\Section{Pointer to an array}

\Section{Length}

\section{\idx{is_array_}}
\input Arrays/is_array_

\Section{Indexing}

\Section{Synonyms}

\pagebreak

\Section{Range}

\Section{Bit arrays}

\Section{Iterators}
