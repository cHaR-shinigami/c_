\head{Syntax}

\tt{isNBO_ (} \it{type-name} \tt{)}

\head{Constraints}

\it{type-name} shall specify an integer type.

\head{Semantics}

\idx{isNBO_} returns one if \it{type-name} specifies a
multibyte type that uses network byte ordering or big-endian,
and it returns zero if the type uses little-endian representation;
in either case, the outcome is of type \tt{Bool_}.
The outcome is zero for the single-byte types \tt{UByte}, \tt{Char}, \tt{Byte},
and it is implementation-defined for any other byte ordering.

\note Endianness refers to the ordering of bytes for types wider than \tt{Byte}:
big-endian means a byte with more significant value is stored at lower address;
little-endian is the reverse ordering, where the least
significant byte comes first, followed by more significant bytes.
Many networking protocols use big-endian representation for data transmission,
which can require a conversion (reversal of bytes) if the
native byte ordering for a host machine is little-endian.
Very rarely, exotic architectures can support some other permutation of bytes,
which are collectively grouped as middle-endian or mixed-endian:
one historical example is the PDP-endian ordering.
For the reference implementation, \tt{isNBO_} only checks the starting
byte value when the integer 1 is represented using the given type.

\example Most common architectures natively support
\tt{width_(Byte) == 8} and \tt{sizeof (Short) == 2}.
If \tt{Short} uses big-endian, \tt{256} (\tt{0x0100} in hexadecimal) would be
stored as \tt{0x01 0x00} (\tt{0x01} at starting address, followed by \tt{0x00});
on the other hand, it would be stored as \tt{0x00 0x01} in little-endian form.
