\def\Subsubsection#1{\subsubsection{#1}\input{Types/Properties/Width/#1}}

\head{Syntax}

\tt{width_ (} \it{type-name} \tt{)}

\head{Constraints}

\it{type-name} shall specify an integer type.

\head{Semantics}

\idx{width_} returns the number of useful bits for the type specified by
\it{type-name}: for unsigned types, it counts the number of value bits,
and for signed types, it also counts the sign bit along with value bits;
padding bits are ignored.

\example \tt{width_(Bool)} is always 1; both \tt{width_(UBitInt (}$w$\tt{))}
and \tt{width_(BitInt (}$w$\tt{))} are equal to $w$.

\Subsubsection{Precision}
