A variable designates a memory object commonly known as lvalue;
if the lvalue can be accessed by an expression other than the variable,
then that expression is an alias:
the simplest form of aliasing is a pointer to the variable.

\example Non-modifiable types only disallow updating the variable,
but the lvalue can still be modifiable.\\

\tt{Int capacity = BUFSIZ;}

\tt{/* ++capacity; // constraint violation */}

\tt{++*(Int_ *)&capacity; // can compile}\\

To make a variable truly immutable, we need to enforce non-aliasing,
which can be done by declaring with \idx{let}.

\head{Syntax}

\tt{let} \it{variable-declaration}

\head{Constraints}

\it{variable-declaration} shall not declare a function,
but it can declare a function pointer.

\head{Semantics}

For a block scope declaration that does not specify any storage-class,
\tt{let} disallows aliasing,
so a pointer cannot be obtained to the lvalue designated by the variable;
this also applies to function parameters.
The behavior is implementation-defined if \tt{let} is used in conjunction
with any storage-class specifier, or with external declarations.

\example In the earlier code, \tt{&capacity} will cause an
error if we declare it as \tt{let Int capacity = BUFSIZ;}

\head{Recommended practice}

The reference implementation provides \tt{let} as the \tt{register} keyword,
so it cannot be used with external declarations,
or with other storage-class specifiers;
other implementations can provide \tt{let} as a separate built-in feature.

\note \tt{let} can be used with \tt{auto} only if the latter
is meant for type inference, which is standardized in C23.
The use of \tt{let} is encouraged, as disabling aliasing
can sometimes provide an opportunity for better optimizations.
