The design of C is characterized by economy of syntax,
which can make some of its rules less intuitive for beginners;

\example C uses the same syntax for declaring
pointer variables and dereferencing pointer expressions.

\tt{Char *}

\tt{p_ = 0,}

\tt{ch = 0;}

The previous declaration is badly misleading: it can create a wrong
impression that both \tt{p_} and \tt{ch} are pointers of type \tt{Char *}.
However, the \tt{*} is part of the variable and not the type,
so \tt{p_} is declared as a modifiable pointer to \tt{Char},
whereas \tt{ch} is a non-modifiable variable of type \tt{Char}.
The same declaration can be better written as:\\

\tt{Char}

\tt{*p_ = 0,}

\s\tt{ch = 0;}\\

The rewritten declaration conveys the intent more clearly:
both \tt{*p_} and \tt{ch} are of type \tt{Char},
so \tt{p_} itself must be a pointer to \tt{Char}.
But now there is another subtlety: an assignment expression \tt{*p_ = 0} means
the dereferenced lvalue \tt{*p_} is set to zero, but within a definition,
the initialization \tt{*p_ = 0} means that \tt{p_} is set to the null pointer.

C\_ takes a different approach: we can avoid such confusion
entirely by declaring pointers with \type{Ptr} or \tt{Ptr_}.

\head{Syntax}

\tt{Ptr}\s\s\tt{(} \it{type-name} \tt{)}

\tt{Ptr}\_\s\tt{(} \it{type-name} \tt{)}

\head{Constraints}

\it{type-name} shall not provide an incompatible redefinition of an aggregate
type that has already been defined in the current scope, and it shall not
redefine an enumeration type that was earlier defined in the same scope.

\head{Semantics}

\tt{Ptr} specifies the type ``non-modifiable pointer to \it{type-name}'',
whereas \tt{Ptr_} specifies the type ``modifiable pointer to \it{type-name}''.
If \it{type-name} contains a type definition,
then that type also gets defined in the current scope.

\example In the following declaration,
both \tt{p_} and \tt{q_} are modifiable pointers to \tt{Char}.\\

\tt{Ptr_(Char)}

\tt{p_ = 0 ,}

\tt{q_ = p_;}\\

\note \tt{Ptr} and \tt{Ptr_} denote modifiability of the pointer variable,
which does not affect modifiability of the dereferenced lvalue;
the latter is decided solely by \it{type-name}.

\example Parameters of the \tt{swap} function are non-modifiable,
but the dereferenced lvalue can be updated.

\code{Types/Pointers/Declarations/swap.c_}
