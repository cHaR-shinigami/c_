Pointer to a modifiable type can be converted as pointer to the corresponding
non-modifiable type, but the other way around should be avoided.
For example, an expression of type \tt{Ptr (Char_)} can be assigned to a
variable of type \tt{Ptr (Char)}, but attempting the converse will trigger a
warning from most compilers; in particular, our use of the option \tt{-Werror}
in the \tt{cc_} command alias will turn this into a compilation error.
C allows casting away the non-modifiability, but our use of the option
\tt{-Wcast-qual} disallows that as well; the intent is to diagnose a potentially
erroneous type cast, where the type should actually be non-modifiable.
Explicitly discarding non-modifiability is strongly discouraged, but in some
cases this can be unavoidable; such a need may arise during the implementation
of legacy APIs, where changing some declaration can break existing code.
The standard library has some notable examples, such as the function \tt{strchr}
declared in the header \tt{<string.h>}: it accepts a pointer to
\tt{Char} as one of its arguments and returns a pointer to \tt{Char_},
which points to a part of the same object as pointed to by the argument.
An alternative design is to return an offset relative to the argument pointer,
but most library functions are based on ``prior art'' from the early days of C,
when non-modifiability was not as prioritized as it is today.
Nevertheless, in recognition of the occasional necessity for discarding
non-modifiability, C\_ provides \idx{unqual__} and \idX{unqual_}*.

\head{Syntax}

\tt{unqual_}\_\s\tt{(} \it{pointer} \tt{)}

\tt{unqual_}\s\s\tt{(} \it{pointer} \tt{)}

\head{Constraints}

The argument shall be pointer to an object type.

\head{Semantics}

Outcome of both \tt{unqual__} and \tt{unqual_} has the same value as
\it{pointer}, but its type is pointer to the corresponding modifiable type,
without any type qualifier; more precisely, if the argument is pointer
to a possibly qualified type \tt{T}, then the result is the same pointer,
but typed as \tt{typeof_unqual (T) *}.
\tt{unqual__} can evaluate the argument more than once only if it is pointer
to a variably modified type; \tt{unqual_} evaluates the argument exactly once.

\example The following code exemplifies a need to forgo
non-modifiability with an implementation of \tt{strchr}.

\code{Types/Pointers/Compatibility/c_strchr.c_}

\note Strong type safety is an integral aspect of C,
and it is also respected by C\_.
Concerning the potential abuse of \tt{unqual__} and \tt{unqual_},
C\_ trusts the programmer to use them judiciously,
and it is the responsibility of API designers to declare prototypes in
such a way that implementors are not compelled to discard non-modifiability.
