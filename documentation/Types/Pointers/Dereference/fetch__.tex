\head{Syntax}

\tt{# include <pointer._>}\\

\idx{fetch__}\s\s\s\s\tt{(} \it{pointer} [\tt{,} \it{default-value}] \tt{)}

\idx{fetch__1__}\s\tt{(} \it{pointer} \tt{)}

\idx{fetch__2__}\s\tt{(} \it{pointer} \l\tt{,} \it{default-value}\r\ \tt{)}

\head{Constraints}

The first argument shall be pointer to a complete object type.
If \it{pointer} is an expression of type ``pointer to \tt{T}'',
then it shall be possible to assign \it{default-value} to a
variable of type \tt{T}, without requiring an explicit type cast.

\head{Semantics}

\tt{fetch__} invokes \tt{fetch__}$n$\_\_ if the
expanded argument sequence has $n$ arguments.
If \it{pointer} is not null,
then outcome is the lvalue obtained on dereferencing the pointer.
When \tt{POINTER__} expands to \tt{1}, if \it{pointer} is null,
then the behavior is same as if \tt{notnull__(}\it{pointer}\tt{)} is invoked.
When \tt{POINTER__} expands to \tt{0}, a null pointer is not dereferenced,
in which case the outcome is \it{default-value},
but with the same type as of dereferencing the pointer.

If \it{pointer} is of type \tt{T *},
then the outcome is an lvalue of type \tt{T}.
When \tt{POINTER__} expands to \tt{0}, if \it{default-value} is used as the
outcome, then the dereferenced lvalue is a temporary object with automatic
storage duration, whose lifetime ends after the innermost enclosing block.
\it{pointer} can be evaluated more than once only if
 it points to an object with variably modified type;
\it{default-value} is always evaluated once, even if its outcome remains unused.

\note As the outcome is an lvalue, the dereference can be
suppressed by applying address-of operator \tt{&} to the result,
which does not impact null pointer diagnosis in debugging mode compilation
(\tt{POINTER__} expands to \tt{1}).
