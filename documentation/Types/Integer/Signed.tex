Signed types can represent positive and negative
integers within a nearly-symmetric range.
Their representation contains the necessary value bits to support the range,
one sign bit, and optional padding bits (not allowed in \tt{Byte}).
C23 requires two's complement representation,
but C\_ does not forbid one's complement and sign-magnitude forms.

The standard signed types are listed in non-decreasing order of
their actual width and increasing order of rank; the range
of a lower ranked type is contained within the range of a higher ranked type.
In arithmetic expressions, \tt{Byte} and \tt{Short} are promoted to \tt{Int_},
and if required by the operator, a lower ranked operand is promoted to
the type of a higher ranked operand, which is also the type of the result
(overflow behavior is implementation-defined).

\table{cc}
Type & Minimum width\\
\type{Byte}  &  8\\
\type{Short} & 16\\
\type{Int}   & 16\\
\type{Long}  & 32\\
\type{LLong} & 64
\elbat

\note Width includes sign and precision bits.
\tt{Byte_} is a synonym for \tt{signed char}, so \tt{sizeof (Byte)} is 1.
For two's complement representation,
a signed type cannot represent the negation of its most negative value.
