\def\Subsection#1{\subsection{#1}\input{Types/Pointers/#1}}

We start this section with a quote by Donald Knuth,
from his article \it{Structured Programming with go to Statements}:

``\it{I do consider assignment statements and pointer variables
to be among computer science's `most valuable treasures'.}''

Pointers are ubiquitous in C,
and their basic use is to refer to an lvalue by its memory address.
The address of an lvalue expression (often just a variable) can be obtained
with the unary operator \tt{&} (its binary form is used for bitwise AND).
The pointed-to object can be accessed by dereferencing the pointer
with unary operator \tt{*} (its binary form denotes multiplication);
the dereferenced expression denotes an lvalue, so it can be subjected to
the \tt{&} operator, in which case the dereference is not done at runtime.
Using pointers, the same object can be referred to in multiple places
without creating separate replicas that increase memory footprint;
this approach works best if the shared object is immutable, so we can have
multiple readers but no writers: one classic example is string literals.

As an analogy, consider a realtor who advertises the address of a
house to potential buyers, all of whom can come and view the house,
but none of them can alter it (until they buy it of course).
The idea of physically replicating an entire
house for each interested buyer is absurd.
On a similar reasoning, aggregate types (structures and unions) can have large
sizes, and passing them by value can be a significant function call overhead;
this can be easily avoided by passing a pointer to the data,
which has a small fixed size (commonly 4 bytes or 8 bytes,
and sometimes even 2 bytes on memory-constrained devices).
Number of usable bits in a pointer determines size of virtual address space.

Another common use of pointers is to modify an lvalue outside its lexical scope,
but within the object's lifetime;
the traditional example is swapping the values of two
variables with a function call, which we shall discuss soon.
Dynamic memory allocators work on the same principle:
a function such as \tt{malloc} returns the pointer to an untyped object of the
required size, and the object can be subsequently accessed outside the lexical
scope where it was allocated, as its lifetime continues until a \tt{free} call.
A pointer can itself be a memory object, and therefore an lvalue, so we
can have pointer to pointer, which is sometimes called a ``double pointer''
(not to be confused with ``pointer to \tt{double}''); similarly, we can
have higher levels of indirection that will require additional dereferences.

One underappreciated use of pointers is for working with opaque data types,
where the structural representation of some type is hidden by the interface,
which only names the type.
In C, these can be implemented with incomplete structures and unions,
where only the tag is declared, but the type definition is not available.
Pointers to incomplete types cannot be dereferenced,
as the type information is unavailable; instead, a set of functions acts as an
interface that provide controlled access to various attributes of each object.
We shall defer an elaborate discussion on data hiding and encapsulation to
chapters 7 and 8, where we reach a middle ground called ``translucent object'',
where the dynamic type information is available,
but the structural details are abstracted away from the programmer.

\note Pointers are not integers, even though their representation
is typically an unsigned integer that denotes a memory address;
on modern hosted environments, this is a virtual address.
Pointers are not even arithmetic types,
even though we can add a valid integer offset to a pointer
(such that the resulting pointer points to some part of the same object),
or subtract two pointers that both point to parts of the same object.
Being able to obtain a memory address from a pointer is only its bare minimum
requirement; additional bits may be utilized to store extra information known as
``provenance''; however, such possibilities have not yet been explored in C\_.

\Subsection{Declarations}

\subsection{\tt{Pointer} type}
\type{Pointer} is the generic object pointer type,
which can be assigned to and from any object pointer without type cast.

\note \tt{Pointer_} is a synonym for \tt{volatile Void *};
it may not be suitable for representing function pointers.


\subsection{\tt{is_pointer_}}
\input Types/Pointers/is_pointer_

\Subsection{Null pointer}

\Subsection{Dereference}

\Subsection{Compatibility}
