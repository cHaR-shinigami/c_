Except \tt{False} and \tt{True}, the types discussed so far are
synonyms for C types, which can be created with \idx{typedef_}.

\head{Syntax}

\tt{typedef\_ (} \it{Synonym} \tt{,} \it{type-name} \tt{)}

\head{Constraints}

\it{Synonym} shall be an identifier;
if that identifier is redeclared within the same scope,
then the other declaration shall also be a synonym for the same type.
\it{type-name} shall be an object type;
within a single scope, it shall not redefine a tagged enumeration type,
and shall not provide an incompatible redefinition for a tagged aggregate type.

\head{Semantics}

\tt{typedef_} is syntactically a declaration.
It creates two synonyms for \it{type-name}:
\tt{Synonym} is non-modifiable, whereas \tt{Synonym_} is modifiable.
\it{type-name} can specify any object type, including array or function pointer,
such as \tt{Char [BUFSIZ]} or \tt{Void_ (*)(Void_)};
it can also be an incomplete type.
If \it{type-name} contains the definition of an
enumeration or aggregate type (\tt{struct} or \tt{union}),
then that type itself gets defined in the current scope.
If \it{type-name} is a variably modified type, then it can be evaluated
to determine sizes of variable length arrays (VLA).

\head{Recommended practice}

For consistency with our naming conventions, the identifier \it{Synonym} should
begin with an uppercase letter, contain at least one lowercase letter, and
not end with an underscore; it should also not end with \tt{_C} or \tt{_c}.
Also, if \it{type-name} is a variably modified type,
then side effects should be avoided, which can be hoisted outside \tt{typedef_}.
