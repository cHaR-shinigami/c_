\head{Syntax}

\tt {Auto\ } \it{identifier}\s\s\tt{=} \it{initializer} \tt{;}

\tt {Auto\_} \it{identifier}\_\s\tt{=} \it{initializer} \tt{;}

\s\tt{Var\ } \it{identifier}\s\s\tt{=} \it{initializer} \tt{;}

\s\tt{Var\_} \it{identifier}\_\s\tt{=} \it{initializer} \tt{;}

\head{Constraints}

\it{initializer} shall be an expression,
and a comma expression shall be parenthesized.

\head{Semantics}

\type{Auto} defines a non-modifiable variable named \it{identifier} whose
type is inferred from resulting type of \it{initializer} expression,
after performing lvalue conversion, array to pointer decay,
and function to function pointer conversion;
the expression is evaluated and the variable is initialized to its value.
The behavior is implementation-defined if the
\it{initializer} expression contains any type definition,
or if multiple variables are defined in a single declaration.

\tt{Auto_} is the modifiable twin of \tt{Auto},
and defines a variable which can be updated thereafter.
\type{Var} and \tt{Var_} are respectively equivalent
to \tt{Auto} and \tt{Auto_} preceded by \tt{let};
in other words, they make the variable unaliasable.

\head{Recommended practice}

Variable names should start with a lowercase
letter to differentiate them from type names.
As a convention,
\it{identifier} in \tt{Auto} and \tt{Var} should not end with underscore;
\it{identifier\_} in \tt{Auto_} and \tt{Var_} should end with underscore.

\note Type inference cannot be used to define arrays or functions,
which get converted to pointer types.
