Searching is a basic functionality that is required for most applications.
A bare minimum design would require two inputs:
a collection and an element to find in the collection.
If the element is found and the collection is a sequence,
it may be desirable to know the position at which element
was found; usually the first occurrence is returned.

\code{../compile/search/solver.c_}

The given code looks through an array in increasing order of indices, stopping
when the element is found, and index of the first occurrence is returned;
if return value equals array length, it indicates the element was not found.

The function \tt{finder} is an example of solver that
actually does the work of going through an array and
comparing each of its elements with the given search value.
It assumes that \tt{arr} points to a valid
\tt{Int} array having (at least) \tt{len} elements.
The implementation uses an iterative approach to
access the array elements from low to high indices.

To express the behavior of searching without specifying any particular
implementation detail, we can write another function that verifies the outcome
of \tt{finder} for a given pair of arguments: this would be a post-condition.
Additionally, this function would also be responsible for
argument validation: in this example, it can check that the
array pointer is not null, which would be a pre-condition.
The following function \tt{find} concretizes this specification.

\code{../compile/search/verifier.c_}

In a sense, both pre-conditions and post-conditions are a form of guard clauses
that allow execution to continue only if the given condition is satisfied.
In this example, there are two pre-conditions: validating that the length is
non-zero and the array pointer is not null (note that null is just one invalid
pointer that can be easily diagnosed; there is no general mechanism in C
to detect whether a pointer refers to a valid object for the given type).

Fulfilling the pre-conditions is a responsibility of the calling function; a
function can be called from multiple locations, and if any pre-condition is
unsatisfied, it is important to know the precise invocation that caused a violation.
The limitation of writing pre-conditions as assertions (such as with the
\tt{assert} macro from \tt{<assert.h>}) is that it only reports a violation
for an argument, not the offending invocation that supplied the bad argument.
In our example, the first parameter \tt{_site} is meant to capture call site
details, which is used by the \tt{pre_} family to provide additional information
in the diagnostic message; for an invocation of \tt{find}, the non-modifiable
lvalue \tt{SITE} can be used as the first argument to capture the values
of \tt{__func__}, \tt{__FILE__}, and \tt{__LINE__} at the call site.

After the pre-conditions are found to be satisfied, \tt{find} invokes the solver
\tt{finder} and verifies that the return value satisfies the given post-conditions.
The first post-condition verifies that the outcome is within a valid range.
The second post-condition verifies that for all indices less than
the outcome, none of the elements are equal to the search value;
this describes the specification of finding the first occurrence
of search value when moving from low to high indices.
If outcome is less than array length, it is a valid index, and the third
post-condition verifies that the corresponding element is equal to search value.
A verifier should not alter the return value, and treat
it as read-only: once the post-conditions are satisfied,
verifier simply returns the unmodified outcome given by solver.

The following code shows how the invocation can be simplified with a wrapper
macro: the first argument would typically provide call site details with
\tt{SITE}, and length of a complete array can be inferred with \tt{length__}.
Most invocations would provide these details in the same manner, so they need
not be specified for each call: the macro \tt{find__} eliminates such boilerplate
code in source files by pushing extra arguments necessary for the function call.

\code{../compile/search/main.c_}

\note Throughout the rest of this documentation, we shall use the term
``validation'' for pre-conditions on the arguments supplied by the caller, and
``verification'' for post-conditions on the return value given by the solver.

\head{Recommended practice}

Diagnostic messages become meaningful and precise
when multiple conditions are specified separately;
if they are written as a single logical conjunction,
it becomes more tedious to find out which condition was violated.
