\head{Syntax}

\idx{post_}\s\s\s\tt{(} $condition$
[\tt{,} $text$\tt{="}$condition$\tt{"}]
[\tt{,} $site$\tt{=SITE}] \tt{)}

\idx{post_1_}\s\tt{(} $condition$ \tt{)}

\idx{post_2_}\s\tt{(} $condition$
\phantom{[}\tt{,} $text$\phantom{]} \tt{)}

\idx{post_3_}\s\tt{(} $condition$
\phantom{[}\tt{,} $text$\phantom{]}
\tt{,} $site$ \tt{)}

\head{Constraints}

\it{condition} shall be a scalar expression.
\it{text} shall be a string.
\it{site} shall be of type \tt{Site}.

\head{Semantics}

\tt{post_} invokes \tt{post_}$n$\_ if the
expanded argument sequence contains $n$ arguments.

If $condition$ compares equal to zero, \tt{post_1_}
prints a diagnostic message of the following form:

\begin{center}
\tt{Post-condition failed:} \it{condition}\tt{, function}
\it{function-identifier}\tt{, file}
\it{file-name}\tt{, line}
\it{line-number}\tt{.}
\end{center}

\tt{post_2_} and \tt{post_3_} print \it{text}
instead of \it{condition} in the message.
\tt{post_1_} and \tt{post_2_} obtain \it{function-identifier}, \it{file-name},
and \it{line-number} from \tt{__func__}, \tt{__FILE__}, and \tt{__LINE__}
(respectively), whereas \tt{post_3_} obtains these ``source code
coordinates'' from \it{site}; if \tt{(}$site$\tt{).func} or
\tt{(}$site$\tt{).file} is null, an empty string is used.

In all cases, the diagnostic message is written to the standard error stream
\tt{stderr}, and the process terminates as if by calling \tt{exit(1)}.
\it{text} and \it{site} are always evaluated once;
their results are discarded if \it{condition} is non-zero.
