\head{Syntax}

\idx{pre_}\s\s\s\tt{(} $condition$
[\tt{,} $text$\tt{="}$condition$\tt{"}
[\tt{,} $site$\tt{=SITE}]] \tt{)}

\idx{pre_1_}\s\tt{(} $condition$ \tt{)}

\idx{pre_2_}\s\tt{(} $condition$
\phantom{[}\tt{,} $text$\phantom{]} \tt{)}

\idx{pre_3_}\s\tt{(} $condition$
\phantom{[}\tt{,} $text$\phantom{]}
\tt{,} $site$ \tt{)}

\head{Constraints}

\it{condition} shall be a scalar expression.
\it{text} shall be a string.
\it{site} shall be of type \tt{Site}.

The \tt{pre_} family can only be used inside blocks where the
identifier \tt{_site} refers to a variable of type \tt{Site}.

\note \tt{Site_} is a synonym for \tt{struct Site} having three
members: \tt{func}, \tt{file}, and \tt{line}; \tt{func} and \tt{file}
are pointers to \tt{Char}, whereas \tt{line} is of type \tt{Int}.
Both the type and the synonym pair are defined in \tt{<assert._>}.

\head{Semantics}

\tt{pre_} invokes \tt{pre_}$n$\_ if the
expanded argument sequence contains $n$ arguments.

If $condition$ compares equal to zero, \tt{pre_1_}
prints a diagnostic message of the following form:

\begin{center}
\tt{Pre-condition failed:} \it{condition}\tt{, function}
\it{function-identifier}\tt{, file}
\it{file-name}\tt{, line}
\it{line-number}\tt{.}\\\indent
\tt{Called from function}
\it{caller-function-identifier}\tt{, file}
\it{call-site-file-name}\tt{, line}
\it{call-site-line-number}\tt{.}
\end{center}

\it{caller-function-identifier}, \it{call-site-file-name}, and
\it{call-site-line-number} are respectively obtained from \tt{_site.func},
\tt{_site.file}, and \tt{_site.line}; if either \tt{_site.func} or
\tt{_site.file} is a null pointer, an empty string is used instead.

\tt{pre_2_} and \tt{pre_3_} print \it{text}
instead of \it{condition} in the message.
\tt{pre_1_} and \tt{pre_2_} obtain \it{function-identifier}, \it{file-name},
and \it{line-number} from \tt{__func__}, \tt{__FILE__}, and \tt{__LINE__}
(respectively), whereas \tt{pre_3_} obtains these ``source code
coordinates'' from \it{site}; if \tt{(}$site$\tt{).func} or
\tt{(}$site$\tt{).file} is null, an empty string is used instead.

In all cases, the diagnostic message is written to the standard error stream
\tt{stderr}, and the process terminates as if by calling \tt{exit(1)}.
\it{text} and \it{site} are always evaluated once;
their results are discarded if \it{condition} is non-zero.
