\tt{public} and \tt{private} are object-like macros;
the reference implementation defines them in \tt{<specifiers._>} header.
\tt{public}  expands to the keyword \tt{inline}, and
\tt{private} expands to the keyword \tt{static} (for internal linkage).

\note These macros are nothing but alternative names,
and they do not add anything new to the language.
However, one advantage of macros is that they can be easily undefined,
which can be convenient for certain purposes; for example, the reference
implementation provides inline definitions for several functions in header
files, using the macro \tt{public} instead of the keyword \tt{inline}.
The benefit is that the inline definitions can be made visible
in multiple translation units, which can be used by compilers
for static analysis and optimizations of function calls.
An external definition is still required for each such function, so the
file \tt{lib.c_} first includes the file \tt{<specifiers._>}, then undefines
the macro \tt{public}, and redefines it with an empty replacement text.
This ensures that when the header files containing \tt{public} function
definition are included in \tt{lib.c_}, they are no longer \tt{inline} definitions.
