The \tt{inline} keyword is used to suggest that the code
generated for calling a function should be ``efficient''.
For most purposes, efficiency is desirable in terms of runtime,
and if the generated code for a function definition is small, the code can be
substituted at call site(s), thereby avoiding the overhead of a function call.
However, substituting the code of another function at multiple call
sites can increase size of the object file; also, too many inline
substitutions inside a function can bloat its code to an extent
that the resulting code itself becomes unsuitable for inlining.

\enlargethispage*{\baselineskip}
\pagebreak

The \tt{inline} keyword is only a hint to the compiler (similar
to the \tt{register} keyword); however, even in the absence of an
explicit hint from the programmer, translators can still perform
inline substitutions at their own discretion, if the function
definition is visible in the same translation unit as the function call.
Increasing code size to gain speed (space-time tradeoff)
may not always work as expected, and large code expansion
can actually degrade performance (depending on several factors
of the execution environment, such as instruction caching).
In particular, minimizing the size of an executable is an
important concern for memory constrained devices, even if that
comes at the cost of a tolerable increase in execution time.
In several contexts, it may be desirable to have a portable
mechanism to explicitly disable inline substitutions at
specific call sites, without the use of compiler-specific flags.

\head{Syntax}

\tt{no_inline_ (} $function$ \tt{)}

\head{Semantics}

If $function$ is a function name or a function pointer, then invoking
the outcome of \tt{no_inline_} ensures that the call is not inlined,
even if an inline definition is visible in the translation unit.
The outcome is an expression that compares equal with $function$,
and it has the same function type as $function$; if $function$ is
a function pointer, then type of the outcome is the corresponding
function type obtained on dereferencing the pointer.
