\head{Syntax}

\tt{proxy_ (} \it{prefix} \tt{,} \it{method-name} \tt{)}

\head{Semantics}

\tt{proxy_} gives pointer to a function having
the same type as specified by \tt{Method_}.

The header \tt{<method._>} defines an object-like macro \tt{METHOD__} that
records the \tt{defined} state of \tt{DEBUG} macro every time \tt{<method._>} is
included: it expands to \tt{1} if \tt{DEBUG} was defined, and \tt{0} otherwise.
Conventionally, both parts of a method (protocol and procedure) are compiled in one
translation unit: if the macro \tt{METHOD__} expands to \tt{1} when the protocol
is defined, the function pointer specified by \tt{proxy_} refers to the protocol;
otherwise \tt{METHOD__} shall expand to \tt{1}, and the proxy points to a
\tt{private} function (internal linkage) that ignores the first argument
of type \tt{Site} and invokes the procedure with rest of the arguments.
Any return value is forwarded to the caller.

\note Unlike a protocol, function declaration of a procedure is not
modified to push an extra \tt{Site} parameter at the beginning:
this causes a mismatch between their function types.
\tt{proxy_} is an intermediary that provides a
common type for both the protocol and the procedure.
However, the protocol can be bypassed only if the translation
unit containing the method has been compiled in non-debugging mode,
$i.e.$ proxy points to a function that directly calls the procedure.
The only purpose of this extra function call
overhead is to remove the \tt{Site} argument.
