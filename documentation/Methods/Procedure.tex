\def\Subsection#1{\subsection{#1}\input{Methods/Procedure/#1}}

Procedure defines the solver that operates on the arguments
and possibly generates a return value (or side effects).
Unlike protocols, function definition (and declaration) of a procedure does
not push an extra \tt{Site} parameter at the start: this is because a
procedure is neither required nor supposed to know from where it is invoked.
The information provided by \tt{_site} is primarily meant for use by
the \tt{pre_} family whenever a pre-condition is not satisfied, in order
to track the call site that caused a violation; however, pre-conditions
(and also post-conditions) should be established only within protocols,
so there is no need to forward the original call site details to the procedure.

\note Should the need arise, programmers can supply call site details to
the procedure by explicitly specifying a parameter of type \tt{Site}, and also
providing a corresponding argument for the same when the function is called.

\head{Syntax}

\idx{procedure_}\s\s\s\s\s\tt{(} [\it{return-type} \tt{,}]\s\tt{(} \it{prefix}
\tt{,} \it{method-name} \tt{)} [\tt{,} \it{parameter-tuples}]\s\tt{)}\\

\idx{procedure_0_}\s\s\s\tt{(} [\it{return-type} \tt{,}]\s\tt{(} \it{prefix}
\tt{,} \it{method-name} \tt{)} \l\tt{,} \it{parameter-tuples}\r\s\tt{)}

\idx{procedure_0_1_}\s\tt{(} \l\it{return-type} \tt{,}\r\s\tt{(}
\it{prefix} \tt{,} \it{method-name} \tt{)} \l\tt{,} \it{parameter-tuples}\r\s\tt{)}

\idx{procedure_0_2_}\s\tt{(}   \tt{(} \it{prefix}
\tt{,} \it{method-name} \tt{)} \tt{,} \it{parameter-tuples} \tt{)}\\

\idx{procedure_1_}\s\s\s\tt{(} \tt{(} \it{prefix}
\tt{,} \it{method-name} \tt{)} \tt{)}\\

\idx{procedure_2_}\s\s\s\tt{(} \l\it{return-type} \tt{,}\r\s
\tt{(} \it{prefix} \tt{,} \it{method-name} \tt{)} \tt{)}

\idx{procedure_2_1_}\s\tt{(} \l\it{return-type} \tt{,}\r\s
\tt{(} \it{prefix} \tt{,} \it{method-name} \tt{)} \tt{)}

\tt {procedure_2_}\s\s\s\tt{(} \tt{(} \it{prefix} \tt{,} \it{method-name} \tt{)}
\tt{,} \tt{(} \it{parameter-type} \tt{,} \it{parameter-name} \tt{)} \tt{)}

\idx{procedure_2_2_}\s\tt{(} \tt{(} \it{prefix} \tt{,} \it{method-name} \tt{)}
\tt{,} \tt{(} \it{parameter-type} \tt{,} \it{parameter-name} \tt{)} \tt{)}

\head{Constraints}

The \tt{procedure_} family shall have precisely the same
constraints as those applicable for the \tt{protocol_} family.

\head{Semantics}

\tt{procedure_} family is used to provide function definition
for a solver; if \tt{private} (or equivalently \tt{static})
is specified before the associated \tt{protocol_} definition,
it means that the \tt{procedure_} definition has internal linkage.

\tt{procedure_} invokes \tt{procedure_0_} if the expanded argument sequence has
more than two arguments; otherwise it invokes \tt{procedure_}$n$\tt{_} if the
expanded argument sequence contains $n$ arguments, with $n$ not exceeding two.

The first argument is peeled and expanded: if the resulting list
has a single element, it is considered as \it{return-type};
otherwise it shall have two elements, \it{prefix} and
\it{method-name}, which are used to generate procedure identifier.

\tt{procedure_0_} invokes \tt{procedure_0_}$n$\_ if the first argument expands
to a list with $n$ elements on being subjected to the \tt{peel_} macro.
\tt{procedure_0_1_} defines a function that accepts a sequence of arguments as
specified by \it{parameter-tuples}, and returns a value of type
\it{return-type}; \tt{procedure_0_2_} uses \tt{Void_} as the return type.

\tt{procedure_1_} defines a \tt{Void_} function with a single parameter
\tt{_args_} of type \tt{VA_list_} (declared in \tt{<stdarg._>}
as a synonym for \tt{va_list}): it provides access to a variable
number of arguments given to the protocol or the proxy.

\tt{procedure_2_} invokes \tt{procedure_2_}$n$\_ if the first argument expands
to a list with $n$ elements on being subjected to the \tt{peel_} macro.
\tt{procedure_2_1_} defines a function having a single parameter \tt{_args_}
of type \tt{VA_list_}, and it returns a value of type \it{return-type}.
Return type of a \tt{procedure_2_2_} definition is \tt{Void_}.

\example The following procedure implements a
well-known adaptive version of bubble sort algorithm.

\code{../compile/Integers/bubble.c_}

In the above code, the array is conceptually divided into two parts:
the left side is assumed to be unsorted, and the right side is sorted.
The variable \tt{last_} marks the last index of the left side;
\tt{last_} is initialized to \tt{len - 1}, so the left side
spans the entire array, and the right side is initially empty.
The inner \tt{for} loop ``bubbles up'' the maximum element of the left
side to its end, by swapping consecutive elements that are out of order.
Once the maximum value is moved to the \tt{last_} index, it is augmented to the
right side by decreasing the value of \tt{last_}; this element will not exceed
any previous maximum that was bubbled up earlier, so the right side remains sorted.

As an enhancement, another variable \tt{omega_} keeps track of
the latest (highest) index at which a swapping was performed:
This implies that elements from \tt{omega_} to \tt{last_} are
already sorted, so they can be combined with the right side.
Doing this is trivial: we simply set \tt{last_} to \tt{omega_}
after completing each round of bubbling up the maximum.
Note that the assignment \tt{(last_ = omega_)} also acts the condition of the
\tt{do-while} loop, which stops when \tt{omega_} remains zero: this indicates
that the left side is empty, and the right side now spans the entire array.

\enlargethispage*{\baselineskip}
\enlargethispage*{\baselineskip}

The optional enhancement makes the algorithm ``adaptive''
by achieving linear time complexity in the best case:
if the array is already sorted, then no swapping is
performed, and \tt{omega_} remains zero after the first round of bubbling.
This indicates that elements from index \tt{omega_} (zero) to
\tt{last_} (\tt{len} - 1) are sorted, spanning the entire array.
After the first round, \tt{last_} is set to \tt{omega_},
which stops the \tt{do-while} loop as its condition becomes zero.

\pagebreak

\subsection{\idx{solver_}}
\head{Syntax}

\tt{solver_ (} \it{prefix} \tt{,} \it{method-name} \tt{)}

\head{Semantics}

\tt{solver_} gives mangled name of the function defined by
\tt{procedure_}; this function is also declared by \tt{protocol_}.

\note \tt{prototype_} family does not declare the identifier given by \tt{solver_}:
this is because a procedure can be defined with internal linkage, in which
case the function will not be visible outside of its own translation unit.


\Subsection{Multiple procedures}
