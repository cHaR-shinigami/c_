\def\Subsection#1{\subsection{#1}\input{Methods/Contracts/#1}}

The concept of protocols is nothing but a natural extension of the idea
behind function declarations: a declaration only specifies the return type
and parameter types of a function, for static type checking of function calls
and performing implicit argument promotions or conversions during translation.
A protocol describes additional checks to be performed
on the arguments and return value during execution.
In a sense, a protocol is a form of contract that expects
the caller to ensure certain prerequisites for the arguments,
and promises that the return value will meet certain criteria.
These requirements are established with the
help of pre-conditions and post-conditions.

The reference implementation provides \tt{<contract._>}
which includes two other headers: \tt{<pre._>} and \tt{<post._>}.
\tt{<pre._>} provides facilities for specifying pre-conditions,
and \tt{<post._>} provides facilities for writing post-conditions.

\note Pre-conditions are input-oriented,
whereas post-conditions are output-oriented.

\Subsection{Pre-conditions}

\Subsection{Post-conditions}

\Subsection{Example}
