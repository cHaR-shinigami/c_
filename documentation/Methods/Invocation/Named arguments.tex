For \tt{(} $prefix$ \tt{,} \it{method-name} \tt{)} style invocation of a method
whose prototype permits a fixed number of arguments, a named argument of the form
\tt{.}$parameter$ \tt{=} $argument$ is used to associate an argument with a parameter.
The idea is influenced by keyword arguments in the Python programming language,
and it can be used to provide arguments out of order, i.e. in a different
sequence as compared to the parameter list declared in the prototype.
If a named argument is followed by a conventional unnamed argument, the latter
corresponds to the subsequent parameter; further unnamed arguments after that
are assigned in the declared order of parameters as per the prototype.

If the argument corresponding to a parameter is provided multiple times,
only the last expression is considered, and rest of them are not evaluated;
if an argument is omitted for some parameter, the default
value \tt{0} is used instead (null for pointer type parameters).
The latter property can be convenient for augmenting a method with additional
parameters, without having to refactor the source code: this is because
existing invocations with \tt{call_} will still compile successfully, with
each newly added parameter at the end receiving the default argument \tt{0}.

\note A crucial point is that existing object files should not be linked with a
method extended with additional parameters, but their source codes should be
recompiled to generate object files; otherwise the default argument \tt{0} will
not be used for the extra parameters, potentially causing undefined behavior at runtime.
The \tt{call_} family comes with a mixed blessing that only avoids
refactoring the source code if more parameters are added to a method.
It still requires recompiling all translation units that invoke the method, based
on the updated prototype; failing to do so can silently break existing code.
As a general recommendation, an existing method prototype in wide
use should never be updated except in very rare circumstances,
and it is best to avoid such breaking changes altogether.
