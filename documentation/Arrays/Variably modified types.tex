Historically, C required array lengths to be compile-time constants.
C99 standardized the practice of variable length arrays (VLA) that allow array
definitions to specify a length that can be known at runtime; as a consequence,
applying the \tt{sizeof} operator on a VLA works during process execution,
and the operand is evaluated as well (side effects can be of concern).
VLAs with automatic storage duration are typically allocated on the stack,
which can quickly exhaust the stack space that is often very limited on most
execution environments; the chances are higher if a function that creates VLAs
is called recursively in contexts where tail call optimization is not possible.

C11 revised VLAs to be an optionally supported feature;
however, C23 mandates support for variably modified types.
The most common example of variably modified types
is pointer to VLA, which is heavily used in C\_.

\note Automatic creation of VLA continues to remain an optionally
supported feature, and for portability concerns, they are not used by
the reference implementation or the examples in this documentation.
