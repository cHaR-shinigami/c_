\head{Syntax}

\tt{# include <array._>}

\idx{at__}\s\s\s\s\tt{(} \it{pointer-to-array} \tt{,}
\it{index} [\tt{,} \it{default-value}] \tt{)}

\idx{at__2__}\s\tt{(} \it{pointer-to-array} \tt{,}
\it{index} \tt{)}

\idx{at__3__}\s\tt{(} \it{pointer-to-array} \tt{,}
\it{index} \phantom{[}\tt{,} \it{default-value}\phantom{]} \tt{)}

\head{Constraints}

\it{pointer-to-array} shall be pointer to a complete array.
\it{index} shall have an integer type.
It shall be possible to use \it{default-value} to initialize a variable
declared with element type of the array on dereferencing \it{pointer-to-array}.

\head{Semantics}

\tt{at__} invokes \tt{at__}$n$\_\_ if the
expanded argument sequence contains $n$ arguments.
Both \tt{at__2__} and \tt{at__3__} infer
the array bound as if by using \tt{length__}.
\it{index} is converted to the type \tt{Ptrdiff}:
if \it{index} is negative, then \tt{at__3__} adjusts it by adding
the (inferred) length; no such adjustment is done by \tt{at__2__}.
\it{default-value} is converted to the type of \tt{**(}\it{pointer-to-array}%
\tt{)}, as if by initializing a variable declared with the element type.

The (adjusted) index is compared with the length: if it is non-negative and
less than the length, then the array is accessed at that index, and the
corresponding element is the outcome; otherwise the index is out of bounds.

When compiled with \tt{ARRAY__} expanding to \tt{1},
one of the following cases can happen for index out of bounds:

\begin{itemize}

\item If \it{index} is negative after conversion to \tt{Ptrdiff},
\tt{at__2__} prints a diagnostic message of the following form:

\begin{center}
\tt{Assertion failed: (}\it{index}\tt{) >= 0, function}
\it{function-identifier}\tt{, file}
\it{file-name}\tt{, line}
\it{line-number}\tt{.}\\
\end{center}

\item If \it{index} is negative after conversion to \tt{Ptrdiff},
and it remains negative after adding length of the array,
\tt{at__3__} prints a diagnostic message of the following form:

\begin{center}
\tt{Assertion failed: (}\it{index}\tt{) >= -length_(}\it{pointer-to-array}\tt{),}

\tt{function} \it{function-identifier}\tt{,}
\tt{file} \it{file-name}\tt{,}
\tt{line} \it{line-number}\tt{.}
\end{center}

\item If \it{index} is greater than array length, \tt{at__2__} and
\tt{at__3__} print a diagnostic message of the following form:

\begin{center}
\tt{Assertion failed: (}\it{index}\tt{) < length_(}\it{pointer-to-array}\tt{),}

\tt{function} \it{function-identifier}\tt{,}
\tt{file} \it{file-name}\tt{,}
\tt{line} \it{line-number}\tt{.}
\end{center}

\end{itemize}

After writing the diagnostic message to the standard error stream \tt{stderr},
the process terminates as if by calling \tt{exit(1)}.
\it{function-identifier}, \it{file-name}, and \it{line-number}
are obtained from \tt{__func__}, \tt{__FILE__}, and \tt{__LINE__}.

When compiled with \tt{ARRAY__} expanding to \tt{0}, an invocation of
\tt{at__2__} is same as \tt{(*(}\it{pointer-to-array}\tt{))[}\it{index}\tt{]},
and the behavior is undefined if \it{index} is out of bounds.
If the adjusted index is out of bounds, then \it{default-value} is
used as the outcome after conversion to element type of the array.

The outcome is an lvalue having the same type as array element.
Dereference can be suppressed with address-of operator \tt{&}
on the outcome; however, this does not disable bounds-checking.
If \it{default-value} is used as the outcome of \tt{at__3__},
then the lvalue has automatic storage duration,
and its lifetime is limited to the nearest enclosing block.

\it{pointer-to-array} can be evaluated more than once only if
it has a variably modified type; both \it{index} and \it{default-value}
are evaluated exactly once, even if the latter value is unused.
The behavior is implementation-defined if \it{index} causes
a signed overflow during conversion to the type \tt{Ptrdiff}.
The behavior is undefined if the inferred length is
greater than the actual number of elements in the array,
and the (adjusted) index is outside the true array bounds.
