The \tt{range_} family is used to instantiate a \tt{Range}:
it creates a triplet array and gives a pointer to that array.

\head{Syntax}

\idx{range_}\s\s\s\tt{(} $stop$  \tt{)}

 \tt{range_}\s\s\s\tt{(} $alpha$ \tt{,} $omega$ [\tt{,} $delta$\tt{=1}] \tt{)}

\idx{range_1_}\s\tt{(} $stop$  \tt{)}

\idx{range_2_}\s\tt{(} $alpha$ \tt{,} $omega$ \tt{)}

\idx{range_3_}\s\tt{(} $alpha$ \tt{,} $omega$ \l\tt{,} $delta$\r \tt{)}

\head{Constraints}

$alpha$, $omega$, $delta$, and $stop$ shall be expressions having integer types.

\head{Semantics}

\tt{range_} invokes \tt{range_}$n$\_ if the
expanded argument sequence contains $n$ arguments.
\tt{range_3_} creates a non-modifiable array of integers initialized with
$alpha$, $omega$, and $delta$ (in order), and gives a pointer to that array.
Element type of the array is same as type of the
expression \tt{(}$alpha$\tt{) - (}$omega$\tt{)}.
If \tt{range_} family is used within a function, the array has automatic
storage duration, and its lifetime is limited to the nearest enclosing block.

\tt{range_1_(}$stop$\tt{)} is equivalent to
\tt{range_2_(0, (}$stop$\tt{) - 1)}.

\tt{range_2_(}$alpha$\tt{,} $omega$\tt{)} is equivalent to
\tt{range_3_(}$alpha$\tt{,} $omega$\tt{, 1)}.

\note The value of $delta$ is converted to type of the
expression \tt{(}$alpha$\tt{) - (}$omega$\tt{)}.
