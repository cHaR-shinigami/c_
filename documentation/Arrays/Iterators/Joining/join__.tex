\head{Syntax}

\tt{# include <join._>}

\idx{join__}\s\s\s\s\tt{(} $sentence$
[\tt{,} $separator$ [\tt{,} $range$]] \tt{)}

\idx{join__1__}\s\tt{(} $sentence$ \tt{)}

\idx{join__2__}\s\tt{(} $sentence$
\phantom{[}\tt{,} $separator$\phantom{]} \tt{)}

\idx{join__3__}\s\tt{(} $sentence$
\phantom{[}\tt{,} $separator$ \phantom{[}\tt{,} $range$\phantom{]]} \tt{)}

\head{Constraints}

$sentence$ shall be a pointer to \tt{Sentence} or \tt{WSentence}.

If $sentence$ is a pointer to \tt{Sentence}, then $separator$
shall be of type \tt{String} or pointer to \tt{Char};\\
otherwise $sentence$ is a pointer to \tt{WSentence}, and
$separator$ shall be of type \tt{WString} or pointer to \tt{WChar}.

$range$ shall be an expression whose type is \tt{Range (Int)}.

\head{Semantics}

\tt{join__} invokes \tt{join__}$n$\_ if the
expanded argument sequence contains $n$ arguments.

\tt{join__1__} concatenates the (wide) strings pointed to by the elements
of $sentence$, with a space between every pair of consecutive elements.
\tt{join__3__} concatenates only the elements at
indices specified by $range$, and in the given order.
For \tt{join__2__} and \tt{join__3__}, $separator$
is placed between every pair of consecutive elements.
In all cases, a newline is placed after the concatenated (wide)
string, followed by a null (wide) character to mark the end.

If $sentence$ is a pointer to \tt{Sentence}, the outcome is a pointer to
\tt{String_} (modifiable array of characters); otherwise $sentence$ is a
pointer to \tt{WSentence}, and the outcome is a pointer to \tt{WString_}
(modifiable array of wide characters).
In both cases, the array allocation is obtained with \tt{malloc} (or equivalent),
and it can be deallocated by passing the pointer to  \tt{free};
the pointer is null if the required allocation is not available.
The array is of incomplete type, and if the outcome is not null, then the
array length can be obtained by calling \tt{strlen} or \tt{wcslen} function.

If an element of $sentence$ is a null pointer, that element is ignored.
The behavior is undefined if $separator$ or any non-null element of $sentence$
points to a (wide) character array that is not terminated by a null (wide) character.

When compiled with \tt{JOIN__} expanding to \tt{1}, only the argument
pointers are asserted to be not null, as done by \tt{notnull__}.
When compiled with \tt{JOIN__} expanding to \tt{0},
the outcome is null if $sentence$ is null.
If $separator$ is null, a (wide) space is placed between elements.
If $range$ is null, all non-null elements of $sentence$ array are joined.

\note If the element type of $sentence$ array is a pointer
to \tt{Char_} or \tt{WChar_}, then $sentence$ should be cast
to \tt{Ptr (Sentence)} or \tt{Ptr (WSentence)}; the cast is
safe and it is used to avoid qualifier mismatch in element type.
