\head{Syntax}

\tt{# include <join._>}

\idX{join_}\s\s\s\tt{(} $sentence$
 [\tt{,} $separator$  [\tt{,} $range$]] \tt{)}

\idX{join_1_}\s\tt{(} $sentence$ \tt{)}

\idX{join_2_}\s\tt{(} $sentence$
\l\tt{,} $separator$\r \tt{)}

\idX{join_3_}\s\tt{(} $sentence$
\l\tt{,} $separator$ \l\tt{,} $range$\r\r\ \tt{)}

\head{Constraints}

The \tt{join_} family shall have precisely the same
constraints as those applicable for the \tt{join__} family.

\head{Semantics}

\tt{join_} family evaluates each expression exactly only once;
rest of the semantics are same as \tt{join__} family.

\note A ``two-dimensional'' array of characters or wide
characters cannot be used with the \tt{join__} and \tt{join_}
families, which expect pointer to an array of pointers
(an ``array of arrays'' is different from an ``array of pointers'').
