The header \tt{<map._>} defines the macro \tt{MAP__} that configures
the behavior of \tt{map_} family; \tt{MAP__} records the \tt{defined}
state of \tt{DEBUG} macro every time \tt{<map._>} is included:
it expands to \tt{1} if \tt{DEBUG} was defined, and \tt{0} otherwise.

\head{Syntax}

\tt{# include <map._>}

\idx{map_}\s\s\s\tt{(} [$destination$ \tt{,}]
$function$ \tt{,} $source$ [\tt{,} $range$] \tt{)}

\idx{map_2_}\s\tt{(} $function$ \tt{,} $source$ \tt{)}

\idx{map_3_}\s\tt{(} \phantom{[}$destination$ \tt{,}\phantom{]}
$function$ \tt{,} $source$ \tt{)}

\idx{map_4_}\s\tt{(} \phantom{[}$destination$ \tt{,}\phantom{]}
$function$ \tt{,} $source$ \phantom{[}\tt{,} $range$\phantom{]} \tt{)}

\head{Constraints}

Both $source$ and $destination$ shall be pointers to complete arrays.

For \tt{map_2_}, $source$ array shall be modifiable;
for \tt{map_3_} and \tt{map_4_}, $destination$ array shall be modifiable.

$function$ shall be a function type expression that can be called with an
element of $source$ array with requiring any type cast; return type of the
function shall be suitable to store the return value in $destination$ without
any type casting (for \tt{map_2_}, $source$ is itself the destination).
$range$ shall be an expression having a \tt{Range} type.

\head{Semantics}

\tt{map_} invokes \tt{map_}$n$\_ if the
expanded argument sequence contains $n$ arguments.

For each element in $source$, \tt{map_2_} invokes $function$ with that element
as the argument, and the return value of that invocation replaces the element
in array; the sequence of invocation and replacement is implementation-defined.

\tt{map_3_} is similar to \tt{map_2_}, except that the return values
are stored in $destination$, and $source$ can be non-modifiable.
If $destination$ has length $n$ and $n$ is not greater than length of $source$,
only the first $n$ elements of $source$ are mapped to $destination$; otherwise
$destination$ is longer than $source$, and extra elements are not modified.

\tt{map_4_} invokes $function$ only for the index
series specified by $range$, and in the given order.
For each selected element of $source$, the return value is stored at
the same index in $destination$; rest of the elements are not modified.

\note \tt{map_2_(}$f$\tt{,} $source$\tt{)} is equivalent
to \tt{map_3_(}$source$\tt{,} $f$\tt{,} $source$\tt{)},
except that $source$ is evaluated once.
