The header \tt{<fold._>} defines a macro \tt{FOLD__} that configures
the behavior of \tt{fold_} family; \tt{FOLD__} records the \tt{defined}
state of \tt{DEBUG} macro whenever \tt{<fold._>} is included:
it expands to \tt{1} if \tt{DEBUG} was defined, and \tt{0} otherwise.

\head{Syntax}

\tt{# include <fold._>}

\idx{fold_}\s\s\s\tt{(} $accumulator$ \tt{,}
$function$ \tt{,} $source$ [\tt{,} $range$] \tt{)}

\idx{fold_3_}\s\tt{(} $accumulator$ \tt{,}
$function$ \tt{,} $source$ \tt{)}

\idx{fold_4_}\s\tt{(} $accumulator$ \tt{,}
$function$ \tt{,} $source$ \l\tt{,} $range$\r\ \tt{)}

\head{Constraints}

$source$ shall be pointer to a complete array type.
$range$  shall be an expression having a \tt{Range} type.

$function$ shall be a function type expression that can be called with
$accumulator$ as the first argument and an element of $source$ as the
second argument, without requiring any type cast for each argument.

$accumulator$ shall be a modifiable lvalue that can be assigned with the
return value of $function$ without any type cast; additionally, it shall be
possible to obtain a pointer to $lvalue$ with the address-of operator \tt{&}.

\enlargethispage*{\baselineskip}

\head{Semantics}

\tt{fold_} invokes \tt{fold_}$n$\_ if the
expanded argument sequence contains $n$ arguments.
If the length of $source$ array is inferred to be $n$, \tt{fold_2_}
invokes $function$ $n$ times, with $accumulator$ as the first argument
and an element of $source$ as the second argument: the first invocation
is done with the element at index zero in $source$; subsequent
invocations use the element next to the one for the previous iteration.
For each invocation, the return value of $function$ is stored in
$accumulator$ itself, which is then used in the next iteration (if any).

\tt{fold_4_} is similar to \tt{fold_3_}, except that it invokes
$function$ only for the index series specified by $range$.

\enlargethispage*{\baselineskip}
\pagebreak
