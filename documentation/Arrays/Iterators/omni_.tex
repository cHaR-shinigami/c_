The header \tt{<omni._>} defines the macro \tt{OMNI__} that configures
the behavior of \tt{omni_} family; \tt{OMNI__} records the \tt{defined}
state of \tt{DEBUG} macro whenever \tt{<omni._>} is included:
it expands to \tt{1} if \tt{DEBUG} was defined, and \tt{0} otherwise.

\head{Syntax}

\tt{# include <omni._>}

\idx{omni_}\s\s\s\tt{(} [$destination$ \tt{,}]
\it{left-operand} \tt{,} $function$ \tt{,} \it{right-operand} \tt{)}

\idx{omni_3_}\s\tt{(} \l\it{left-operand} \tt{,}\r\
$function$ \tt{,} \it{right-operand} \tt{)}

\idx{omni_4_}\s\tt{(} \l$destination$ \tt{,}\r\
\it{left-operand} \tt{,} $function$ \tt{,} \it{right-operand} \tt{)}

\head{Constraints}

$destination$, \it{left-operand}, and \it{right-operand} shall have
scalar types: if the type is a pointer type, then it shall be pointer
to a complete array; otherwise the type shall be an arithmetic type.
\it{left-operand} (for \tt{omni_3_}) and $destination$ (for \tt{omni_4_})
shall be pointer to a modifiable array, or a modifiable lvalue whose
address can be obtained with the \tt{&} operator, interpreted as
pointer to an array having a single element of arithmetic type.

$function$ shall be a function type expression that can be called with an element
of \it{left-operand} as the first argument and an element of \it{right-operand}
as the second argument, without requiring type cast for each argument.

\head{Semantics}

\tt{omni_} invokes \tt{omni_}$n$\_ if the
expanded argument sequence contains $n$ arguments.
These features are modeled after  \tt{omni__} macro from the ellipsis
framework (provided by the header \tt{<templates._>}), and their operational
semantics are identical (the precise behavior is described in chapter 4).
If $destination$, \it{left-operand}, or \it{right-operand} has an
arithmetic type, it is interpreted as an array having a single element.
\tt{omni_3_} stores the return values in \it{left-operand} itself, whereas
\tt{omni_4_} stores them in $destination$ (so \it{left-operand} can be pointer
to a non-modifiable array or an arithmetic expression that is not an lvalue).
If length of $destination$ is $n$ and $n$ is less than length of
\it{right-operand}, then only the first $n$ elements of \it{right-operand}
are used; all elements of \it{right-operand} are utilized.

\note If \it{left-operand} array is shorter than \it{right-operand} array,
then elements will be reused from \it{left-operand} in a round-robin sequence
(going back to the first element after using the last one);
however, \tt{omni_4_} stores the return values in $destination$, so the earlier
results will be used instead of the original value(s) from \it{left-operand}.
