\tt{<filter._>} defines the macro \tt{FILTER__} that configures the
behavior of \tt{filter_} family;  \tt{FILTER__} records the \tt{defined}
state of \tt{DEBUG} every time \tt{<filter._>} is included:
it expands to \tt{1} if \tt{DEBUG} was defined, and \tt{0} otherwise.

\head{Syntax}

\tt{# include <filter._>}

\idx{filter_}\s\s\s\tt{(} $destination$ \tt{,}
$predicate$ \tt{,} $source$ [\tt{,} $range$] \tt{,} $key$ \tt{)}

\idx{filter_4_}\s\tt{(} $destination$ \tt{,}
$predicate$ \tt{,} $source$ \phantom{[}\tt{,} $key$\phantom{]} \tt{)}

\idx{filter_5_}\s\tt{(} $destination$ \tt{,}
$predicate$ \tt{,} $source$ \phantom{[}\tt{,} $range$\phantom{]} \tt{,} $key$ \tt{)}

\head{Constraints}

Both $source$ and $destination$ shall be pointers to complete arrays.
$destination$ array shall be modifiable, and its element type shall be
suitable to store an element of $source$ array without any type cast.

$predicate$ shall be a function type expression that can be called with an element
of $source$ as the first argument, with $key$ being the subsequent argument(s)
without any type cast; return type of $predicate$ shall be a scalar type.

$range$ shall be an expression having a \tt{Range} type.

If $key$ needs to specify multiple arguments for calling $predicate$,
it shall be a fully parenthesized list for \tt{filter_}.

\head{Semantics}

\tt{filter_} invokes \tt{filter_}$n$\_ if the
expanded argument sequence contains $n$ arguments.
$key$ is subjected to the \tt{peel_} macro before counting the
number of elements in it using \tt{COUNT_}: if the count is one,
then $key$ is used without peeling; otherwise the resulting text
after applying \tt{peel_} is used, which can be a list of arguments.

\tt{filter_4_} calls $predicate$ for each element of $source$:
in each invocation, the element is the first argument, followed
by $key$ (peeled in case it expands to multiple arguments).
If the return value is non-zero, then that element is copied
to $destination$; if $destination$ gets full before reaching
the end of $source$, subsequent iterations are skipped.

\tt{filter_5_} calls $predicate$ only for elements in the indices
specified by $range$, as long as $destination$ is not full.

\enlargethispage*{\baselineskip}
\pagebreak
