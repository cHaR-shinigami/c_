The header \tt{<rel._>} defines the macro \tt{REL__} that configures
the behavior of \tt{rel_} family; \tt{REL__} records the \tt{defined}
state of \tt{DEBUG} macro whenever \tt{<rel._>} is included:
it expands to \tt{1} if \tt{DEBUG} was defined, and \tt{0} otherwise.

\head{Syntax}

\tt{# include <rel._>}

\idx{rel_}\s\s\s\tt{(} $flag$ \tt{,}
\it{unary-operator}  \tt{,} \it{operand} \tt{)}

 \tt{rel_}\s\s\s\tt{(} $flag$ \tt{,} \it{left-operand} \tt{,}
\it{binary-operator} \tt{,} \it{right-operand} \tt{)}

\idx{rel_3_}\s\tt{(} $flag$ \tt{,}
\it{unary-operator}  \tt{,} \it{operand} \tt{)}

\idx{rel_4_}\s\tt{(} $flag$ \tt{,} \it{left-operand} \tt{,}
\it{binary-operator} \tt{,} \it{right-operand} \tt{)}

\head{Constraints}

$flag$ shall be a modifiable lvalue of arithmetic type.
\it{left-operand} and \it{right-operand} shall have scalar types:
if the type is a pointer type, it shall be pointer to a
complete array; otherwise the type shall be an arithmetic type.

For \tt{rel_3_}, it shall be possible to use \it{binary-operator} with two
consecutive elements of $operand$; for \tt{rel_4_}, \it{binary-operator} shall
not cause semantic error when an element of \it{left-operand} is placed before
the operator, and an element of \it{right-operand} is placed after the operator.
An expression with \it{binary-operator} shall have scalar type.

\note \it{binary-operator} is intended to be a relational or
equality operator, but it is not imposed as a constraint.

\head{Semantics}

\tt{rel_} invokes \tt{rel_}$n$\_ if the
expanded argument sequence contains $n$ arguments.
These features are modeled after  \tt{rel__} family from the ellipsis
framework (provided by the header \tt{<templates._>}), and their operational
semantics are identical (the precise behavior is described in chapter 4).
If $operand$, \it{left-operand}, or \it{right-operand} has an arithmetic type,
then it is interpreted as an array having a single element.
Only the lvalue $flag$ is modified.

If $operand$ has a single element, then
\tt{rel_3_} sets $flag$ to \tt{TRUE_()} (one).
If the length of $operand$ is inferred to be $n$ and $n$ is greater than
one, then every pair of consecutive elements is used as the operands of
\it{binary-operator} as long as the outcome of the operation is non-zero.
If an outcome is zero, then rest of the iterations are skipped,
and $flag$ is set to \tt{FALSE_()} (zero); otherwise all the
$n - 1$ outcomes are non-zero, and $flag$ is set to \tt{TRUE_()}.

If both \it{left-operand} and \it{right-operand} have a single
element each, then \tt{rel_4_} sets $flag$ to \tt{TRUE_()};
otherwise let $n$ be the maximum length of the two arrays.
At most $n$ iterations are performed, and in each iteration,
\it{binary-operator} is used with an element of \it{left-operand}
and an element of \it{right-operand}: if the outcome is zero,
then subsequent iterations are skipped and $flag$ is set to \tt{FALSE_()};
otherwise all the $n$ outcomes are non-zero and $flag$ is set to \tt{TRUE_()}.
Whenever a shorter array is exhausted,
its elements are reused in a round-robin sequence.

\note A non-trivial assignment to $flag$ is the logical
conjunction of all the outcomes for \it{binary-operator}.
