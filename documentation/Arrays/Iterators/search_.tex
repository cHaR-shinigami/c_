\tt{<search._>} defines the macro \tt{SEARCH__} that configures the
behavior of \tt{search_} family;  \tt{SEARCH__} records the \tt{defined}
state of \tt{DEBUG} every time \tt{<search._>} is included:
it expands to \tt{1} if \tt{DEBUG} was defined, and \tt{0} otherwise.

\head{Syntax}

\tt{# include <search._>}

\idx{search_}\s\s\s\tt{(} \it{found-at} \tt{,}
$predicate$ \tt{,} $source$ [\tt{,} $range$] \tt{,} $key$ \tt{)}

\idx{search_4_}\s\tt{(} \it{found-at} \tt{,}
$predicate$ \tt{,} $source$ \l\tt{,} $key$\r\ \tt{)}

\idx{search_5_}\s\tt{(} \it{found-at} \tt{,}
$predicate$ \tt{,} $source$ \l\tt{,} $range$\r\ \tt{,} $key$ \tt{)}

\head{Constraints}

\it{found-at} shall be a modifiable lvalue of arithmetic type,
and $source$ shall be pointer to a complete array.

$predicate$ shall be a function type expression that can be called with an element
of $source$ as the first argument, with $key$ being the subsequent argument(s)
without any type cast; return type of $predicate$ shall be a scalar type.

$range$ shall be an expression having a \tt{Range} type.

If $key$ needs to specify multiple arguments for calling $predicate$,
it shall be a fully parenthesized list for \tt{search_}.

\head{Semantics}

\tt{search_} invokes \tt{search_}$n$\_ if the
expanded argument sequence contains $n$ arguments.
$key$ is subjected to the \tt{peel_} macro before counting the
number of elements in it using \tt{COUNT_}: if the count is one,
then $key$ is used without peeling; otherwise the resulting text
after applying \tt{peel_} is used, which can be a list of arguments.

\tt{search_4_} calls $predicate$ for each element of $source$,
starting at index zero and moving towards higher indices:
in each invocation, the element is the first argument, followed
by $key$ (peeled in case it expands to multiple arguments).
If a return value is non-zero, then that index is copied to \it{found-at},
and rest of the iterations are skipped; otherwise $predicate$ returns zero for
all elements of $source$, and \it{found-at} is set to the length of $source$.

\tt{search_5_} calls $predicate$ only for elements in the indices specified
by $range$: \it{found-at} is set to the first index (as per the sequence
specified by $range$) for which $predicate$ returns non-zero, skipping rest
of the index sequence; otherwise $predicate$ returns zero for all elements
in the given $range$, and \it{found-at} is set to the length of $source$.

\note If $predicate$ returns non-zero, \it{found-at} is set
to a non-negative index, even if $range$ has negative values.
