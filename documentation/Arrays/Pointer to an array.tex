Pointer to an array refers to the same address as pointer to its base element;
however, pointer to array has a fundamentally different type from
pointer to base element, and this distinction is important for address
arithmetic with pointers: adding one with ``pointer to an array''
makes it point to the next ``array'', not to the next element.
One advantage of using pointer to array type is that
it not implicitly ``decay'' to a base element pointer.

Pointer to an array is also a natural extension of pointer to other object types,
and using pointer to array as a function parameter clearly conveys the intent
that some function expects (pointer to) an array, not pointer to a single element.
For instance, the standard library function \tt{fputs} can be declared as
\tt{int fputs(Char *, File_ *)} where the first parameter expects an array of
characters, and the second parameter expects pointer to a single \tt{File_} object.
The distinction of array and pointer is not evident from the
declaration itself, which can be achieved with pointer to arrays
(doing so with existing functions would change their type).

As an example,
\tt{strcpyn(Size length, Ptr (Char_ [length]) target, Ptr (Char_ []) source)}
declares both \tt{target} (destination) and \tt{source} as pointers to arrays:
\tt{target} expects pointer to a complete array having a capacity of \tt{length}
elements, whereas source is a pointer to an incomplete array whose length is
not specified in the type (it is determined by looking for the first null byte).
Most features of C\_ that work with arrays expect a pointer to a complete array,
so that length information can be extracted from the pointer itself.
