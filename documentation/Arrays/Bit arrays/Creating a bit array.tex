The \tt{bits_} family can be used to create a dynamically resizeable bit array.

\head{Syntax}

\idx{bits_}\s\s\s\tt{(} $bitcount$ [\tt{,} $initializer$] \tt{)}

\idx{bits_1_}\s\tt{(}   $bitcount$ \tt{)}

\idx{bits_2_}\s\tt{(}   $bitcount$ \l\tt{,} $initializer$\r\ \tt{)}

\head{Constraints}

$bitcount$ shall be an expression having integer type,
and $initializer$ shall be a scalar expression.

\head{Semantics}

\tt{bits_} invokes \tt{bits_}$n$\_ if the
expanded argument sequence contains $n$ arguments.
On success, the outcome is a pointer to \tt{Bits_},
whose length (number of words) is given by \tt{wordcount_(}$bitcount$\tt{)};
the outcome is a null pointer if the required allocation cannot be obtained.
If $initializer$ compares equal to zero, then all bits are zeroed out (reset);
otherwise $initializer$ is non-zero, and all bits are set to one.
The array can be deallocated by passing the pointer to \tt{free}, and it can be
resized with \tt{realloc}, or with the help of \tt{renew__}/\tt{renew_}* families.

\note If the converted value of $bitcount$ is not a multiple
of \tt{BITS_WIDTH}, there will be extra bits at the end.
