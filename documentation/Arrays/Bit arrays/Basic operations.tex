The basic operations on bit arrays include getting the bit value
at a given bit index, clearing a bit, and setting it to \tt{1};
these facilities are respectively provided by \tt{bit_}, \tt{rst_},
and \tt{set_}, all of which can be configured with the \tt{DEBUG} macro.
The macro \idx{BITS__} records the \tt{defined} state of \tt{DEBUG}
when the header \tt{<bits._>} is included: \tt{BITS__} expands
to \tt{1} if \tt{DEBUG} was defined, configuring the features
in debugging mode; otherwise \tt{BITS__} expands to \tt{0}.

\head{Syntax}

\tt{# include <bits._>}

\idx{bit_} \tt{(} \it{pointer-to-bits} \tt{,} \it{bit-index} \tt{)}

\idx{rst_} \tt{(} \it{pointer-to-bits} \tt{,} \it{bit-index} \tt{)}

\idx{set_} \tt{(} \it{pointer-to-bits} \tt{,} \it{bit-index} \tt{)}

\head{Constraints}

\it{pointer-to-bits} shall be pointer to a complete array whose unqualified
element type is \tt{UInt_fast8_};\\for \tt{rst_} and \tt{set_}, the array shall be modifiable.
\it{bit-index} shall be an expression having integer type.

\head{Semantics}

Bit count is inferred from the type of \it{pointer-to-bits}.
\it{bit-index} is converted to type \tt{LLong}:
if it is negative, bit count is added to it to get the adjusted index.
If the adjusted index is non-negative and less than bit count, then:

\begin{itemize}[nosep]

\item \tt{bit_} gives the current bit value at the adjusted \it{bit-index}

\item \tt{rst_} sets the bit to \tt{0} at the adjusted \it{bit-index},
and gives the old bit value

\item \tt{set_} sets the bit to \tt{1} at the adjusted \it{bit-index},
and gives the old bit value

\end{itemize}

In all cases, the outcome is of type \tt{Bool_}.
When compiled with \tt{BITS__} expanding to \tt{1},
\it{pointer-to-bits} is asserted to be not null as if by using \tt{notnull_}.
If \it{bit-index} is out of bounds, one of the following cases can happen:

\begin{itemize}[nosep]

\item If \it{bit-index} remains negative after adding the inferred bit count,
the following diagnostic message is printed:

\begin{center}
\tt{Assertion failed: (}\it{bit-index}\tt{) >= -bitcount_(}\it{pointer-to-bits}\tt{),}

\tt{function} \it{function-identifier}\tt{,}
\tt{file} \it{file-name}\tt{,}
\tt{line} \it{line-number}\tt{.}
\end{center}

\item If \it{bit-index} is not less than the inferred bit count,
the following diagnostic message is printed:

\begin{center}
\tt{Assertion failed: (}\it{bit-index}\tt{) < bitcount_(}\it{pointer-to-bits}\tt{),}

\tt{function} \it{function-identifier}\tt{,}
\tt{file} \it{file-name}\tt{,}
\tt{line} \it{line-number}\tt{.}
\end{center}

\end{itemize}

After writing the diagnostic message to the standard error stream \tt{stderr},
the process terminates as if by calling \tt{exit(1)}.
\it{function-identifier}, \it{file-name}, and \it{line-number}
are obtained from \tt{__func__}, \tt{__FILE__}, and \tt{__LINE__}.

When compiled with \tt{BITS__} expanding to \tt{0}, the outcome of
\tt{bit_}, \tt{rst_}, and \tt{set_} is zero if \it{pointer-to-bits}
is null, or the adjusted value of \it{bit-index} is out of bounds.
The behavior is undefined if the inferred bit count
is greater than the actual number of bits in the array,
and the adjusted \it{bit-index} is outside the true array bounds.
