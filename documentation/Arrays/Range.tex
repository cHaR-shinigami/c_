A \type{Range} type specifies a pointer type to a non-modifiable
integer array of three elements: $alpha$, $omega$, and $delta$.
This triplet encodes an arithmetic sequence that begins at $alpha$ and does not go
beyond $omega$, with $delta$ being the difference between two consecutive members.
The last member of the series is of the form $alpha + k*delta$, where $k$ is the
maximum non-negative integer for which the last member does not go beyond $omega$.

\note If $omega - alpha$ is a multiple of $k$,
then the sequence ends with $omega$ as its last member.

\head{Syntax}

\tt{Range}\s\s\tt{(} $type$ \tt{)}

\tt{Range_}\s\tt{(}  $type$ \tt{)}

\head{Constraints}

$type$ shall specify an integer type.

\head{Semantics}

Both \tt{Range} and \tt{Range_} specify pointer types to a non-modifiable
array of three elements. $type$ is subjected to integer promotion rules,
and non-modifiable form of the promoted type becomes the array element type.

\note \tt{Range_} means that the pointer itself is modifable,
but the array it points to remains non-modifiable.

\subsection{\tt{range_}}
\input Arrays/Range/range_

\subsection{\tt{alpha_}, \tt{omega_}, \tt{delta_}}
\input Arrays/Range/greek_
