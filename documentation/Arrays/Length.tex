The number of elements in an array is termed as its length.
C\_ offers the facility to determine array length
using \tt{length__} and \tt{length_}*:
both expect pointer to a complete array, and length is inferred
from the pointer type.

\note An array type is not directly allowed in a cast, but pointer to an
array can be used instead; this can be used to provide incorrect length
information that is different from the actual number of elements in an array.
C does not offer any mechanism to invalidate such constructs, and the
cast itself is well-defined; trouble brews only when executing some
code that attempts to access the array outside its actual bounds.
Casting as (pointer to) an array of smaller size is always safe, since an array
of $n$ elements ($n > 0$) can act as an array of $n - 1$ elements as well.
For example, pointer to a variable declared as \tt{Int num;} can be cast
as pointer to an array: \tt{(Ptr (Int [1]))&num}; this works because a single
\tt{Int} can be interpreted as an array of \tt{Int} having a single element.

\subsection{\idx{length__}}
\input Arrays/Length/length__

\subsection{\idX{length_}*}
\input Arrays/Length/length_
