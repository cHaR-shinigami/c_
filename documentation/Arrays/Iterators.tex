\def\Subsection#1{\subsection{#1}\input{Arrays/Iterators/#1}}

The header \tt{<iterators._>} provides several features for iterating
over arrays, and applying a function or an operator to the elements.
These iterators are influenced by higher order functions from functional
programming, and their semantics are similar to the macro iterators from the
ellipsis framework; the fundamental difference is that the macro iterators
operate during preprocessing, but the array iterators operate during execution.
With the exception of \tt{join_} family, rest of the iterators are
statements, so they cannot be used within expressions under ISO C syntax.

\tt{<iterators._>} only aggregates several other headers that are described
in the subsections; each iterator family can be configured for debugging
by defining the \tt{DEBUG} macro before including the associated header.
The purpose of \tt{<iterators._>} is to ensure a uniform
configuration for all iterators; however, an iterator family can
be individually (re-)configured simply by (re-)including its associated
header, possibly preceded by an active definition of \tt{DEBUG}.

Each header that is included by \tt{<iterators._>} defines an object-like
macro for recording the \tt{defined} state of \tt{DEBUG} macro every time
that header is included: if that macro expands to \tt{1}, then the
corresponding iterator family is configured in debugging mode, and it
asserts that all pointer arguments are not null, as done by \tt{notnull_}.
Some of the iterator families accept a $range$ argument, which specifies
an arithmetic progression of indices; negative indices are adjusted
by adding length of the array that is inferred from the array type.
When compiled in debugging mode, an iterator asserts that
adjusted values of $alpha$ and $omega$ are within the index range,
in addition to checking that the $range$ pointer is not null.
If a $range$ assertion fails, then one of the
following diagnostic messages is printed:

\begin{center}
\tt{Assertion failed:} \tt{(}\it{range}\tt{) != NULL}\tt{, function}
\it{function-identifier}\tt{, file}
\it{file-name}\tt{, line}
\it{line-number}\tt{.}\\
\end{center}

\begin{center}
\tt{Assertion failed:} \tt{alpha_(}\it{range}\tt{) >= -length_(}\it{source}\tt{),}\\
\tt{function}
\it{function-identifier}\tt{, file}
\it{file-name}\tt{, line}
\it{line-number}\tt{.}\\
\end{center}

\begin{center}
\tt{Assertion failed:} \tt{alpha_(}\it{range}\tt{) < length_(}\it{source}\tt{),}\\
\tt{function}
\it{function-identifier}\tt{, file}
\it{file-name}\tt{, line}
\it{line-number}\tt{.}\\
\end{center}

\begin{center}
\tt{Assertion failed:} \tt{omega_(}\it{range}\tt{) >= -length_(}\it{source}\tt{),}\\
\tt{function}
\it{function-identifier}\tt{, file}
\it{file-name}\tt{, line}
\it{line-number}\tt{.}\\
\end{center}

\begin{center}
\tt{Assertion failed:} \tt{omega_(}\it{range}\tt{) < length_(}\it{source}\tt{),}\\
\tt{function}
\it{function-identifier}\tt{, file}
\it{file-name}\tt{, line}
\it{line-number}\tt{.}\\
\end{center}

\it{function-identifier}, \it{file-name}, and \it{line-number} are
respectively obtained from \tt{__func__}, \tt{__FILE__}, and \tt{__LINE__}.
Diagnostic message is written to standard error stream \tt{stderr},
and the process terminates as if by calling \tt{exit(1)}.

When compiled in non-debugging mode, if $alpha$ or $omega$ is less than
negative of the array length, it is adjusted to zero; if it is not less
than the array length, it is adjusted to one less than the array length.
$delta$ is then applied on the adjusted bounds.
It is possible that the resulting index series is
empty, in which case no iteration is performed.

\note If an iterator also accepts a $destination$ array,
length of the smaller array is used for $range$ adjustments.

\subsection{\tt{map_}}
\input Arrays/Iterators/map_

\subsection{\tt{fold_}}
\input Arrays/Iterators/fold_

\subsection{\tt{reduce_}}
\input Arrays/Iterators/reduce_

\subsection{\tt{omni_}}
\input Arrays/Iterators/omni_

\subsection{\tt{op_}}
\input Arrays/Iterators/op_

\subsection{\tt{rel_}}
\input Arrays/Iterators/rel_

\subsection{\tt{filter_}}
\input Arrays/Iterators/filter_

\subsection{\tt{search_}}
\input Arrays/Iterators/search_

\subsection{\tt{permute_}}
\input Arrays/Iterators/permute_

\Subsection{Joining}
